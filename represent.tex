% -*- root: main.tex -*-

\chapter{Representability}\label{RepChap}

\todo{This used to live in the obstruction theory section, where it no longer makes sense because we don't know what infinite loopspaces are.
It probably belongs somewhere in this chapter.}
% \begin{example}
% Another interesting case is when $Y = \OS{E}{0}$ is an infinite-loopspace.  This spectral sequence then recovers the Atiyah--Hirzebruch spectral sequence from the previous lecture.
% \end{example}




\section{Brown representability}

Early on, when we were getting used to the categorical approach to homotopy theory, we noted in \Cref{HomSheafRemark} that $\CatOf{Spaces}(-, T)$ forms a \define{sheaf} on the category $\CatOf{Spaces}$.
This isn't our only example of such an object: cohomology functors can also be thought of as sheaves, as the Eilenberg--Steenrod axioms include the sheaf axioms as a subset.
However, in the course of our study of the homotopy theory of CW--complexes, we discovered that these two examples are actually not separate: \Cref{OrdinaryCohIsRepresentable} gave a natural isomorphism \[\widetilde H^n(X; A) \xrightarrow\cong [X, K(A, n)],\] where $K(A, n)$ is an \define{Eilenberg--Mac Lane space} as in \Cref{EMSpacesExist}.
Today we prove that this is not an accident: all sheaves in the homotopy category are representable.

\begin{theorem}[Brown]\label{BrownRepThm}\marginnote{\citep[Theorem 9.12]{Switzer}}
Let $F\co h\CatOf{Spaces}^{\op}_{\mathrm{conn}, */} \to \CatOf{Sets}_{*/}$ be a functor on pointed, connected spaces which satisfies the following pair of axioms:
\begin{description}
    \item[Wedge axiom:] $F$ converts wedges to products, as in \[F(\bigvee_\alpha X_\alpha) \cong \prod_\alpha F(X_\alpha).\]
    \item[Gluing condition:] Elements in the image of $F$ glue.
    For a decomposition $X = A_1 \cup A_2$ and for elements $f_1 \in F(A_1)$, $f_2 \in F(A_2)$ which agree on the intersection, as in \[f_1|_{A_1 \cap A_2} = f_2|_{A_1 \cap A_2},\] then there exists a glued element $f \in F(X)$ which satisfies $f_1 = f|_{A_1}$ and $f_2 = f|_{A_2}$.
\end{description}
There then exists a \define{representing CW-complex} $Y$ and a \define{universal element} $u \in F(Y)$ such that the natural transformation
\begin{align*}
[T, Y] & \to F(T), \\
\phi & \mapsto \phi^*(u)
\end{align*}
is a natural bijection.
Moreover, there is a compatible bijection between natural transformations $F \to F'$ between such functors and homotopy classes $Y \to Y'$ between their representing objects.
\end{theorem}

\noindent We construct this in stages.

\begin{definition}\marginnote{\citep[Definition 9.6]{Switzer}}
An element $u \in Y$ is said to be \define{$n$--universal} if the associated natural transformation
\begin{align*}
[S^q, Y] & \to F(S^q) \\
\phi & \mapsto \phi^*(u)
\end{align*}
is surjective for $q \le n$ and bijective for $q < n$.
\marginnote{It follows that $Y$ will represent $F$ for all CW--complexes of dimension less than $n$.}
\end{definition}

\begin{lemma}\label{NUnivToNPlusOne}\marginnote{\citep[Lemma 9.8]{Switzer}}
If there exists an $n$--universal element, then there exists an $(n+1)$--universal element.
\end{lemma}
\begin{proof}
Suppose we have an $n$--universal element $u_n$ on a complex $Y$.
From this, we would like to construct an $(n+1)$--universal element $u_{n+1}$ on a complex $Y'$.
We set about trying to ``fix'' $[S^n, Y] \to F(S^n)$, which might have too many elements to be a bijection, and $[S^{n+1}, Y] \to F(S^{n+1})$, which might be missing some elements to be a bijection.
Note that because $S^n$ is an $H$--cogroup for $n \ge 1$, the map $[S^n, Y] \to F(S^n)$ is actually a map of groups.
It follows that if we merely ensure that this surjective map does not have a kernel, it will be an isomorphism.
\marginnote{This is exactly why we restricted attention to \emph{connected} spaces: it gives us control over all the fibers of the map $[S^n, Y] \to F(S^n)$, rather than just the fiber over the constant map.}

This inspires us to consider the defect sets
\begin{align*}
A & = \{\alpha \in \pi_n Y \mid \alpha^* u_n = 0\}, &
L & = F(S^{n+1}),
\end{align*}
\marginnote{Here we are being a bit glib: perhaps some items in $F(S^{n+1})$ can be expressed as pullbacks of $u_n$, but there's no harm in adding more things to make sure we hit.}
and form the mapping cone
\begin{align*}
\bigvee_{\alpha \in A} S^n_\alpha & \xrightarrow{\alpha} Y \vee \bigvee_{\lambda \in L} S^{n+1}_\lambda \to Y'.
\intertext{Applying $F$, we have}
0 = \bigvee_\alpha \alpha^*(u_n) & \mapsfrom u_n \vee \bigvee_\lambda \lambda,
\end{align*}
hence we can lift it to an element $u_{n+1} \in F(Y')$.
Since $Y'$ is formed from $Y$ using $(n+1)$--cells, it agrees with $Y$ on $\pi_{< n}$, hence it is $n$--universal.
As for $(n+1)$--universality, it is designed to fix the defect at $\pi_n$ exactly, and the wedge over $\lambda$ forces surjectivity at $\pi_{n+1}$.
\end{proof}

\begin{lemma}\label{ClassifiedMapsAlwaysExtend}%
\marginnote{\citep[Corollary 9.9, Lemma 9.11]{Switzer}}
Let $Y$ be a space with universal element $u$, let $(X, A)$ be a CW-pair, let $v \in F(X)$ be a choice of element, and let $g\co A \to Y$ be a cellular map which classifyies $v|_A$.
There then exists a cellular map which classifies $v$ and which extends $g$.
\end{lemma}
\begin{proof}[Proof idea]
We define a ``double mapping cylinder'' $T$ which consists of the space $X$, the space $Y$, and the space $A \times I$, so that the leading edge of $A \times I$ is sewn to its image in $X$ and the trailing edge of $A \times I$ is sewn to its image under $g$ in $Y$.
\marginnote{This is the ``homotopy pushout'' of $X \xleftarrow{i} A \xrightarrow{g} Y$.}
\todo{There's a picture here.}
This space has a decomposition into $A_1$, which consists of $X$ and half of the cylinder, and $A_2$, which consists of $Y$ and half of the cylinder.
Since $A_1$ and $A_2$ are respectively homotopy equivalent to $X$ and $Y$, we may respectively consider $u$ and $v$ as elements of $F(A_1)$ and $F(A_2)$.
Definitionally, they agree on $A_1 \cap A_2$ (i.e., on the cylinder, which is equivalent to $A$), and hence they give rise to the glued element $w \in F(T)$.
We can extend $T$ to a CW pair $(Y', T)$ with universal element $u'$ restricting to $w$ (and hence to $u$).
\marginnote{Just re-run \Cref{NUnivToNPlusOne}, starting instead with $T$ and $w$ rather than a point and the trivial class.}
We apply Whitehead's theorem to the induced weak equivalence $Y \to Y'$ to produce an inverse, and the composite gives the desired map: $X \to Y' \to Y$.
\end{proof}

\begin{proof}[{Proof of \Cref{BrownRepThm}}]
To get surjectivity, set $A = \{x_0\}$ and apply \Cref{ClassifiedMapsAlwaysExtend}.
To get injectivity, set $X' = X \times I$, $A' = X \times \partial I$, and apply \Cref{ClassifiedMapsAlwaysExtend} again.
To get the statement about natural transformations, one need only chase $\id$ through \[[Y, Y] \xrightarrow\cong F(Y) \xrightarrow{T} F'(Y) \xleftarrow\cong [Y, Y']\] to produce an element $f$.
\end{proof}

There is a useful companion result that works with functors $F$ defined only on \emph{finite} CW-complexes.

\begin{theorem}[Adams]\marginnote{\citep[Theorem 9.21]{Switzer}}
If $F$ is a functor to \emph{groups} from \emph{finite} CW-complexes satisfying the two conditions of Brown's theorem, then it is representable.
Natural transformations induce maps of representing objects that are unique up to \emph{weak homotopy}: restricting along an incoming map from \emph{any} finite complex gives two homotopic maps. \qed
\marginnote{
A key trick is to define $\widehat F(X) = \lim_\alpha F(X_\alpha)$.
This modification of $F$ satisfies the wedge axiom on the nose, it satisfies only a weak form of Mayer-Vietoris, but it gains the fact that the usual projection is an isomorphism: \[F(\lim{}_\alpha X_\alpha) \to \lim{}_\alpha F(X_\alpha).\]
}
\end{theorem}




\section{Spectra}

Our discussion last time was motivated by our observation in \Cref{EMSpacesExist} that there is a natural isomorphism \[H^n(X; A) \xrightarrow\cong [X, K(A, n)],\] which we generalized to any functor satisfying the wedge and Mayer--Vietoris axioms.
These axioms are most of what it means to be a cohomology theory.
The remaining axiom, which we have not yet discussed, is the following:

\begin{definition}[Suspension axiom]
There is a natural isomorphism
\marginnote{It's pleasing to write this identity as $\Susp \widetilde H^*(X) \cong \widetilde H^*(\Susp X)$.}
\[\widetilde H^n(X) \xrightarrow\cong \widetilde H^{n+1}(\Susp X).\]
\end{definition}

Its role is an interesting one, and it is best understood in the context of $\pi_*$, which we have shown in \Cref{FreudenthalThm} to \emph{partially} have this property.%
\marginnote{%
The fibration appearing in \Cref{PinSnWithoutHurewicz} for $n = 2$ has the form \[S^1 \to S^3 \to S^3.\]
From this we can conclude $S^3 \simeq S^2[3, \infty)$, and hence $\pi_3 S^2 \cong \Z$.
This gives a concrete counterexample to any extension of Freudenthal beyond the advertised range, as $0 \cong \pi_2 S^1 \not\cong \pi_3 S^2 \cong \Z$.
}
An interesting feature of \Cref{FreudenthalThm} is that the range in which suspension invariance holds improves the more times you suspend: if $\pi_* X \to \pi_* \Susp X$ is an isomorphism through degree $2n$, then $\pi_* \Susp X \to \pi_* \Susp^2 X$ is an isomorphism through degree $2(n+1)$, and so on.
It follows that we can associate to $X$ its \define{stable homotopy groups}, given by $\colim_n \pi_{*+n} \Susp^n X$.
A second interesting feature of \Cref{FreudenthalThm} is that a spectral sequence argument shows it to generalize away from spheres to CW--complexes of bounded dimension:

\begin{corollary}[{of \Cref{FreudenthalThm}}]
\todo{Find me a citation.  I don't seem to be in Switzer.}
Let $X$ be an $s$--connected CW--complex, and let $Y$ be a CW--complex of dimension $t$.
Then \[F(Y, X) \to F(\Susp Y, \Susp X)\] is a $(2s-t)$--equivalence. \qed
\end{corollary}

\noindent It follows again that $\pi_m F(\Susp^n Y, \Susp^n X)$ is independent of $n$ for $n \gg 0$.
This spurs us to make the following categorical definition:

\begin{definition}\marginnote{\citep[Example 8.2]{Switzer}}
Let $h\CatOf{SuspensionSpectra}$ denote the category which has an object $\Susp^\infty X$ for every pointed space $X$ and whose morphism sets%
\marginnote{A funny consequence of this definition is that $[\Susp^\infty Y, \Susp^\infty X]$ (and, later, $[E, E']$ generally) is \emph{always} an abelian group, since one can always take at least $2$ suspensions to be involved.}
are given by the formula \[[\Susp^\infty Y, \Susp^\infty X] := \colim_n [\Susp^n Y, \Susp^n X].\]
\end{definition}

One need not leave $\CatOf{Spaces}$ to understand this new category:
\begin{align*}
{[\Susp^\infty Y, \Susp^\infty X]} & := \colim_n [\Susp^n Y, \Susp^n X] \\
& = \colim_n [Y, \Loops^n \Susp^n X] \\
& = [Y, \colim_n \Loops^n \Susp^n X] =: [Y, QX].
\end{align*}
\marginnote{The map $h\CatOf{SuspensionSpectra} \to h\CatOf{Spaces}$ produced in this way is commonly denoted by $\Loops^\infty$.}
The stable homotopy groups then appear as honest homotopy groups: \[\pi_n \Susp^\infty X \cong \pi_n QX.\]

The functor $Q$ gives rise to a host of stable invariants: fixing a space $X$, the family of functors \[X^{-*}(-) := [-, Q \Susp^* X]\] satisfy the wedge and Mayer--Vietoris axioms (because the functors are representable) and suspension invariance (because $\Loops Q \Susp^{n+1} X = Q \Susp^n X$).
Unfortunately, not all cohomology theories arise in this way: there is generally no space $X$ so that $QX \simeq K(A, n)$.
This failure, however, is interesting on its own and measureable.
Consider the space $QK(A, n)$: \Cref{FreudenthalThm} shows that its homotopy is given by \[\pi_* K(A, n) \cong \begin{cases} 0 & \text{when $* < n$}, \\ A & \text{when $* = n$}, \\ 0 & \text{when $n < * \le 2n$}, \\ ??? & \text{otherwise}. \end{cases}\]
As $n$ increases, the range through which $K(A, n) \to QK(A, n)$ is an equivalence grows like $2n$.
If we were to permit ourselves to take the colimit in $n$ \emph{and} to shift spaces downward by desuspension, then we could write \[K(A, n) = \Loops^\infty \left( \colim_n \Susp^{-n} \Susp^\infty K(A, n) \right).\]
Here the colimit is taken along the maps $\Susp K(A, n) \to K(A, n+1)$ adjoint to $K(A, n) \to \Loops K(A, n+1)$.

\begin{definition}
A \define{spectrum} (up to homotopy) is an ind-diagram of formal desuspensions of suspension spectra.
\end{definition}

\begin{example}
Suspension spectra themselves qualify as spectra: $\Susp^\infty X$ is trivially an ind-diagram.
The \define{sphere spectrum} is the special case of $\S = \Susp^\infty S^0$.
\end{example}

\begin{example}
The \define{Eilenberg--Mac Lane spectrum} is given by the formula above: \[HA = \colim_n \Susp^{-n} \Susp^\infty K(A, n).\]
\end{example}

\begin{example}
The functor $X \mapsto (\pi_* \Susp^\infty X) \otimes \Z_{(p)}$ is exact, hence has an associated spectrum $\S_{(p)}$, the \define{$p$--local sphere spectrum}.
\todo{This is \emph{covariant}.  What is the correct statement for covariant Brown representation?
This is 14.35 in Switzer.  It's actually clear with spectra: replace $\pi_n E \sm X$ with $\pi_n F(DX, E)$ for finite $X$ and appeal to the finite form of Brown.  This absolutely should appear in the section on Spectra and homology theories.}
\end{example}

\begin{example}
The functor $X \mapsto \CatOf{AbGps}(\pi_0^s X, \Q/\Z)$ is exact, hence has an associated spectrum $\I$.  This is called the \define{Brown--Comenetz dualizing object}.
\end{example}

Given all this, one might be motivated by a need for concreteness\marginnote{Less perjoratively: a desire to build a bridge between these ideas and geometry.} to pursue a point-set model for spectra, maps of spectra, and homotopies among maps, whose homotopy category recovers $h\CatOf{Spectra}$.
There are many such models available, with competing strengths and deficiencies.
We reproduce one here, due to Boardman and Vogt.

\begin{definition}\marginnote{\citep[Definition 8.1]{Switzer}}
A \define{spectrum} is a collection $\{E_n\}_n$ of CW--complexes together with cellular maps $i_n\co \Susp E_n \to E_{n+1}$ which are homeomorphisms onto their images.
\marginnote{This is not so restrictive: given a suitable notion of homotopy equivalence, one may use the mapping cylinder construction to make the subcomplex and homeomorphism conditions apply.}
\end{definition}

As indicated by the ``ind--system'' appearing in the abstract definition, maps between spectra are not quite given by levelwise maps which commute with the inclusions $i_n$.%
\marginnote{%
For instance, we know $\pi_n S^n \cong \Z$ for $n \ge 1$, but $\pi_0 S^0 = \{\pm 1\}$.
If we were to define maps of spectra as such commuting sequences, then we would get $\pi_0 \Susp^\infty \S = \{\pm 1\}$---the \emph{wrong} answer.
}
Instead, one asks for maps to only be defined \emph{eventually}.

\begin{definition}\marginnote{\citep[Definitions 8.9, 8.10, 8.12, 8.15]{Switzer}}
A \define{subspectrum} $F \subseteq E$ is a sequence of subcomplexes of $E_n$, forming a spectrum by restriction.
It is \define{cofinal} when every cell $e^m_\alpha \subseteq E_n$ has $\Susp^{j_\alpha} e^m_\alpha \subseteq F_{n+j_\alpha}$---it eventually appears in $F$.
A map $E \to E'$ is required only to be defined on a cofinal $F \subseteq E$, and two maps are equal if they agree on a mutually cofinal subspectrum.%
\marginnote{In particular, the inclusion of a cofinal subspectrum is equivalent to the identity map.}
Finally, two maps of spectra are \define{homotopic} if there is a common cofinal subspectrum $F'$ and a map $F' \sm I_+ \to E'$ witnessing the homotopy.
\end{definition}

The advantage of having a model available is that we can use it to lift some familiar constructions from $\CatOf{Spaces}$.

\begin{lemma}\marginnote{\citep[8.17, 8.18]{Switzer}}
Spectra have wedge sums and mapping cones, both given level-wise. \qed
\end{lemma}

This, together with our knowledge of $\CatOf{Spaces}$ generally, is enough to copy the proof\marginnote{Which we did not give for $\CatOf{Spaces}$ either!} of Whitehead:

\begin{theorem}\marginnote{\citep[Corollary 8.24]{Switzer}}
If a map $f\co E \to E'$ induces a weak equivalence, it is a homotopy equivalence. \qed
\end{theorem}

\begin{corollary}\marginnote{\citep[Theorem 8.26]{Switzer}}
The spectra $\{E_n \sm S^1\}_n$ and $\{E_{n+1}\}_n$ are equivalent, and so the spectrum $\{E_{n-1} \sm S^1\}_n$ is equivalent to $E$. \qed
\end{corollary}

% Spectra are also set up to short-circuit Brown representability:

% \begin{lemma}
% $\CatOf{Spectra}(\Susp^\infty X, E) \cong \CatOf{Spaces}(X, \Loops^\infty E)$, where $\Loops^\infty E = \colim_k (\Loops^k E_k)$.
% \end{lemma}
% \begin{proof}
% Definitionally,
% \begin{align*}
% \CatOf{Spectra}(\Susp^\infty X, E) & \cong \colim{}_n \CatOf{Spaces}(\Susp^n X, E_n) \\
% & \cong \colim{}_n \CatOf{Spaces}(X, \Loops^n E_n) \\
% & \cong \CatOf{Spaces}(X, \colim{}_n \Loops^n E_n)
% \end{align*}
% when $X$ is CW.
% \end{proof}




\section{Co/homology theories from spectra}

We defined spectra in such a way that a cohomology theory gives rise to a spectrum by extracting the representing objects $E^n(-) = [-, E_n]$ and building from them the inductive system \[E := \colim_n \Susp^{-n} \Susp^\infty E_n.\]
Our definition was lax enough, though, that the converse is not quite as clear: do spectra precipitate cohomology theories?
If so, how tight is the correspondence between the two?

\begin{definition}\marginnote{\citep[8.33]{Switzer}}
For a spectrum $E$, we define its \define{associated (reduced) co/homology theories} as follows:
\begin{itemize}
    \item $\widetilde E^n(X) = [\Susp^\infty X, \Susp^n E]$.
    \item $\widetilde E_n(X) = \pi_n(E \sm X)$, where the smash product $(E \sm X)_n =  E_n \sm X$ is induced up from $\CatOf{Spaces}$.
\end{itemize}
\end{definition}

In order to see that these are co/homology functors, it's useful to record
\begin{lemma}
\marginnote{In general, homotopy classes of maps of spectra are presented by $\pi_0$ of a kind of pro-ind-space.}%
For $X$ a pointed space and $E$ a generic spectrum,%
\marginnote{If one uses Brown representability of a cohomology functor to manufacture the spaces $E_n$ in the definition of $E$, then suspension invariance makes the system in the Lemma \emph{constant}.  This is called an ``$\Omega$--spectrum''.}
\[[\Susp^\infty X, E] = \colim_{n,m} [\Susp^m X, \Susp^{m-n} E_n].\]
\end{lemma}
\begin{proof}
One couples the formula for suspension spectra \[[\Susp^\infty X, \Susp^\infty Y] = \colim_m [\Susp^m X, \Susp^m Y]\] to a presentation of $E$: \[E = \colim_n \Susp^{-n} \Susp^\infty E_n. \qedhere\]
\end{proof}

To feel confident in our definition, we should check that these functors indeed satisfy the Eilenberg--Steenrod axioms.
\begin{enumerate}
    \item We've built in suspension invariance:
    \begin{align*}
    \widetilde E_{n+1}(\Susp X) & \cong [S^{n+1}, E \sm \Susp X] \cong [S^n, E \sm X] \cong \widetilde E_n(X), \\
    \widetilde E^{n+1}(\Susp X) & \cong [\Susp^\infty \Susp X, \Susp^{n+1} E] \cong [\Susp^\infty X, \Susp^n E] \cong \widetilde E^n(X).
    \end{align*}
    \item A coexact sequence \[A \xrightarrow i X \to X \cup_i CA\] of pointed spaces induces a coexact sequence of spectra \[E \sm A \to E \sm X \to E \sm (X \cup_i CA) = (E \sm X) \cup_i C(E \sm A),\] so we get the desired long exact sequence
    \begin{center}
    \begin{tikzcd}
    \cdots \arrow{r} & \pi_n E \sm A \arrow{r} \arrow[equal]{d} & \pi_n E \sm X \arrow{r} \arrow[equal]{d} & \pi_n E \sm C(i) \arrow{r} \arrow[equal]{d} & \cdots \\
    \cdots \arrow{r} & \widetilde E_n(A) \arrow{r} & \widetilde E_n(X) \arrow{r} & \widetilde E_n(X, A) \arrow{r} & \cdots.
    \end{tikzcd}
    \end{center}
    For the analogous fact in cohomology, the sequence of suspension spectra \[\Susp^\infty A \to \Susp^\infty X \to \Susp^\infty (X \cup_i CA)\] is coexact, so mapping into $E$ makes it exact.
    \todo{We haven't talked about function spectra... so is this really fair?}
    \item The cohomological wedge axiom is easy: by pulling coproducts out on the left to products, we get \[\left[\bigvee_\alpha \Susp^\infty X_\alpha, E\right] = \prod_\alpha [\Susp^\infty X_\alpha, E].\]
    Homology is harder and requires a filtration trick.\marginnote{\citep[Lemma 8.34]{Switzer}}
    We know that our homology functor satisfies the \emph{finite} wedge axiom by appeal to the Mayer--Vietoris axiom.
    Smash products also commute with colimits, hence one may check \[E \sm \colim_{\substack{S \subseteq A \\ \text{$S$ finite}}} \bigvee_{\alpha \in S} X_\alpha \cong \colim_S E \sm \bigvee_{\alpha \in S} X_\alpha \cong \colim_S \bigvee_{\alpha \in S} E \sm X_\alpha \cong \bigvee_\alpha E \sm X_\alpha.\]
    \todo{This feels clumsy.  Is it really necessary?}
    From this, the wedge axiom follows.
\end{enumerate}

\begin{remark}
The Mayer--Vietoris axiom amounts to the assertion that co/homology functors commute with finite (homotopy) colimits, and the wedge axiom adds a special case on top of that.
Cohomology \emph{does not} commute with general colimits.%
\marginnote{\citep[Propositions 7.66 and 8.37]{Switzer}}
Instead, there is a \define{Milnor sequence}: \[0 \to R^1 \lim_\alpha E^{n-1} X_\alpha \to E^n(\colim_\alpha X_\alpha) \to \lim_\alpha E^n(X_\alpha) \to 0.\]
\end{remark}

Satisfied that we have indeed produced co/homology theories, we can investigate whether these assignments are mutual inverses.
To compare objects, this mostly comes down to Whitehead's theorem for spectra.
Using Brown representability for natural transformations, we can lift maps of cohomology theories up to maps of spectra: a map $E^* \xrightarrow f F^*$ induces a unique sequence of maps $E_n \xrightarrow{f_n} F_n$ and hence a map $\widetilde f\co E \to F$ of spectra.%
\marginnote{It's even reasonable to extend the definition of cohomology to $F^0(E) = [E, F]$, of which $\widetilde f$ is then an element.}
From this, one sees that a natural isomorphism of cohomology theories induces a weak equivalence of spectra, and conversely.
However, Brown representability falls short of giving a \emph{functorial} correspondence: the same natural transformation of cohomology theories can be induced by multiple homotopy-inequivalent maps of spectra.
More precisely, Brown's result shows that the construction
\begin{align*}
\CatOf{Spectra} & \to \CatOf{CohomologyTheories} \\
E & \mapsto [\Susp^\infty - , \Susp^* E]
\end{align*}
is full and bijective on isomorphism classes.

\begin{theorem}[Hurewicz]\marginnote{\citep[Theorem 10.25]{Switzer}}
\todo{Where does this belong?}
There is a map $\S \to H\Z$ which has $0$--connected fiber.
By consequence, the difference between $\S_*(X)$ and $H\Z_*(X)$ begins one degree above the bottommost cell in $X$.
By consequence, for $X$ $n$--connected and $n \ge 1$, $\pi_n X \cong \pi_n^s X \cong H\Z_n X$. \qed
\end{theorem}




\section{The smash product}

For all our discussion of homotopy and homology \emph{groups}, we have not yet found a framework for the cohomology \emph{ring} of a space.
The following observation is key:

\begin{definition}
A \define{ring} is a (commutative, unital) monoid in $\CatOf{AbelianGroups}$ under the $\otimes$--product.%
\marginnote{It is particularly important that one does \emph{not} use the Cartesian / categorical $\times$.}
\end{definition}

Our discussion around \Cref{RepresentableGroupsLemma} then indicates a way forward: since we have constructed an object $HR$ which represents ordinary cohomology with coefficients in $R$, a monoidal structure on $H^*(-; R)$ should induce a monoidal structure on $HR$.%
\marginnote{Ideally, it would even be visibly related to the original monoidal structure on $R$.}
In order to make sense of this, we need a monoidal structure on $h\CatOf{Spectra}$ which is compatible with the other monoidal structures in play, in the sense that the following pair of functors should be made monoidal: \[\CatOf{AbelianGroups} \xrightarrow H \CatOf{Spectra} \xleftarrow{\Susp^\infty} \CatOf{Spaces}_{*/}.\]
We have already introduced the operation \[\Susp^\infty X \sm \Susp^\infty Y := \Susp^\infty(X \sm Y)\] for two pointed spaces $X$ and $Y$.
Since $\CatOf{Spectra}$ is suitably generated by the image of $\Susp^\infty$, it should seem likely that this will pin down any putative monoidal structure.

For inspiration as to how to define the smash product in general, recall that we have also already defined $E \sm \Susp^\infty X$ for a generic spectrum $E$ and a pointed space $X$.
Given a presentation $E = \{\Susp^{n_j} \Susp^\infty E_j\}_j$, we set \[E \sm \Susp^\infty X := \{\Susp^{n_j} \Susp^\infty (E_j \sm X)\}_j.\]
That is, we commuted the smash product through the ind--system, where we reduced to the case of the smash product of suspension spectra.
In the fully general case of $E \sm F$, we may also choose a presentation of the second spectrum $F$ as $F = \{\Susp^{m_k} \Susp^\infty F_k\}_k$, and then we set \[(E \sm F)_{j,k} := \{\Susp^{n_j + m_k} \Susp^\infty E_j \sm F_k\},\] another ind--system.

\begin{remark}\marginnote{\citep[pg.\ 254--267]{Switzer}}
To define the smash product in terms of Boardman and Vogt's concrete model, we must convert this doubly-indexed ind--system into a sequential system.
One option is to select any cofinal subsystem, but this destroys the associativity of the product (and often destroys the commutativity).
A superior option is to ``sum over possible choices'': we set $(E \sm F)_n$ to be the colimit of the diagram under the $n$\textsuperscript{th} antidiagonal (after replacing the maps by cofibrations).
\end{remark}

From here, the main task is to show that this definition is sufficiently insensitive to the choice of presentation: given a pair of weakly equivalent presentation, one must show that this induces a weak equivalence after smashing.
This is possible, and hence one learns:

\begin{theorem}\marginnote{\citep[Theorem 13.40]{Switzer}}
\marginnote{\textbf{Warning:} This product is \emph{not} especially nice before passing to the homotopy category.  It turns out that this is unavoidable.}
In the homotopy category, this determines a symmetric monoidal product, $\sm$, on $\CatOf{Spectra}$. \qed
\end{theorem}

\begin{definition}\marginnote{\citep[Definition 13.50]{Switzer}}
A \define{ring spectrum} is a spectrum $E$ equipped with a multiplication map $\mu\co E \sm E \to E$ and a unit map $\eta\co \S \to E$ making $E$ into a (unital) monoid object in $h\CatOf{Spectra}$.
\end{definition}

\begin{corollary}
\marginnote{The statement here applies to any ring-valued theory.}
The cup product maps \[H^n(-; R) \times H^m(-; R) \xrightarrow\smile H^{n+m}(-; R)\] induce maps \[K(R, n) \sm K(R, m) \xrightarrow\smile K(R, n+m)\] which altogether induce a product \[HR \sm HR \to HR. \qed\]
\end{corollary}

\begin{example}
The sphere spectrum, $\S$, is the monoidal unit and hence also a ring.
\end{example}

The inverse construction is now straightforward.
Suppose that we have a pair of cohomology classes $\omega_n \in E^n(X)$ and $\omega_m \in E^m(X)$, for which we choose representatives $\omega_n\co \Susp^\infty X \to \Susp^n E$ and $\omega_m\co \Susp^\infty X \to \Susp^m E$.
Given a multiplication $\mu\co E \sm E \to E$, we define the product $\omega_n \smile \omega_m$ like so:
\marginnote{This same product can be used to make $E^* X$ (and $E_* X$) into $E_*$--modules.}
\[X \xrightarrow{\Delta} X \sm X \xrightarrow{\omega_n \sm \omega_m} \Susp^n E \sm \Susp^m E \simeq \Susp^{n+m}(E \sm E) \xrightarrow\mu \Susp^{n+m} E.\]

\begin{remark}
One can check that $- \sm E$ preserves colimits.
By positing an exponential adjunction, one can use Brown representability on the putative formula \[F(E_1, E_2)^*(X) = \text{``}\pi_0 F(E_1 \sm X, E_2)\text{''} = \CatOf{Spectra}(E_1 \sm X, E_2)\] to define a notion of \define{function spectrum}.
As with other objects extracted from Brown's result, this definition does not have excellent functoriality properties.
\marginnote{Of course, one can also give a direct definition.}
However, there is a version of the adjoint functor theorem that also applies to give a fully functorial statement---but this is beyond our current technology.
\end{remark}

Ring spectra induce a useful duality pairing between their associated co/homology theories.
\begin{definition}\marginnote{\citep[pg.\ 281]{Switzer}}
Given co/homology classes
\begin{align*}
(\sigma\co \S^n \to E \sm \Susp^\infty X) & \in E_n X, \\
(\omega \co \S^m \sm \Susp^\infty X \to E) & \in E^m(X),
\end{align*}
we define their pairing to be \[\<\omega, \sigma\>\co \S^{n+m} \xrightarrow{\Susp^m \sigma} E \sm \S^m \sm \Susp^\infty X \xrightarrow{1 \sm \omega} E \sm E \xrightarrow\mu E.\]
\end{definition}

\begin{lemma}\marginnote{\citep[Proposition 13.62.i]{Switzer}}
Under this pairing, the maps $f^*$, $f_*$ induced by $f$ are adjoint: \[\<f^* \omega, \sigma\> = \<\omega, f_* \sigma\>.\]
\end{lemma}
\begin{proof}
This is a consequence of the following diagram:
\begin{center}
\begin{tikzcd}
\S^{n+m} \arrow["\Susp^m \sigma"]{r} \arrow["\Susp^m f_* \sigma"']{rd} & E \sm \S^m \sm \Susp^\infty X \arrow["1 \sm 1 \sm f"]{d} \arrow["1 \sm f^* \omega"]{rd} \\
& E \sm \S^m \sm \Susp^\infty Y \arrow["\omega"]{r} & E \sm E \arrow["\mu"]{r} & E.
\end{tikzcd}
\end{center}
\end{proof}




\section{$G$--bundles and fiber bundles}

A major up-shot of representability is that the tools of algebraic topology can be turned on themselves.
We have previously announced our intention to understand the collection of natural transformations \[H^n(-; A) \to H^m(-; B).\]
By appealing to representability, this is not only equivalent to the collection of homotopy classes \[K(A, n) \to K(m, B)\] but also to the cohomology group \[H^m(K(A, n); B).\]
If we can get a sufficiently explicit handle on $K(A, n)$, we can use such a presentation to finish our original analysis.

There is a particularly restrictive form of fiber bundle that appears very often in geometric contexts:

\begin{definition}\marginnote{\citep[Definition 11.1]{Switzer}}
A \define{(real) vector bundle (of rank $k$)} over a base $B$ is a fiber bundle $p\co E \to B$ with fiber $\R^k$ and whose transition maps are linear functions.
\end{definition}

\begin{example}
The co/tangent bundles of a manifold are vector bundles.
\end{example}

\noindent
This constraint on the transition maps admits a universal form:

\begin{definition}\marginnote{\citep[Definition 11.4]{Switzer}}
A \define{$G$--bundle} is a fiber bundle $p\co E \to B$ where $G$ acts on $E$ (and trivially on $B$, and the map $p$ is equivariant), the identifications $\phi_U\co p^{-1}(U) \cong G \times B$ are equivariant, and the compatibilities $\phi_U|_{U \cap V} = \phi_V|_{U \cap V}$ are equivariant too.
\end{definition}

\begin{remark}\marginnote{\citep[11.21]{Switzer}}
This construction is universal in the following sense: if $G$ acts on an auxiliary space $F$, one can extract from a $G$--bundle $p\co E \to B$ an $F$--fiber bundle by \[E' = (F \times E) / (fg, e) \sim (f, ge).\]
Conversely, a fiber bundle with fiber $F$ has an associated $(\Aut F)$--bundle.
\end{remark}

\begin{example}\marginnote{\citep[Theorem 11.20, Proposition 11.22]{Switzer}}
Real vector bundles correspond with $\GL(\R^n)$--bundles under this construction.
The maximal compact subgroup of $\GL(\R^n)$ is the orthogonal group $O(\R^n)$, and the equivariant retraction $O(\R^n) \to \GL(\R^n)$ gives an equivalence between $O(\R^n)$--bundles and $\GL(\R^n)$--bundles (and hence with real vector bundles as well).
\end{example}

The local nature of the definition of a vector bundle gives rise to the following observation:

\begin{lemma}\marginnote{\citep[Proposition 11.32]{Switzer}}
The assignment $X \mapsto \{\text{isomorphism classes of $G$--bundles on $X$}\}$ satisfies the wedge axiom and Mayer--Vietoris. \qed
\end{lemma}

\begin{corollary}\marginnote{\citep[11.33]{Switzer}}
There is a homotopy type $BG$ representing this functor. \qed
\end{corollary}

\noindent
Purely through abstract principles, one can make an interesting qualitative statement about this homotopy type:

\begin{lemma}\marginnote{\citep[Proposition 11.27, Theorem 11.35]{Switzer}}
Let $E \to B$ be a $G$--bundle with $E$ $n$--connected.
The classifying map $B \to BG$ is then an $n$--equivalence.
It follows that the induced natural transformation $[-, B] \to [-, BG] \to k_G(-)$ is an equivalence on complexes of dimension $\le n$.
\qed
\end{lemma}

\begin{corollary}\label{CharacterizationOfBG}
The universal bundle $EG$ classified by $\id\co BG \to BG$ has \emph{contractible} total space.
Conversely, $G$--bundle with contractible total space is a model for the universal such bundle.
\qed
\end{corollary}

\begin{remark}\marginnote{\citep[11.43]{Switzer}}
With \Cref{LoopsShiftsPi} and \Cref{ShiftingEMSpaces} in mind, we can now smell the connection between these ideas and our pursuit of Eilenberg--Mac Lane spaces.
Namely, \Cref{LoopsShiftsPi} shows that the natural map \[G \to \Loops BG\] is an equivalence, so that \[K(A, n+1) \to BK(A, n)\] is also an equivalence.%
\marginnote{%
Making sense of this requires a model of $K(A, n)$ as an honest group, rather than as an $H$--group.
We blithely assert to the reader that such a model exists---its precise form won't turn out to be important.
}
\end{remark}

We now turn to the problem of producing a reliable model of $BG$, using \Cref{CharacterizationOfBG} as a guide.
\marginnote{The first-time reader will probably find it easier to conceptualize the following under the further condition that $G$ be finite (e.g., $\Z/2$).}
Our route, as ever, will pass through some creative category theory.
\todo{From here on, you owe the reader citations.}

\begin{definition}
Let $\CatOf C$ be a category.
Its \define{nerve} $N(\CatOf C)$ is a simplicial set with $0$--simplices the objects of $\CatOf C$, $1$--simplices the arrows of $\CatOf C$, $2$--simplices commuting triangles, $3$--simplices commuting tetrahedra, \ldots .
\marginnote{Equivalently: the $n$--simplices are given by length $n$ chains of composable morphisms.}
\end{definition}

\begin{remark}
This construction is a very faithful encoding of a category: the original category and its composition law can be recovered from the $0$--. $1$--, and $2$--simplices.
It also translates categorical ideas to recognizable topological objects: for instance, functors become continuous maps and natural transformations become homotopies of maps.
\end{remark}

\todo[inline]{Inject a comment about being able to extend this business to topologically enriched categories too.}

\begin{example}
For $G$ a group, we define two categories:
\begin{enumerate}
    \item $G \mmod G$ has objects $g \in G$ and maps $g \xrightarrow h gh$.
    \item $* \mmod G$ has one object $*$ and maps $* \xrightarrow h *$.
\end{enumerate}
\end{example}

\begin{lemma}
$G \mmod G$ is contractible.
\end{lemma}
\begin{proof}[Proof sketch]
This amounts to showing that any ``outer horn'' (i.e., a chain of morphisms of length $n-1$ and a morphism with either the same ultimate source or same ultimate target) is ``fillable'' (i.e., there is a chain of morphisms of length $n$ which extends the original chain and whose composite is the auxiliary morphism).
This is so: given \[g_1 \xrightarrow{g_2} g_1 g_2 \xrightarrow{g_3} \cdots \xrightarrow{g_n} g_1 \cdots g_n\] and \[g_1 \xrightarrow{h} h_{n+1},\] we can set $g_{n+1} = g_n^{-1} \cdots g_1^{-1} h_{n+1}$ to get \[g_1 \xrightarrow{g_2} g_1 g_2 \xrightarrow{g_3} \cdots \xrightarrow{g_n} g_1 \cdots g_n \xrightarrow{g_{n+1}} h_{n+1}. \qedhere\]
\end{proof}

\begin{remark}
The $G$--action on $G \mmod G$ is \emph{free}.
\end{remark}

\begin{corollary}
The quotient map $N(G \mmod G) \to N(* \mmod G)$ has fiber $G$, hence it models models $EG \to BG$. \qed
\end{corollary}

\begin{remark}
This is a more conceptual statement than you might think.
There are equivalences $G \mmod G \to \{\text{$G$--torsors with a trivialization}\}$ and $* \mmod G \to \{\text{$G$--torsors}\}$, and a map $X \to N(\{\text{$G$--torsors}\})$ for $X$ a simplicial set assigns each point in $X$ to a $G$--torsor, each path to a map of torsors, \ldots .
This \emph{sounds} like it's building a $G$--bundle on $X$ by specifying the fibers.
The \define{Grothendieck construction} makes this precise.
\end{remark}

This very concrete model for $BG$ has one really excellent feature: is has a naturally occurring skeletal filtration (viz., by simplex dimension) with identifiable quotients:

\begin{figure*}
\begin{center}
\begin{tikzcd}
BG^{(0)} \arrow[equal]{d} \arrow{r} & BG^{(1)} \arrow{d} \arrow{d} \arrow{r} & BG^{(2)} \arrow{d} \arrow{r} & \cdots \arrow{r} & BG^{(n-1)} \arrow{d} \arrow{r} & BG^{(n)} \arrow{d} \arrow{r} & \cdots \\
* & \Susp G & (\Susp G)^{\sm 2} & \cdots & (\Susp G)^{\sm (n-1)} & (\Susp G)^{\sm n} & \cdots.
\end{tikzcd}
\end{center}
\end{figure*}

If $h$ is a homology theory with K\"unneth isomorphisms, this gives a spectral sequence \[E^1_{*, *} = (\widetilde h_* \Susp G)^{\otimes *} \Rightarrow h_* BG.\]  More than this, the $d^1$--differential in then identifiable: \[d_1(g_1 \otimes \cdots \otimes g_n) = \sum_{j=2}^n (g_1 \otimes \cdots \otimes g_{j-1} g_j \otimes \cdots \otimes g_n).\]  This is a standard resolution appearing in homological algebra:
\marginnote{
This is a very common situation: some ``fully derived'' construction appearing in homotopy theory has behavior mediated by analogous homological algebra and a spectral sequence.
Since $\Tor^{h_* G}(h_*, h_*) = \pi_*(h_* \otimes_{h_* G}^{\mathbb L} h_*)$, this leads one to think of $BG$ as some kind of $* \times_G *$.
This turns out to be fruitful.
}
\[E_{*, *}^2 = \Tor^{h_* G}_{*, *}(h_*, h_*).\]




\section{The Steenrod algebra: calculation}

\todo{Throughout today, you owe the reader citations.}
Today we put the machinery of yesterday to work in the case of $H^m(K(\F_2, n); \F_2)$.
Our method is \emph{inductive}, and it ultimately rests on the following key observations:

\begin{example}
$\Tor$ algebras are generally remarkably computable: there is an algorithm, due to Tate, which forms a resolution of $h_*$ by a DGA which is levelwise $(h_* G)$--free.
To cover the algebras appearing in this computation, we will only need the following observations:
\begin{align*}
\Tor^{A \otimes B}_{*, *}(\F_2, \F_2) & \cong \Tor^A_{*, *}(\F_2, \F_2) \otimes \Tor^B_{*, *}(\F_2, \F_2), &
\Tor^{\Lambda[x]}_{*, *}(\F_2, \F_2) & \cong \bigotimes_{j=0}^\infty \Gamma[\sigma x],
\end{align*}
where $A$ and $B$ are $\F_2$--algebras, $\Lambda[x]$ denotes an exterior $\F_2$--algebra generated by the lone element $x$, and $\Gamma[\sigma x]$ denotes a divided power $\F_2$--algebra generated by the suspension of the element $x$.
\end{example}

\begin{lemma}
The pairing $K(\F_2, n) \times K(\F_2, 1) \xrightarrow\smile K(\F_2, n+1)$ induces a pairing ``$\circ$'' of spectral sequences
\begin{center}
\begin{tikzcd}
\Tor^{H_* (K(\F_2, n); \F_2)}_{*, *} \otimes H_*(K(\F_2, 1); \F_2) \arrow["\circ"]{r} \arrow[Rightarrow]{d} & \Tor^{H_* (K(\F_2, n+1); \F_2)}_{*, *} \arrow[Rightarrow]{d} \\
H_* (K(\F_2, n+1); \F_2) \otimes H_*(K(\F_2, 1); \F_2) \arrow["\smile"]{r} & H_* (K(\F_2, n+2); \F_2)
\end{tikzcd}
\end{center}
which converges to the cup product and which satisfies $d(x \circ y) = (dx) \smile y$.
\qed
\end{lemma}

We begin with the base case:
\begin{lemma}
The spectral sequence \[E_{*, *}^2 = \Tor^{H_*(\F_2; \F_2)}_{*, *}(\F_2, \F_2) \Rightarrow H_*(K(\F_2, 1); \F_2)\] collapses to give \[H_*(K(\F_2, 1); \F_2) \cong \Gamma[\sigma a].\]
\end{lemma}
\begin{proof}
We analyze the input to the spectral sequence \[E_{*, *}^2 = \Tor^{H_*(\F_2; \F_2)}_{*, *}(\F_2, \F_2) \Rightarrow H_*(K(\F_2, 1); \F_2).\]
The homology algebra $H_*(\F_2; \F_2)$ can be equivalently presented as \[H_*(\F_2; \F_2) \cong \F_2[\underline{1}] / (\underline 1^2 = 1) \cong \F_2[\underline 1 - 1] / (\underline 1 - 1)^2.\]
Since this algebra is exterior, we may compute \[\Tor^{H_*(\F_2; \F_2)}_{*, *} \cong \Gamma[\sigma a],\] for $a = \underline 1 - 1$.
The homology groups of $K(\F_2, 1) \simeq \RP^\infty$ have one class in every degree.
The bar spectral sequence also has one class in every degree, and there can therefore be no nonzero differentials, so that the spectral sequence collapses at $E_2$.
To establish convention, we write \[\Gamma[\sigma a] \cong \F_2[a_{(0)}, a_{(1)}, a_{(2)}, \ldots] / (a_{(j)}^2 = 0)\] for the algebra generators.
\end{proof}

\begin{theorem}
The above Lemma generalizes in $n$ to give \[H_* (K(\F_2, n); \F_2) \cong \F_2[a_{(j_1)} \circ \cdots \circ a_{(j_n)}] / (\text{squares}).\]
\end{theorem}
\begin{proof}[Proof sketch]
We proceed by induction, having shown the claim in the case $n = 0$.
By assumption, $H_* K(\F_2, n)$ is a tensor product of exterior algebras, so the K\"unneth formula for $\Tor$--algebras gives
\begin{align*}
\Tor^{H_* K(\F_2, n)}_{*, *}(\F_2, \F_2) & \cong \bigotimes_J \Tor^{\Lambda[a_{(j_1)} \smile \cdots \smile a_{(j_n)}]}_{*, *}(\F_2, \F_2) \\
& \cong \bigotimes_J \Gamma[a_{(j_1)} \smile \cdots \smile a_{(j_n)}].
\end{align*}
One can show the identity \todo{Wilson 8.16} \[(a_{(J)})_{(k)} \equiv a_{(J)} \smile a_{(k)} \pmod{\text{decomposables}}.\]
It follows that there are no differentials, since the spectral sequence for $H_*(K(\F_2, 1); \F_2)$ had none.
\end{proof}

This has a great many consequences.

\begin{corollary}\marginnote{\citep[Theorem 18.14]{Switzer}}
\marginnote{In terms of the operations $\Sq^n$ below, $H^*(K(\F_2, n); \F_2)$ is given by the algebra $\F_2[\Sq^I \iota_n \mid I_j \ge 2I_{j+1}, 2I_1 - I_+ < n]$.}
On cohomology, we have the calculation \[H^*(K(\F_2, n+1); \F_2) \cong \F_2[a_{(j_1)} \circ \cdots \circ a_{(j_n)}].\]
\end{corollary}
\begin{proof}[Proof sketch]
One employs that the dual of a primitively-generated divided-power Hopf algebra is a primitively-generated polynomial Hopf algebra.
\end{proof}

\noindent
This finishes the task we set out for ourselves at the beginning of this excursion, but we can collect a bit more at the level of spectra.

\begin{corollary}\marginnote{\citep[Theorem 18.20]{Switzer}}
The \define{dual Steenrod algebra} is given by \[\A_* := H\F_2{}_* H\F_2 \cong \F_2[\xi_1, \xi_2, \ldots, \xi_n, \ldots],\] where $|\xi_n| = 2^n - 1$ is represented by $a_{(n)} \in H_{2^n-1}(\Susp^{-1} K(\F_2, 1); \F_2)$.
\end{corollary}
\begin{proof}
From our definition of $H\F_2 \sm H\F_2$, we have \[H\F_2 \sm H\F_2 \simeq H\F_2 \sm (\colim_n \Susp^{-n} K(\F_2, n)) \simeq \colim (H\F_2 \sm \Susp^{-n} K(\F_2, n)),\] so that applying $\pi_m$ gives \[(H\F_2)_m H\F_2 = \lim_n H_{m-n}(K(\F_2, n); \F_2).\]
The Theorem gives us access to these groups, provided we can describe the maps participating in the colimit.
These maps turn out to be $- \smile a_{(0)}$, owing to the factorization
\begin{center}
\begin{tikzcd}
S^1 \sm K(\F_2, n) \arrow{rr} \arrow["{a_{(0)} \times \id}"]{rd} & & K(\F_2, n+1) \\
& K(\F_2, 1) \sm K(\F_2, n) \arrow["\smile"]{ru}.
\end{tikzcd}
\end{center}
\end{proof}

The analogous formula for stable cohomology is harder: the answer as a coalgebra is encoded in the above, but to produce a description as a Hopf algebra we need to understand the comultiplication on $\A_*$.
This is more complicated, so we merely quote the result:
\begin{lemma}
The comuliplication on $\A_*$ is given by
\[\Delta \xi_n = \sum_{j=0}^n \xi_j \otimes \xi_{n-j}^{2^j}.\]
The primitive elements of this algebra are $\xi_1^{2^j}$.
\qed
\end{lemma}
\todo{Lots more formulas like this can be read off.  Maybe on your homework?}

\begin{corollary}[{\citep[pg.\ 451]{Switzer}}]
The Steenrod algebra, $\A^* := H\F_2^* H\F_2$, is noncommutative and generated by elements $\Sq^{2^j}$ dual to $\xi_1^{2^j}$.
\marginnote{The space-level versions of these calculations are called the \define{unstable (dual) Steenrod algebra}.}
\todo{How is $\Sq^n$ defined in general?  Dual to $\xi_1^n$?  I don't think so.}
\qed
\end{corollary}

\begin{remark}
A lot can be computed about $\A^*$ by studying universal cases.
For instance, $\Delta(x^2)^* = 1 \mid (x^2)^* + \xi_1 \mid (x)^* \in \A_* \otimes \widetilde H_* \RP^\infty$ says $\Sq^0(x^2) = x^2$ and $\Sq^1(x) = x^2$ for $|x| = 1$.
In fact, we have
\begin{enumerate}
    \item $\Sq^0(x) = x$.
    \item $\Sq^{>|x|}(x) = 0$.
    \item $\Sq^{|x|}(x) = x^2$.
    \item $\Sq^n(x+y) = \Sq^n x + \Sq^n y$.
    \item $\Sq^n(xy) = \sum_{n_1 + n_2 = n} \Sq^{n_1}(x) \cdot \Sq^{n_2} y$.
    \item ``The Adem relations'', summarized by
    \begin{enumerate}
        \item $\Sq^{2n-1} \Sq^n = 0$, and
        \item $d(\Sq^n) = \Sq^{n-1}$ extends to a derivation.
    \end{enumerate}
    So, for instance,
    \begin{align*}
    0 & = d^3(\Sq^5 \Sq^3) \\
    & = d(\Sq^3\Sq^3 + \Sq^5 \Sq^1) \\
    & = \Sq^2 \Sq^3 + \Sq^3 \Sq^2 + \Sq^4 \Sq^1 + \Sq^5 \Sq^0.
    \end{align*}
\end{enumerate}
\end{remark}
\todo{Emphasize that the cohomology of spaces and spectra have natural actions of these endomorphism algebras.}




\begin{subappendices}

\section{$K$--theory}

\todo[inline]{Since we're talking about $G$--bundles, I think it would be smart to give a brief treatment of complex $K$--theory, even if it doesn't directly factor into our future calculations.}

% \begin{remark}
% There is a \emph{really} slick proof of complex Bott periodicity that uses these nerve constructions.  It's on your homework.\todo{Inject this.}
% \end{remark}

\end{subappendices}
