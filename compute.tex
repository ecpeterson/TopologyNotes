% -*- root: main.tex -*-

\chapter{Computations in the Homotopy Groups of Spheres}

At last, we come to the program outlined in the introduction to this book, which employs all of the technology developed so far.

Our goal now is to perform some computatins in the homotopy groups of spheres, themselves interesting for their role in determining what attaching maps are available during the cellular construction of homotopy types.
Our access to these groups is indirect and manifold.
In the first method, known as \define{Serre's method}, we will essentially construct the Postnikov decomposition of a sphere by using the Serre spectral sequence and our intimate knowledge of the Steenrod algebra to deduce how its $k$--invariants \emph{must} fit together.
In the second method, we will assemble various stable phenomena to produce a simplification of the computation that is valid for spheres of large dimension, known as the \define{Adams spectral sequence}.
In the third method, we will use certain exact sequences among the spheres themselves, called \define{EHP sequences}, to propagate the knowledge garnered in the first two attempts to otherwise unreachable areas.

By the conclusion of this chapter, we will have computed $\pi_{n+k} S^n$ for all $n \ge 1$ and for $0 \le k \le 7$, though our methods could certainly be pushed farther by the interested reader.




\section{Serre's method}\label{SerresMethod}

We now use these techniques to start computing homotopy groups.
Recall that when we were constructing Eilenberg--Mac Lane spaces, we started with the sphere, with known homology and unknown homotopy, and we repeatedly applied the Hurewicz theorem to identify a layer of homotopy, used coexact sequences to remove it, and ultimately produced a space with minimal homotopy.
It turns out that the \define{Eckmann--Hilton dual} to this approach is also possible, as motivated by the following observation:

\begin{remark}\label{KillingCohWithExact}
\marginnote{\citep[Formula 11.1]{MosherTangora}}
\marginnote{If a cohomology class is \define{spherical}, one can take the cofiber to kill the class.
This makes more obvious use of the Eilenberg--Steenrod axioms.
However, not all cohomology classes are spherical, and even for a spherical class a choice of spherical representative is potentially unnatural.}
Consider a cohomology class $\omega \in H^n(X; A)$, represented by a map $\omega\co X \to K(A, n)$.
In the spectral sequence for \[K(A, n-1) \to * \to K(A, n),\] we know that the fundamental class transgresses.
Now consider the pulled back fibration, $P_\omega$:
\begin{center}
\begin{tikzcd}
K(A, n-1) \arrow[equal]{d} \arrow{r} & P_\omega \arrow{d} \arrow{r} & X \arrow["\omega"]{d} \\
K(A, n-1) \arrow{r} & * \arrow{r} & K(A, n).
\end{tikzcd}
\end{center}
Because the fundamental class transgresses in the universal case, it also transgresses in this spectral sequence: \[d_{n+1} \iota_n = \omega.\]
Aside from killing $\omega$, this operation has somewhat wild behavior.
First, note that the entire Steenrod module spanned by $\omega$ is also destroyed by the ensuing pattern of differentials: \[d_{n+1+I_+} \Sq^I \iota_n = \Sq^I \omega.\]
However, if $\Sq^I \omega = 0$, this leaves $\Sq^I \iota_n$ as a survivor of the spectral sequence.
These surviving classes can participate in nontrivial extensions of the Steenrod action.
\end{remark}

With this technique in mind, the rough idea in the section is to perform the reverse of the construction of Eilenberg--Mac Lane spaces from \Cref{EMSpacesExist}.
We start with an Eilenberg--Mac Lane space, which has \emph{both} known homotopy and known co/homology.
We will then repeatedly apply the exact sequence technique from \Cref{KillingCohWithExact} to excise the bottom layer of cohomology (which gives some information about homotopy via Hurewicz), ultimately producing a space with minimal homology.
In order to make full use of \Cref{KillingCohWithExact}, we will work with mod--$2$ homology rather than integral homology---which means, according to \Cref{WhatIsModPGoodFor}, that we will produce a model for $(S^n)_2$.

\marginnote{\citep[pg.\ 114--118]{MosherTangora}}
Let's get started, beginning with $X_n = K(\Z, n)$.
This space models a downward Postnikov trunction of $S^n$: \[S^n \to S^n[0, n] = K(\Z, n).\]
The mod--$2$ cohomology of $S^n[0, n]$ matches that of $S^n$ through degree $n+1$, but there is a class $\Sq^2 \iota_n \in H^{n+2}(S^n[0, n]; \F_2)$ which is not present in $H^{n+2}(S^n; \F_2)$.
\marginnote{%
This class in $H^{n+2}$ indicates a discrepancy in $\pi_{n+1}$.
The difference in degree arises from the torsion shift in the universal coefficient theorem.}
To correct this discrepancy, we apply \Cref{KillingCohWithExact} to this class.
\marginnote{
\[
\begin{tikzcd}[ampersand replacement=\&, column sep=1em]
\& K(\Z/2, n+1) \arrow{d} \arrow[equal]{r} \& K(\Z/2, n+1) \arrow{d} \\
\& S^n[0, n+1] \arrow{r} \arrow{d} \& * \arrow{d} \\
S^n \arrow[densely dotted]{ru} \arrow{r} \& K(\Z, n) \arrow["{\Sq^2 \iota_n}"']{r} \& K(\Z/2, n+2).
\end{tikzcd}
\]}

We would like to continue this process by identifying the next discrepancy between the mod--$2$ cohomologies of $S^n$ and $S^n[0, n+1]$---which means we must compute the cohomology of $S^n[0, n+1]$.
\Cref{KillingCohWithExact} gives us access to this via a Serre spectral sequence with \emph{lots} of differentials.

\begin{note}
In order to simplify the analysis of this spectral sequence, we make the assumption $n \gg 0$.
This assumption eliminates the inequality condition in \Cref{EMCohAsASteenrodAlg} and pushes all product terms into the high degree $2n \gg n$.
\end{note}

\noindent
With this established, one works through the applications of \Cref{KudosThm} to produce the following family of transgressive differentials:
\begin{align*}
& & H^{n+k-1}(K(\Z/2, n+1); \F_2) & \xrightarrow{d} H^{n+k}(S^n[0, n]; \F_2) \\
k & = 0: &
& \quad \textcolor{red}{\iota_n} \\
k & = 2: &
\iota_{n+1} & \mapsto \Sq^2 \iota_n, \\
k & = 3: &
\Sq^1 \iota_{n+1} & \mapsto \Sq^1 \Sq^2 \iota_n, \\
k & = 4: &
\textcolor{red}{\Sq^2 \iota_{n+1}}, & \quad \textcolor{red}{\Sq^4 \iota_n}, \\
k & = 5: &
\textcolor{red}{\Sq^1 \Sq^2 \iota_{n+1}}, \\
& & \Sq^2 \Sq^1 \iota_{n+1} & \mapsto \Sq^5 \iota_n, \\
k & = 6: &
\textcolor{red}{\Sq^3 \Sq^1 \iota_{n+1}}, \\
& & \Sq^4 \iota_{n+1} & \mapsto \Sq^4 \Sq^2 \iota_n, \\
k & = 7: &
\textcolor{red}{\Sq^5 \iota_{n+1} + \Sq^4 \Sq^1 \iota_{n+1}}, & \quad \textcolor{red}{\Sq^7 \iota_n}, \\
& & \Sq^5 \iota_{n+1} & \mapsto \Sq^5 \Sq^2 \iota_n, \\
k & = 8: &
\textcolor{red}{\Sq^5 \Sq^1 \iota_{n+1}}, \textcolor{red}{\Sq^4 \Sq^2 \iota_{n+1}}, \\
& & \Sq^6 \iota_{n+1} & \mapsto \cdots
\end{align*}
The classes colored red do not participate in differentials and so survive to the $E_\infty$--page.
This table thus has enough information to read off $H^{\le n+7}(S^n[0, n+1]; \F_2)$, as well as much of its Steenrod action.

The first abberration from $H^{\le n+7}(S^n; \F_2)$ is now \[\Sq^2 \iota_{n+1} \in H^{n+3}(S^n[0, n+1]; \F_2).\]
\marginnote{
\begin{tikzcd}[ampersand replacement=\&, column sep=1em]
\& K(\Z/2, n+2) \arrow{d} \arrow[equal]{r} \& K(\Z/2, n+2) \arrow{d} \\
\& S^n[0, n+2] \arrow{r} \arrow{d} \& * \arrow{d} \\
S^n \arrow[densely dotted]{ru} \arrow{r} \& S^n[0, n+1] \arrow["\Sq^2\iota_{n+1}"']{r} \& K(\Z/2, n+3).
\end{tikzcd}
}
We proceed to use \Cref{KillingCohWithExact} to delete this class.
This yields a new Serre spectral sequence to work through:
\begin{align*}
& & H^{n+k-1}(K(\Z/2, n+2); \F_2) & \xrightarrow{d} H^{n+k}(S^n[0, n+1]; \F_2) \\
k & = 0: & & \quad \textcolor{red}{\iota_n}, \\
k & = 3: & \iota_{n+2} & \mapsto \Sq^2 \iota_{n+1}, \\
k & = 4: & & \quad \textcolor{red}{\Sq^4 \iota_n}, \\
& & \Sq^1 \iota_{n+2} & \mapsto \Sq^3 \iota_{n+1}, \\
k & = 5: & \Sq^2 \iota_{n+2} & \mapsto \Sq^3 \Sq^1 \iota_{n+1} \\
k & = 6: & \textcolor{red}{\Sq^3 \iota_{n+2}}, & \quad \textcolor{red}{\Sq^6 \iota_n}, \\
& & \Sq^2 \Sq^1 \iota_{n+2} & \mapsto (\Sq^5 + \Sq^4\Sq^1) \iota_n \\
k & = 7: & & \quad \textcolor{red}{\Sq^7 \iota_n}, \\
& & \Sq^3 \Sq^1 \iota_{n+2} & \mapsto \Sq^5 \Sq^1 \iota_n, \\
& & \Sq^4 \iota_{n+2} & \mapsto \Sq^4 \Sq^2 \iota_n, \\
k & = 8: & \textcolor{red}{(\Sq^5 + \Sq^4 \Sq^1) \iota_{n+2}}, \\
& & \Sq^5 \iota_{n+2} & \mapsto \cdots.
\end{align*}

The first abberation from $H^{\le n+7}(S^n; \F_2)$ is now \[\Sq^4 \iota_n \in H^{n+4}(S^n[0, n+2]; \F_2).\]
We would like to apply \Cref{KillingCohWithExact} to the map $\Sq^4 \iota_n\co S^n[0, n+2] \to K(\Z/2, n+4)$, but there is no class $\Sq^1 \Sq^4 \iota_n \in H^*(S^n[0, n+2]; \F_2)$, so that $\Sq^1 \iota_{n+3}$ will contribute another class in the same degree.
This signifies that we should use $K(\Z/2^j, n+4)$ instead for some $j > 1$, according to which $j$ satisfies \[\beta_j \Sq^4 \iota_n = \Sq^3 \iota_{n+2}.\]
\todo{The Lemma only works up to filtration.}
To make this calculation, we appeal repeatedly to \Cref{BocksteinLemma}, which says that \[\beta_j \Sq^4 \iota_n = \Sq^1(\Sq^2 \iota_{n+2})\] would hold if \[\beta_{j-1} \Sq^4 \iota_n = \Sq^1(\Sq^2 \Sq^1 \iota_{n+1}) = d(\Sq^2 \iota_{n+2})\] were true, where $d$ denotes a transgressive differential.
In turn, this left equality would hold if \[\beta_{j-2} \Sq^4 \iota_n = \Sq^1 (\Sq^4 \iota_n) = d(\Sq^2 \Sq^1 \iota_{n+1})\] were to hold, where $d$ is now the transgressive differential associated to the \emph{previous} exact sequence.
This can be made to hold by taking $j = 3$, so that \[\beta_3 \Sq^4 \iota_n = \Sq^3 \iota_{n+2}.\]
From this, we calculate the following table of differentials:
\begin{align*}
& & H^{n+k-1}(K(\Z/2^3, n+3); \F_2) & \xrightarrow{d} H^{n+k}(S^n[0, n+2]; \F_2) \\
k & = 0: & & \quad \textcolor{red}{\iota_n}, \\
k & = 4: &
\iota_{n+3} & \mapsto \Sq^4 \iota_n, \\
k & = 5: &
\beta_3 \iota_{n+3} & \mapsto \Sq^3 \iota_{n+2}, \\
k & = 6: &
\Sq^2 \iota_{n+3} & \mapsto \Sq^6 \iota_n, \\
k & = 7: &
\Sq^3 \iota_{n+3} & \mapsto \Sq^7 \iota_n, \\
& & \Sq^2 \beta_3 \iota_{n+3} & \mapsto (\Sq^5 + \Sq^4 \Sq^1)\iota_{n+2}, \\
k & = 8: & \textcolor{red}{\Sq^4 \iota_{n+3}}, \\
& & \Sq^3 \beta_3 \iota_{n+3} & \mapsto \cdots.
\end{align*}

\noindent
From this we conclude that the next abberrant class lies in degree $7$.

\marginnote{\citep[Theorem 12.2]{MosherTangora}}
Looking back at the Postnikov layers we have constructed, we see that we have computed
\todo{Use booktabs}
\[\begin{array}{c|ccccccc}
n & 0 & 1 & 2 & 3 & 4 & 5 \\
\hline
\pi_n \S & \Z & \Z/2 & \Z/2 & \Z/2^3 & 0 & 0
\end{array}\]
We could, presumably, keep going.
\marginnote{%
Actually, there is an indeterminacy at $k = 14$ which \emph{cannot} be resolved with the methods here.
We will meet this again.
}

\begin{remark}\label{UnstableSerresMethodEx}
\marginnote{\citep[pg.\ 123--127]{MosherTangora}}
\marginnote{
\begin{sseqdata}[name = SerreS304, cohomological Serre grading, yscale = 0.6, x range = {0}{10}, y range = {0}{9}, xscale = 0.5]
% bottom row
\class["\iota_3"](3,0)
\class["\Sq^2\iota_3"](5,0)
\class["\iota_3^2"](6,0)
\class["\iota_3\cdot\Sq^2\iota_3"](8,0)
\class["\Sq^4\Sq^2\iota_3"](9,0)
\class["\iota_3^3"](9,0)
\class["(\Sq^2\iota_3)^2"](10,0)

% left column
\class["\iota_4"](0,4)
\class["\Sq^1\iota_4"](0,5)
\class["\Sq^2\iota_4"](0,6)
\class["\Sq^2\Sq^1\iota_4"](0,7)
\class["\Sq^3\iota_4"](0,7)
\class["\Sq^3\Sq^1\iota_4"](0,8)
\class["\iota_4^2"](0,8)
\class["\Sq^4\Sq^1\iota_4"](0,9)
\class["\iota_4\cdot\Sq^1\iota_4"](0,9)

% bulk
\class(3,4)
\class(3,5)
\class(3,6)
\class(3,7)
\class(3,7)
\class(5,4)
\class(5,5)
\class(6,4)

\d5(0,4)(5,0)
\d5(3,4)(8,0)
\d5(5,4)(10,0)
\d5(0,9,-1)(5,5)

\d6(0,5)(6,0)
\d6(3,5)(9,0,-1)

\d9(0,8)(9,0)
\end{sseqdata}
\printpage[name = SerreS304, page = 0]
}
\todo{There are some classes in the bulk in my notes that are neither circled nor participate in differentials.  They are fringe classes, though...?}
For contrast, we also consider the computation in very low degrees for the fibration \[K(\Z/2, 4) \to S^3[0, 4] \to S^3[0, 3],\] where the pertinent differentials coming from \Cref{KudosThm} are
\begin{align*}
d_5 \iota_4 & = \Sq^2 \iota_3, &
d_6 \Sq^1 \iota_4 & = \iota_3^2, &
d_9 \Sq^4 \iota_4 & = \Sq^4 \Sq^2 \iota_3.
\end{align*}
The surviving classes are
\begin{align*}
\iota_3, & &
\Sq^2 \iota_4, & &
\Sq^3 \iota_4, & &
\Sq^2\Sq^1 \iota_4, & &
\Sq^3\Sq^1 \iota_4, & &
\Sq^4 \Sq^1 \iota_4, & &
\iota_3 \otimes \Sq^2 \iota_4.
\end{align*}
This is quite different from the $n \gg 0$ calculation: we have already felt the effects both of the inequalities and the products appearing in \Cref{EMCohAsASteenrodAlg}.

Continuing in this manner, the sufficiently intrepid computationalist can produce the table
\[\begin{array}{c|cccc}
n & 0 & 1 & 2 & 3 \\
\hline
\pi_{n+3} S^3 & \Z & \Z/2 & \Z/2 & \textcolor{red}{\Z/4},
\end{array}\]
which showcases the unstable phenomenon $\pi_6 S^3 \not\cong \pi_3 \S$.
\end{remark}





\section{The Adams spectral sequence}

In the previous section, we made an innocuous-seeming observation: the map \[H^*(H\F_2; \F_2) \to H^{*+n}(K(\F_2, n); \F_2)\] is an isomorphism through degree $n$.
We limited our calculation to this range in order to skirt the excess inequality and the presence of product classes in $H^{*+n}(K(\F_2, n); \F_2)$, two difficult features.
In fact, this restriction skirted much more: even the Serre spectral sequences were quite simple, populated only with transgressive differentials.
\marginnote{For contrast, consider the nontransgressive differential in \Cref{UnstableSerresMethodEx}.}
This is a consequence of \Cref{ConnectedExactAndCoexactAgree}: in the range considered, the exact sequences \[K(\pi_{n+m} S^n, n+m) \to S^n[0, n+m] \to S^n[0, n+m)\] can instead be thought of as coexact sequences of spectra \[\Susp^{n+m} H(\pi_{n+m} S^n) \to \S^n[0, n+m] \to \S^n[0, n+m),\] and the associated Serre spectral sequences can instead be thought of as exact sequences: \[\cdots \from H^*(H(\pi_{n+m} S^n); \F_2) \from H^*(\S^n[0, n+m]; \F_2) \from H^*(\S^n[0, n+m); \F_2) \from \cdots.\]
The pervasiveness of \Cref{KudosThm} in governing the behavior of the Serre spectral sequence is abbreviated stably to this long exact sequence being one of Steenrod modules.

Our goal in this section is to pursue this construction directly within the stable category.
Our intention is essentially the same: we will strip a spectrum of its homotopy by iteratively applying the stable Hurewicz map.
\marginnote{Strictly speaking, this is opposite to our approach using Serre's method, where we were stripping the cohomology of $K(\Z, n)$ to match those of $S^n$, rather than stripping the homotopy of $S^n$ to match $K(\Z, n)$.}
The main difference is that, stably, we are permitted to do this using coexact sequences.

\begin{lemma}
Take $X$ to be $(n-1)$--connected with $\pi_n X$ finite exponent and $2$--torsion.
Consider the exact extension \[X' \to \S \sm X \to H\F_2 \sm X.\]
Then $\pi_n X' < \pi_n X$.
\end{lemma}
\begin{proof}
Various aspects of the Hurewicz theorem tell us
\begin{align*}
\pi_{< n}(H\F_2 \sm X) & = 0, &
\im(\pi_n X \to \pi_n H\F_2 \sm X) & = (\pi_n X) / 2,
\end{align*}
and the map $\pi_{n+1} X \to H_{n+1} X$ is onto.
Altogether, this shows $X'$ to be $(n-1)$--connected and that there is an inclusion $\pi_n X' < \pi_n X$.
\end{proof}

\begin{theorem}\marginnote{\citep[Proposition 19.12]{Switzer}}
Let $\overline H \to \S \xrightarrow \eta H$ be the exact extension of the unit map.
For $X$ a connected spectrum with $\pi_* X$ $2$--torsion and degreewise finite exponent, the associated co-filtered object
\begin{center}
\begin{tikzcd}
\S \sm X \arrow{d} & \overline H \sm X \arrow{d} \arrow{l} & \overline H^{\sm 2} \sm X \arrow{l} \arrow{d} & \cdots \arrow{l} \\
H \sm X & H \sm \overline H \sm X & H \sm \overline H^{\sm 2} \sm X
\end{tikzcd}
\end{center}
has contractible limit, hence gives a spectral sequence which strongly converges to $\pi_* X$.
\marginnote{For $X$ a connective spectrum of finite type, this converges to the homotopy of the $2$--adic completion, $\pi_* X_2$.}
\end{theorem}
\begin{proof}
\marginnote{It is also helpful to note that $H \sm \overline H$ has this same finiteness property.}
This spectral sequence is exactly the Lemma, repeatedly applied.
\end{proof}

This construction is easier to think about after taking $\F_2$--duals.  Then $E^1_{*, t} = (\pi_* H \sm \overline H^{\sm t} \sm X)^* = (\overline{\A}^*)^{\otimes t} \sm H\F_2^* X$.  The differential exactly tracks $\A^*$--module maps $H\F_2^* X \to \F_2$, and a little homological algebra reveals:
\begin{lemma}
$E^2_{*, *} = \Ext^{\A^*}_{*, *}(H\F_2^* X, \F_2)$. \qed
\end{lemma}

\begin{remark}
\todo[inline]{Explain the relationship between the Bockstein lemma and $h_0$--extensions.}
\end{remark}

\todo[inline]{Work through a minimal free resolution?}




\section{The May spectral sequence}

Our goal in this section is to learn a systematic method for computing in the $E_2$--page for the Adams spectral sequence for the sphere, due to Peter May.
\marginnote{It is in this section that we make good on explaining the spectral sequence previously considered in \Cref{ComplicatedExampleSec}.}
The idea is that connected, graded, finite-type Hopf algebras admit filtration (essentially by ``word length'') which \emph{trivialize} their multiplication and comultiplication.
This gives a spectral sequence beginning with the cohomology of a bouquet of exterior algebras.

\begin{theorem}[May]\marginnote{\citep{May}}
There is a spectral sequence \emph{of algebras} with $E_1^{*, *, *} \cong \F_2[h_{ij} \mid i \ge 1, j \ge 0]$ converging to the Adams $E_2$--term for the sphere.
The element $h_{ij}$ is represented by $\xi_i^{2^j}$, hence \[d_1 h_{ij} = \sum_{k=1}^{i-1} h_{kj} h_{(i-k)(k+j)}. \qed \]
\end{theorem}

In order to produce the $E_2$--page out through degree $15$, we will need to study the following generators:
\[\begin{array}{c|ccccccccccc}
\text{name} & h_{10} & h_{11} & h_{12} & h_{13} & h_{14} & h_{20} & h_{21} & h_{22} & h_{30} & h_{31} & h_{40} \\
\text{horiz.\ degree} & 0 & 1 & 3 & 7 & 15 & 2 & 5 & 11 & 6 & 13 & 14. \end{array}\]
All differentials are linear against $h_{1j}$, and the first differential is linear against $h_{ij}^2$.
Applying the Theorem, it is determined on a generating set by
\begin{align*}
d(h_{20}) & = h_{10} h_{11}, &
d(h_{21}) & = h_{11} h_{12}, &
d(h_{22}) & = h_{12} h_{13},
\end{align*} \vspace{-2\baselineskip}
\begin{align*}
d(h_{30}) & = h_{10} h_{21} + h_{20} h_{12}, &
d(h_{31}) & = h_{11} h_{22} + h_{21} h_{13},
\end{align*} \vspace{-2\baselineskip}
\begin{align*}
d(h_{40}) & = h_{10} h_{31} + h_{20} h_{22} + h_{30} h_{13}.
\end{align*}

We now give a flavor of this computation, working column by column:

\begin{enumerate}
    \setcounter{enumi}{-1}
    \item The zeroth column is populated with powers of $h_{10}$, which persist.
    \item The first column is populated by $h_{10}^j h_{11}$, and $d(h_{20}) = h_{10} h_{11}$ removes all but the first one.
    \item The second column is populated by $h_{10}^j h_{11}^2$, and again only $h_{11}^2$ survives.
    \item After clearing the differential $d(h_{10}^j h_{11}^2 h_{20}) = h_{10}^{1+j} h_{11}^3$, the third column is populated by $h_{11}^3$ and $h_{10}^j h_{12}$.
    At this point, we will invoke a useful theorem of Nakamura:
\begin{theorem}[Nakamura]\marginnote{\citep{Nakamura}}
\marginnote{%
Nakamura actually proves these items with various bounds on the size of $x$ and $y$.
His bound is more than enough to pursue the calculation through our range of interest, but the Theorem appears to hold even without a bound.
}
There are operators $\Sq^n$ acting on the May spectral sequence, satisfying\ldots
\begin{enumerate}
    \item $\Sq^0 h_{ij} = h_{i(j+1)}$.
    \item $\Sq^1 h_{ij} = h_{ij}^2$.
    \item $\Sq^n$ is linear.
    \item $\Sq^n(x \cdot y) = \sum_{i+j=n} \Sq^i(x) \Sq^j(y)$.
    \item $\Sq^n(d_? x) = d_? \Sq^n(x)$, where ``$?$'' is as small as possible so that both sides are nonzero.
\end{enumerate}
\end{theorem}
    We can then compute 
    \begin{align*}
    d(h_{20}^2) & = d(\Sq^1 h_{20}) = \Sq^1 d(h_{20}) = \Sq^1(h_{10} h_{11}) \\
    & = \Sq^1(h_{10}) \Sq^0(h_{11}) + \Sq^0(h_{10}) \Sq^1(h_{11}) \\
    & = h_{11}^3 + h_{10}^2 h_{12}.
    \end{align*}
    This differential identifies the classes $h_{11}^3 = h_{10}^2 h_{12}$ and it deletes all classes of the form $h_{10}^{2+j} h_{12}$, $j \ge 1$.
    The only remaining class in this column is therefore $h_{11}^3 = h_{10}^2 h_{12}$.
    \item The differentials considered so far eliminate all classes in the fourth column other than $h_{10}^j h_{11} h_{12}$, which is exactly the target of $d(h_{10}^j h_{21})$.
    No classes survive in this column at all.
    \item Similarly, the differentials considered so far eliminate all classes in the fifth column other than $h_{10} h_{21} + h_{20} h_{12}$, itself composed of two classes targeting the same class in the fourth column: \[d(h_{10} h_{21}) = h_{10}(h_{11} h_{12}) = (h_{10} h_{11}) h_{12} = d(h_{20} h_{12}).\]
    This class is exactly the target of $d(h_{30})$, so that no classes survive.
\end{enumerate}

\noindent
\marginnote[-14\baselineskip]{%
\begin{sseqdata}[name = AdamsE2, xscale = 0.25, yscale = 0.4, x range = {0}{15}, y range = {0}{8}, no ticks]
\class(0,0)
\foreach \j in {1,...,9} { \class(0,\j) \structline }
% first shark fin / spray near h1 and h2
\class(1,1) \structline(0,0)
\class(2,2) \structline
\class(3,3) \structline
\class(3,1)
\class(3,2) \structline
\structline(3,3)(3,2)

% spray near h3
\class(6,2)
\class(7,1)
\class(7,2) \structline
\class(7,3) \structline
\class(7,4) \structline
\class(8,2) \structline(8,2)(7,1)
\class(9,3) \structline
\class(8,3)
\class(9,4) \structline

% shark fin
\class(9,5)
\class(10,6) \structline
\class(11,7) \structline
\class(11,5)
\class(11,6) \structline
\structline(11,7)(11,6)

% spray near h4
\class(14,2)
\class(14,3) \structline
\class(14,4)
\class(14,5) \structline
\class(14,6) \structline
\class(15,1)
\foreach \j in {2,...,8} { \class(15,\j) \structline }
\class(16,2) \structline(16,2)(15,1)
\class(15,5) \structline(15,5)(14,4)
\class(16,6) \structline(16,6)(15,5)
\class(16,7) \structline[densely dotted](15,4)

\d2(15,1)(14,3)
\d3(15,2)(14,5)
\d3(15,3)(14,6)
\end{sseqdata}
\printpage[name = AdamsE2, page = 0]
}
This is already enough to recover the range we worked through with Serre's method.
We include at right a picture of the spectral sequence through degree $15$.

\begin{remark}
We have additionally included a pair of Adams differentials
\begin{align*}
d_2(h_{14}) & = h_{13}^2 h_{10}, &
d_3(h_{10}^{1+j} h_{14}) & = h_{10}^{1+j} d_0.
\end{align*}
Resolving the presence of absence of these differentials manifests in Serre's method as an indeterminacy in the Serre spectral sequence at that stage.
These differentials also have a geometric origin: their presence is equivalent to the nonexistence of a normed real division algebra of dimension $16$.
In fact, the Adams differentials $d_2(h_{1j}) = h_{10} h_{1(j-1)}^2$ prohibit the existence of such division algebras of dimension $2^j$ for $j \ge 4$.
\marginnote{\citep{Adams}}
This differential was originally deduced by Adams through an intensive study of secondary cohomology operations.
\marginnote{\citep{AdamsAtiyah}}
Later, Adams found a second, shorter proof using Adams spectral sequences based on other cohomology theories (viz., $K$--theory and complex bordism).
\marginnote{\citep[Corollary VI.1.5]{BMMS}}
A third proof was produced by Bruner using power operations---a kind of spiritual cousin to Nakamura's theorem above.
These are all well beyond the technology presented in these notes, though they are each logical subjects for the reader to study next.
\end{remark}

\begin{remark}
\marginnote{%
\phantom{a}
\printpage[name = MaySSA1, page = 5, y range={0}{8}, x range={0}{10}]
\phantom{b}
}
There is a version of the May spectral sequence converging to the Adams $E_2$--page for $\pi_* kO^\wedge_2$, beginning with just $\F_2[h_{10}, h_{11}, h_{20}]$.
This was the computation we preformed in \Cref{ComplicatedExampleSec}, recalled at right.
Granting that this is indeed the Adams $E_2$--page for the homotopy of $kO^\wedge_2$, we can read off its homotopy groups:\marginnote{Note: they're periodic!}
\[\begin{array}{c|cccccccc}
n \pmod{8} & 0 & 1 & 2 & 3 & 4 & 5 & 6 & 7 \\
\pi_n kO^\wedge_2 & \Z_2 & \Z/2 & \Z/2 & 0 & \Z_2 & 0 & 0 & 0.
\end{array}\]
More than this, we can use the algebra structure of the May $E_\infty$--page to present it as a ring: \[\pi_* kO^\wedge_2 = \left. \Z_2[h_{11}, b_{20}', h_{20}^4] \middle/ \left( \begin{array}{c} 2 h_{11} = 0, \quad h_{11}^3 = 0, \\ b_{20}' h_{11} = 0, \quad (b_{20}')^2 = 4 \cdot h_{20}^4 \end{array} \right) \right. \]
\end{remark}




\section{Hopf invariants and EHP fiber sequences}

Our goal today is to construct some important maps, called \define{Hopf invariants}, and to identify their fibers in one extremely important case.

The main ingredient is the following reflection of algebra in topology:
\begin{theorem}[James]\marginnote{\citep[Proposition 4.2.2]{Neisendorfer}}
\marginnote{%
Though we will not recount it, the proof of this is surprisingly easy.
The idea is to show it rational homology and in mod--$p$ homology for all $p$, then use Hurewicz.
The whole business is modeled on the tensor algebra functor $T(M) = \bigoplus_j M^{\otimes j}$.
}
For $X$ connected, there is an equivalence
\[\Susp(\Loops \Susp X) \simeq \Susp\left(\bigvee_{j=1}^\infty X^{\sm j}\right). \qed\]
\end{theorem}

\noindent
This splitting is highly nontrivial, and the inverse map that James's theorem guarantees is full of interesting information.
\marginnote{\citep[pg.\ 122--124]{Neisendorfer}}
For instance, the map \[\Susp \Loops \Susp X \xrightarrow\simeq \Susp\left(\bigvee_j X^{\sm j}\right) \to \Susp X^{\sm 2}\] is adjoint to a map \[\Loops \Susp X \xrightarrow h \Loops \Susp X^{\sm 2}\] called the \define{Hopf invariant}.
\marginnote{
This construction is important, for instance, in the vector fields on spheres problem: an $H$--space structure on a sphere gives rise to an interesting map
\begin{center}
\begin{tikzcd}[ampersand replacement=\&]
CS^n \times S^n \arrow{rr} \& \& CS^n \\
\& S^n \times S^n \arrow{lu} \arrow{ld} \arrow["\mu"]{rr} \& \& S^n \arrow{lu} \arrow{ld} \\
S^n \times CS^n \arrow{rr} \& \& CS^n,
\end{tikzcd}
\end{center}
hence a map
\begin{center}
\begin{tikzcd}[ampersand replacement=\&]
S^n * S^n \arrow["H\mu"]{r} \arrow[equal]{d} \& \Susp S^n \arrow[equal]{d} \\
S^{2n+1} \arrow["H\mu"]{r} \& S^{n+1}
\end{tikzcd}
\end{center}
which interacts beautifully with $h$.
}

\begin{theorem}\label{HopfFiberOfLoopsSn}
\todo{The statement of this theorem requires that we actually do localization, not rationalization, back in that section.}
For $X = S^n$, there is a $2$--local exact sequence \[S^n \to \Loops \Susp S^n \xrightarrow h \Loops \Susp S^{2n}.\]
\end{theorem}
\begin{proof}
We will use the associated Serre spectral sequence for integral cohomology.
Recall that in \Cref{CohLoopsSphere} we computed
\begin{align*}
H^* \Loops S^{2n+1} & \cong \Gamma[x_{2n}], &
H^* \Loops S^{2n} & \cong \Lambda[e_{2n-1}] \otimes \Gamma[x_{4n-2}].
\end{align*}
We thus know the cohomology of the base and of the total space, and we know a basic relation among them by the edge homomorphism of \Cref{EdgeHomDefn}.
If we can show the edge homomorphism is injective,
\marginnote{Or, equivalently, if we can show that the edge homomorphism is surjective in homology.}
then the spectral sequence will have to collapse.

Since the cohomology changes based on the parity of $n$, we break into those two cases.
\marginnote{\citep[Corollary 4.4.3]{Neisendorfer}}
Let's begin with $n = 2m + 1$.
In this case, we have
\begin{center}
\begin{tikzcd}
H^* (\Loops \Susp S^{2n}; \Z) \arrow{r} \arrow[equal]{d} & H^* (\Loops \Susp S^n; \Z) \arrow[equal]{d} \\
\Gamma[x_{2n}] \arrow{r} & \Gamma[y_{2n}] \otimes \Lambda[e].
\end{tikzcd}
\end{center}
If we can show $x_{2n} \mapsto y_{2n}$, the algebra structure will finish the job.
This follows from the splitting itself: $h$ started life as \[\Susp \Loops \Susp S^n \to \Susp(S^n)^{\sm 2},\] which is an isomorphism in cohomology degree $2n+1$, so that $h$ has this property in cohomology degree $2n$.
Since the Serre spectral sequence then collapses, in order for its $E_2$--page to have the correct form, the cohomology of the fiber $F$ must be \[H^* F \cong \Susp^n \Z.\]
\marginnote{In general, the identification of the fiber is harder because the cohomology of $X$ is not so sparse.}
Since $F$ is simply connected, we thus learn $F \simeq S^n$.

Next, we turn to the case where $n = 2m$ is even.
\marginnote{\citep[Proposition 4.4.4]{Neisendorfer}}
This time the same map acts by
\begin{align*}
\Gamma[x_{2n}] & \to \Gamma[y_n], \\
x_{2n} & \mapsto \frac{1}{2} y_n^2.
\end{align*}
The algebra structure then gives \[x_{2n}^{[k]} = \frac{1}{k!} x_{2n}^k \mapsto \frac{1}{k!} \left(\frac{1}{2} y_n^2\right)^k = \frac{(2k)!}{2^k k!} y_n^{[2k]},\] which is a unit \emph{$2$--locally}.
\todo{Why is this being a unit relevant?}
From this, running the Serre spectral sequence with $2$--local coefficients, we learn \[H^*(F; \Z_{(2)}) \cong \Susp^n \Z_{(2)}.\]
From this, we conclude $F_{(2)} \simeq S^n_{(2)}$.
\end{proof}

\begin{remark}
In fact, we can even identify the exact extension as a familiar \emph{map}.
\marginnote{
The map ``$e$'' is usually called this as an abbreviation of ``Einh\"angung'', German for ``suspend''.
The continuation of this fiber sequence is \[\cdots \to \Loops S^n \xrightarrow e \Loops^2 S^{n+1} \xrightarrow h \Loops^2 S^{2n+1} \xrightarrow p S^n \xrightarrow e \ldots,\] where ``$p$'' is short for ``Whitehead product''.
The enterprising unstable homotopy theorist can read more in Neisendorfer's book.
}
Simply by connectivity, the Freudenthal map \[S^n \xrightarrow e \Loops \Susp S^n\] becomes null when postcomposed with \[\Loops \Susp S^n \to \Loops \Susp S^{2n}.\]
However, we also know by Freudenthal's theorem that the map $e$ is a cohomology isomorphism in degree $n$.
It follows that its factorization \[S^n \xrightarrow\simeq F \to \Loops \Susp S^n\] is a homotopy equivalence.
\end{remark}




\section{Calculations in the EHP spectral sequence}

\begin{center}
\textbf{In this section, we will implicitly $2$--localize everything.}
\end{center}

Last time, we produced an exact sequence \[S^n \to \Loops S^{2n+1} \to \Loops S^{n+1},\] and \Cref{LexseqInvolvingP} gives rise to exact sequences \[\Loops^m S^n \to \Loops^{m+1} S^{2n+1} \to \Loops^{m+1} S^{n+1}\] for all $m \ge 0$.
Judicious choice of $m$ allows us to knit these exact sequences together into a filtered object:

\begin{figure*}
\begin{center}
\begin{tikzcd}
\Loops^1 S^1 \arrow{r} \arrow[equal]{d} & \Loops^2 S^2 \arrow{r} \arrow{d} & \Loops^3 S^3 \arrow{r} \arrow{d} & \cdots \arrow{r} & \Loops^{n-1} S^{n-1} \arrow{r} \arrow{d} & \Loops^n S^n \arrow{r} \arrow{d} & \cdots \arrow{r} & \Loops^\infty \S \\
\Loops^1 S^1 & \Loops^2 S^3 & \Loops^3 S^5 & \cdots & \Loops^{n-1} S^{2n-3} & \Loops^n S^{2n-1} & \cdots.
\end{tikzcd}
\end{center}
\end{figure*}

\begin{definition}\marginnote{\citep[Proposition 1.5.7]{Ravenel}}
Taking homotopy groups then gives the \define{EHP spectral sequence} of signature \[E^1_{s, t} = \pi_s \Loops^t S^{2t-1} \cong \pi_{s+t} S^{2t-1} \Rightarrow \pi_* \S.\]
\end{definition}

This type signature may seem ridiculous: it takes as input all of the unstable homotopy groups of spheres, and it computes the stable ones---which, by \Cref{FreudenthalThm} we must have already known.
We will see that the utility of the EHP spectral sequence ultimately lies \emph{how} it performs its computation with only \emph{partially} specified inputs, rather than the impossible situation of perfect information.

Indeed, for all our work, we can't fill out very much of this spectral sequence: we know a few stable groups and we know $\pi_{\le 6} S^3$.
\marginnote[-5\baselineskip]{
\begin{sseqdata}[name = EHP, lax degree, yscale = 0.4, x range = {0}{7}, y range = {1}{9}, xscale = 0.4]
% TODO: there's a weird magic number here in the class labels specification

% stable info
% Z
\foreach \i in {1,...,8} {
    \class(\i,\i+1)
}
% Z/2
\foreach \i in {2,...,8} {
    \class[fill](\i+1,\i+1)
}
% Z/2
\foreach \i in {3,...,8} {
    \class[fill](\i+2,\i+1)
}
% Z/8
\foreach \i in {4,...,8} {
    \class[fill, circlen=3](\i+3,\i+1)
}

% S^1
\class(0,1)

% known part of S^3
\class[fill](2,2)
\class[fill](3,2)
\class[fill, circlen=2](4,2)
\end{sseqdata}
\printpage[name = EHP, page = 0]
}
These known groups are tabulated at right.
The $j$\textsuperscript{th} column of the $E_\infty$--page of the spectral sequence forms the associated graded of a filtration on $\pi_j \S$.
These facts together are enough to make a small observation: the $s = 1$ column is empty aside from a torsion-free group in position $(1, 2)$, hence this group must receive a differential if it's to compute the expected $\pi_1 \S \cong \Z/2$.
This is actually part of a family of differentials:

\begin{lemma}\marginnote{\citep[Proposition 1.5.13]{Ravenel}}
The differentials \[d^1\co E^1_{s,s+1} \to E^1_{s-1,s}\] along the main diagonal are given by multiplication by $1 + (-1)^s$.
\end{lemma}
\begin{proof}
These differentials have a geometric origin.
There is a map \[J_n\co O(n) \to \Loops^n S^n\] given by restricting attention to the unit sphere and suspending once.
\marginnote{The suspension here is crucial: the behavior of a point in $O(n)$ on the unit sphere need not fix any particular point, so does not map to the pointed loopspace without first passing to the suspension.}
These maps assemble into a map of exact sequences:
\marginnote{This collection of exact sequences can also be stitched together to form a filtered object, whose associated spectral sequence takes the form \[E^1_{s,t} = \pi_s S^{t-1} \Rightarrow \pi_s O(\infty).\]}
\begin{center}
\begin{tikzcd}
\Loops^{n-1} S^{n-1} \arrow{r} & \Loops^n S^n \arrow{r} & \Loops^n S^{2n-1} \\
O(n-1) \arrow{r} \arrow["J_{n-1}"]{u} & O(n) \arrow{r} \arrow["J_n"]{u} & S^{n-1}. \arrow{u}
\end{tikzcd}
\end{center}
\marginnote{Actually, the right-hand square only commutes after looping once, but it commutes as-is after applying $\pi_*$.}
The behavior of this map and of the exact sequence involving the orthogonal groups is complex, but one may check geometrically that the map $S^{n-1} \to \Loops^n S^{2n-1}$ selects the bottom cell.

To avoid the complexity of the orthogonal groups, we may restrict to an even simpler situation: there is a map \[\RP^{n-1} \to O(n)\] which sends a line in $\R^n$ to the reflection along that line.
These maps also arrange into a commutative diagram
\begin{center}
\begin{tikzcd}
O(n-1) \arrow{r} & O(n) \arrow{r} & S^{n-1} \\
\RP^{n-2} \arrow{r} \arrow{u} & \RP^{n-1} \arrow{r} \arrow{u} & S^{n-1}, \arrow[equal]{u}
\end{tikzcd}
\end{center}
but this time the bottom row is a \emph{coexact} sequence.
The functorial Hurewicz isomorphism $\pi_{n-1} S^{n-1} \cong H_{n-1}(S^{n-1}; \Z)$ means that, for the purposes of understanding this bottommost group, we can study the behavior of the exact sequence in homology.
This family of coexact sequences assembles into the cellular filtration of $\RP^\infty$, and the associated spectral sequence is that of \Cref{CellularHomologyWorks}.
The differential in $C^{\mathrm{cell}}_*(\RP^\infty; \Z)$ is given by the claimed formula.
\marginnote[-10\baselineskip]{
\begin{sseqdata}[name = EHP, update existing]
\d1(2,3)(1,2)
\d1(4,5)(3,4)
\d1(6,7)(5,6)
\d1(8,9)(7,8)
\structline(3,2)(3,3)
\structline[densely dotted](3,3)(3,4)
\end{sseqdata}
\printpage[name = EHP, page = 0]
}
\end{proof}

The term $E^1_{2.2}$ we know from many sources: $\pi_4 S^3$ lies in the stable range; we have computed $\pi_4 S^3$ using Serre's method; if the EHPSS converges to $\Z/2$ in that column, then \emph{at least} a $\Z/2$ must be present; and a fourth method which we now describe.
Rather than studying the full filtered object $\colim_n \Loops^n S^n$, we can stop at any finite stage.
This produces a spectral sequence converging to $\pi_* \Loops^n S^n$, and its $E^1$--page looks like a horizontal trunction of the full EHP spectral sequence, with the bottom $n$ rows surviving.
\marginnote[-4\baselineskip]{
\phantom{weird bug}
\printpage[name = EHP, page = 0, y range={1}{3}]
}
For example, we have produced the truncation converging to $\pi_{*+3} S^3$.
The first four columns compute
\begin{align*}
\pi_3 S^3 & \cong \Z, &
\pi_4 S^3 & \cong \Z/2, &
\pi_5 S^3 & \cong \Z/2, &
\pi_6 S^3 & \cong \Z/4,
\end{align*}
confirming the unstable calculation asserted at the end of \Cref{SerresMethod}.

In order to calculate any further in this spectral sequence---that is, in order to access the $s = 4$ column---we must understand the unknown group $E^1_{3,4} \cong \pi_7 S^5$.
This, too, can be accessed by truncation,
\marginnote{
\phantom{weird bug}
\printpage[name = EHP, page = 0, y range={1}{5}]
}
where it is found to be $\Z/2$.

These kinds of observations power significant inductive computation.
They also power inductive proofs of \emph{qualitative} results, as in the following:

\begin{corollary}
The rational homotopy groups of spheres are given by
\begin{align*}
\pi_* S^{2n+1} \otimes \Q & \cong \Susp^{2n+1} \Q, &
\pi_* S^{2n} \otimes \Q & \cong \Susp^{2n} \Q \oplus \Susp^{4n-1} \Q.
\end{align*}
\end{corollary}
\begin{proof}
We begin by studying the torsion-free groups on the main diagonal.
After the $E_2$--page, all those groups other than $E^1_{0,1}$ get replaced by torsion groups: either $0$ or $\Z/2$.
When truncating the spectral sequence to study the homotopy groups of the even-dimensional sphere $S^{2n}$, the torsion-free group in position $E^1_{2n-1, 2n}$ is no longer able to receive a differential, hence survives to contribute a torsion-free term to $\pi_{4n-1} S^{2n}$.
This phenomenon does not occur for odd-dimensional spheres: the truncation in that case deletes only pairs of terms on the main diagonal.

We must additionally argue that the region below the main diagonal is populated only by finite torsion groups.
Using the truncation method, each term $E^1_{s, t} = \pi_{s+t} S^{2t-1}$ is determined by the terms $E^1_{\le t-s-1, \le 2t-1}$---that is, the distance of a term from the main diagonal determines what vertical strip of the spectral sequence is needed to understand it.
The march of the diagonal line down-and-right outpaces the march of the vertical line to the right, powering the inductive claim that if the corner of the $E^\infty$--page is finite and torsion (apart from $E^\infty_{0,1}$), then the entire $E^\infty$--page is finite and torsion (apart from $E^\infty_{0,1}$).
\end{proof}

\marginnote[-10\baselineskip]{
We also include a picture of the fully specified EHP spectral sequence in this range~\citep[Figure 1.5.9]{Ravenel}.
\begin{sseqdata}[name = EHP, update existing]
\class[fill](5,2)
\class[fill](6,2)
\class[fill](4,3)
\class[fill, circlen=3](5,3)
\class[fill](6,3)
\class[fill](7,3)
\class[fill, circlen=3](6,4)

\d1(5,3)(4,2)
\d1(6,3)(5,2)
\d1(7,3)(6,2)
\d2(5,5)(4,3)
\replaceclass[fill, circlen=2](5,3)
\d2(6,5)(5,3)
\d1(7,5)(6,4)
\replaceclass(5,6)
\d2(5,6)(4,4)
\d2(6,6)(5,4)

\structline[densely dotted](7,8)(7,7)
\structline(7,7)(7,6)
\structline(7,6)(7,5)
\end{sseqdata}
\printpage[name = EHP, page = 0]
}


\begin{subappendices}

\section{Bott periodicity}

\end{subappendices}
