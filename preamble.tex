% -*- root: main.tex -*-

\documentclass[twoside,nofonts,notoc]{tufte-book}

\setcounter{secnumdepth}{2}
\setcounter{tocdepth}{1}

\usepackage{amsmath,amsthm}
\usepackage[urw-garamond]{mathdesign}
\usepackage[protrusion=true,expansion=true]{microtype}
\usepackage{stmaryrd}

\usepackage{todonotes}
% \usepackage{spectralsequences}
\usepackage{tikz-cd}
\usepackage{spectralsequences}

\sseqset{class labels = { above right = 0.2em }}

\usepackage{cleveref}

% for bigast
\usepackage{relsize}

% for per-chapter appendices
\usepackage{appendix}
\usepackage{chngcntr}
\usepackage{etoolbox}
\AtBeginEnvironment{subappendices}{%
\newpage
\counterwithin{figure}{section}
\counterwithin{table}{section}
}

% -*- root: main.tex -*-

% https://tex.stackexchange.com/questions/285996/tikz-circle-with-color-transition

\usepackage{xparse}

\ExplSyntaxOn

\keys_define:nn { colour_transition_circle } {
    inner   .fp_set:N   = \l__inner_radius,
    inner   .initial:n  = {2},
    outer   .fp_set:N   = \l__outer_radius,
    outer   .initial:n  = {3},
    angle   .fp_set:N   = \l__start_angle,
    angle   .initial:n  = {0}
}

\NewDocumentCommand \ColourTransitionCircle { O{} m } {
\group_begin:
    \keys_set:nn { colour_transition_circle } {#1}
    \clist_clear:N \l_tmpa_clist
    \clist_map_inline:nn {#2} {
        \clist_put_right:Nn \l_tmpa_clist {##1}
        %\clist_put_right:Nn \l_tmpa_clist {##1}
    }
    \exp_args:Nx \col_trans_circ:n \l_tmpa_clist
\group_end:
}

\cs_new_protected:Npn \col_trans_circ:n #1 {
    \int_step_inline:nnnn {1} {1} {\clist_count:n {#1} - 1} {
        \path[top~color=\clist_item:nn {#1} {##1}, bottom~color=\clist_item:nn {#1} {##1+1}, shading~angle={270-(180-360/\clist_count:n {#1})/2+(##1-1)*360/\clist_count:n {#1}+\fp_use:N \l__start_angle}] ({\fp_use:N \l__inner_radius*cos((##1-1)*360/\clist_count:n {#1}+\fp_use:N \l__start_angle)},{\fp_use:N \l__inner_radius*sin((##1-1)*360/\clist_count:n {#1}+\fp_use:N \l__start_angle)}) arc[radius = \fp_use:N \l__inner_radius, start~angle={(##1-1)*360/\clist_count:n {#1}+\fp_use:N \l__start_angle}, delta~angle=360/\clist_count:n {#1}] -- ({\fp_use:N \l__outer_radius*cos(##1*360/\clist_count:n {#1}+\fp_use:N \l__start_angle)},{\fp_use:N \l__outer_radius*sin(##1*360/\clist_count:n {#1}+\fp_use:N \l__start_angle)}) arc[radius = \fp_use:N \l__outer_radius, start~angle={##1*360/\clist_count:n {#1}+\fp_use:N \l__start_angle}, delta~angle=-360/\clist_count:n {#1}] -- cycle;
    }
    \path[top~color=\clist_item:nn {#1} {\clist_count:n {#1}}, bottom~color=\clist_item:nn {#1} {1}, shading~angle={180-180/\clist_count:n {#1}+\fp_use:N \l__start_angle}]({\fp_use:N \l__inner_radius*cos((\clist_count:n {#1}-1)*360/\clist_count:n {#1}+\fp_use:N \l__start_angle)},{\fp_use:N \l__inner_radius*sin((\clist_count:n {#1}-1)*360/\clist_count:n {#1}+\fp_use:N \l__start_angle)}) arc[radius = \fp_use:N \l__inner_radius, start~angle={(\clist_count:n {#1}-1)*360/\clist_count:n {#1}+\fp_use:N \l__start_angle}, delta~angle=360/\clist_count:n {#1}] -- ({\fp_use:N \l__outer_radius*cos(\clist_count:n {#1}*360/\clist_count:n {#1}+\fp_use:N \l__start_angle)},{\fp_use:N \l__outer_radius*sin(\clist_count:n {#1}*360/\clist_count:n {#1}+\fp_use:N \l__start_angle)}) arc[radius = \fp_use:N \l__outer_radius, start~angle={\clist_count:n {#1}*360/\clist_count:n {#1}+\fp_use:N \l__start_angle}, delta~angle=-360/\clist_count:n {#1}] -- cycle;
}

\ExplSyntaxOff


\newcommand{\CatOf}[1]{\mathsf{#1}}
\newcommand{\define}[1]{\textit{#1}}
\newcommand{\OS}[2]{\underline{\smash{#1}}_{#2}}

\newcommand{\A}{\mathcal A}
\renewcommand{\C}{\mathbb C}
\newcommand{\CC}{\mathcal{C}}
\newcommand{\CP}{\C \mathrm{P}}
\newcommand{\F}{\mathbb F}
\newcommand{\GL}{\mathrm{GL}}
\renewcommand{\H}{\mathbb H}
\newcommand{\HP}{\H \mathrm{P}}
\newcommand{\I}{\mathbb I}
\newcommand{\Q}{\mathbb Q}
\newcommand{\R}{\mathbb R}
\newcommand{\RP}{\R \mathrm{P}}
\renewcommand{\S}{\mathbb S}
\newcommand{\Z}{\mathbb Z}

\newcommand{\op}{\mathrm{op}}
\newcommand{\Sp}{\mathit{Sp}}
\newcommand{\Sq}{\mathrm{Sq}}

\newcommand{\bigast}{\mathop{\scalebox{2.5}{\raisebox{-0.35ex}{$\ast$}}}}
\newcommand{\co}{\colon\thinspace}
\renewcommand{\emptyset}{\varnothing}
\renewcommand{\epsilon}{\varepsilon}
\newcommand{\eps}{\epsilon}
\newcommand{\from}{\leftarrow}
\newcommand{\Loops}{\Omega}
\newcommand{\mmod}{/\!\!/}
\renewcommand{\phi}{\varphi}
\newcommand{\sm}{\wedge}
\newcommand{\Susp}{\Sigma}
\newcommand{\<}{\langle}
\renewcommand{\>}{\rangle}

\DeclareMathOperator{\Aut}{Aut}
\DeclareMathOperator{\colim}{colim}
\DeclareMathOperator{\Ext}{Ext}
\DeclareMathOperator{\fib}{fib}
\DeclareMathOperator{\id}{id}
\DeclareMathOperator{\im}{im}
\DeclareMathOperator{\Tor}{Tor}

\theoremstyle{plain}
\newtheorem{dummy}{Dummy}[section]
\newtheorem{theorem}[dummy]{Theorem}
\newtheorem*{theorem*}{Theorem}
\newtheorem*{lemma*}{Lemma}
\newtheorem{proposition}[dummy]{Proposition}
\newtheorem{lemma}[dummy]{Lemma}
\newtheorem{corollary}[dummy]{Corollary}
\newtheorem{conjecture}[dummy]{Conjecture}
\theoremstyle{definition}
\newtheorem{definition}[dummy]{Definition}
\newtheorem{construction}[dummy]{Construction}
\newtheorem{warning}[dummy]{Important Warning}
\newtheorem*{note}{Important Note}
\theoremstyle{remark}
\newtheorem{remark}[dummy]{Remark}
\newtheorem{example}[dummy]{Example}
\newtheorem{problem}[dummy]{Problem}
\newtheorem{task}[dummy]{Task}

\title[Topology from an Algebraic Viewpoint]{%
Topology from an \\ Algebraic Viewpoint %
\\ \vspace{3\baselineskip}{\tiny \today}}
\author{Eric Peterson}
