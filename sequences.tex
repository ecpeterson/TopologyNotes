% -*- root: main.tex -*-

\chapter{Homotopy Types and Exact Sequences}

One of our overall goals in this text is to treat spaces as if they were algebraic objects, to dissect and assemble them by methods analogous to those we use with, say, modules.
There are at least two routes to effecting this.
One is to make a re-definition of a ``space'' which is inherently algebraic (or combinatorial).
This eases the import of algebraic operations into this new context, but it requires one to link the chosen alternative definition with the traditional notion of a space.\todo{Cite? Simplicial sets, co--$E_\infty$ chain complexes?}
We will take a second route: beginning with some of the traditional tools of algebraic topology, we will use their formal properties as impetus to re-envision them as the desired algebraic operations.
These will become the fundamental methods of dissection and assembly used in all later chapters.




\section{The categories of spaces, pointed spaces, and pairs}

Today we set up the basics of category theory and introduce a few basic examples, which amounts to setting the stage on which the rest of the course will play out.
One of our express intentions in this course is to try to get away with doing very little geometry or ``point-set'' topology, which means a lot of this Lecture will concern itself with boldly asserting such facts in category-theoretic language, so that they can be guiltlessly referred to later.

We begin with some basic operations that one may perform on spaces.
The categorical perspective is to understand objects in a category not by intrinsic presentations, like how the points in a space are arranged, but by their extrinsic effects of how they relate to other objects in the category, like the ways a given space can be continuously mapped into another.
With this as a guiding principle, we give categorically-minded interpretations of some common operations performed on spaces:
\begin{description}
    \item[Products]\marginnote{\citep[0.2]{Switzer}}
    Let $X$, $Y$ be spaces.
    Their \define{product} $X \times Y$ is a space such that the set $\CatOf{Spaces}(T, X \times Y)$ of continuous maps $T \to X \times Y$ bijects with with the set $\CatOf{Spaces}(T, X) \times \CatOf{Spaces}(T, Y)$ of pairs $(f_X, f_Y)$ of continuous maps $f_X\co T \to X$ and $f_Y\co Y \to T$.
    A useful mnemonic is that products pull out on the right: \[\CatOf{Spaces}(T, X \times Y) \cong \CatOf{Spaces}(T, X) \times \CatOf{Spaces}(T, Y).\]
    The bijection in question is induced by a pair of maps $\pi_X\co X \times Y \to X$ and $\pi_Y\co X \times Y \to Y$ which save the indicated coordinate and drop the other: given a map $f\co T \to X \times Y$, post-composing with either of these maps gives the two values $f_X = \pi_X \circ f$ and $f_Y = \pi_Y \circ f$ in the pair.%
    \marginnote{There's an amusing circularity to this data: if you assume the bijection to begin with, then you can recover the maps $\pi_X$ and $\pi_Y$ by starting with the identity $\id_{X \times Y} \in \CatOf{Spaces}(X \times Y, X \times Y)$ and following it across the bijection to the pair $(\pi_X, \pi_Y) \in \CatOf{Spaces}(X \times Y, X) \times \CatOf{Spaces}(X \times Y, Y)$.
    This sort of gymnastic forms the foothills of the \define{Yoneda Lemma}.}
    \item[Coproducts]
    The \define{disjoint union} $X \sqcup Y$ is a space with the property \[\CatOf{Spaces}(X \sqcup Y, T) \cong \CatOf{Spaces}(X, T) \times \CatOf{Spaces}(Y, T).\]
    For this reason, the disjoint union is sometimes called the \define{coproduct}, since it pulls out to a product on the left.
    \item[Gluing]\marginnote{\citep[0.1]{Switzer}}
    Let $X$ be a space with a \define{decomposition} $X = \bigcup_j A_j$ into closed subsets.
    Then the set $\CatOf{Spaces}(X, Y)$ bijects with the set \[\{(f_j \in \CatOf{Spaces}(A_j, Y)) : f_j|_{A_j \cap A_k} = f_k|_{A_j \cap A_k}\}\] of morphisms on the various components which agree on the overlaps.
    \item[Quotients]\marginnote{\citep[0.3]{Switzer}}
    For $\sim$ an equivalence relation on $X$, there is a space $X / \sim$ such that the set $\CatOf{Spaces}(X / \sim, Y)$ bijects with the subset \[\{f \in \CatOf{Spaces}(X, Y) \mid x \sim x' \Rightarrow f(x) = f(x')\}\] of those continuous maps which are constant on the partition components of $\sim$.
    As a special case, let $A \subseteq X$ be nonempty, and define $X/A$ by extending the total relation on $A$ by the identity relation on $X$ (i.e., the associated partition consists of singletons and $A$).
    \marginnote{As an edge case, we set $X / \emptyset = X \cup \{*\}$.}
    \todo{Explain the edge case: it's because it's a pushout.}
\end{description}

We mention one further construction which we will be very keen to make use of, and we offset it from those above because of its relative fragility.

\begin{description}
    \item[Function spaces/exponential objects]\marginnote{\citep[0.9--11]{Switzer}}
    For $X$, $Y$ spaces, there is a \define{function spaces} $Y^X$ whose underlying set is $\CatOf{Spaces}(X, Y)$.
    If $X$ is locally compact, then the evaluation map $\mathit{ev}\co Y^X \times X \to Y$ is continuous.
    If $X$ and $Z$ are locally compact as well as Hausdorff, then the currying function $Y^{Z \times X} \to (Y^Z)^X$ is a homeomorphism.
\end{description}

\noindent
We will perpetually arrange to be in the situation where our function-spaces are well-behaved.

\begin{remark}
In each of these cases, the property named uniquely determines the space involved: if $Q$ is some unknown space so that $\CatOf{Spaces}(Q, T)$ naturally bijects with the subset \[\{f \in \CatOf{Spaces}(X, T) \mid x \sim x' \Rightarrow f(x) = f(x') \},\] then there is a homeomorphism $Q \xrightarrow\cong X/\sim$.
\marginnote{This, too, is fundamentally the Yoneda Lemma.}
\end{remark}

\begin{remark}\label{HomSheafRemark}
Together, the coproduct and gluing properties characterize the functor $\CatOf{Spaces}(T, -)$ as a \define{sheaf}.
\todo{Find time later to characterize a sheaf as a continuous functor off of the opposite category.}
\end{remark}

\noindent
These constructions also interact well in most cases of geometric interest:

\begin{lemma}\label{ProdQuotInterchange}\marginnote{\citep[0.4]{Switzer}}
If $Y$ is locally compact, then there is a homeomorphism \[\frac{X \times Y}{\sim \times \id} \xrightarrow{\cong} \left( \frac{X}{\sim}\right) \times Y. \qed\]
\end{lemma}

\begin{remark}
Throughout the course, we will quietly apply \Cref{ProdQuotInterchange} in the case of $Y = I := [0, 1]$ as we ``fatten'' various quotient constructions, as it allows us to elect whether to apply the quotient operation before or after we perform the fattening.
\end{remark}

\noindent
\Cref{ProdQuotInterchange} also encodes \define{relative homotopy} in terms of a quotient:

% \begin{corollary}
% Fix a homotopy $H\co X \times I \to Y$.
% If there is a fixed relation $\sim$ so that for every time $t$, $H$ factors as $X \times \{t\} \to X/\sim \times \{t\} \to Y$ then $H$ factors as a whole: $X \times I \to X/\sim \times I \xrightarrow{\widetilde H} Y$. \qed
% \end{corollary}

\begin{corollary}\marginnote{\citep[0.8]{Switzer}}
If $A \subseteq X$ is closed and $H(a, t) = H(a', t)$ for all $a, a' \in A$ and $t \in I$, then it factors as $X \times I \to X/A \times I \to Y$. \qed
\end{corollary}

Relative homotopy provides a convenient segue into two other categories of interest: the category of relative pairs and the category of pointed spaces.

\begin{definition}
The category of \define{relative pairs} has objects given by inclusions $A \subseteq X$, and a morphism $(A \subseteq X) \to (B \subseteq Y)$ between two such pairs is given by a continuous map $f \in \CatOf{Spaces}(X, Y)$ which has the property $f(A) \subseteq B$.
\marginnote{There is no reason one must stop at a single inclusion.
Chains of inclusions---even infinite chains---are interesting, and they are generally referred to as \define{filtered spaces}.}
These pairs and their maps appear naturally when discussing relative homology or relative homotopy.
\end{definition}

\begin{definition}\marginnote{\citep[0.12]{Switzer}}
The category of \define{pointed spaces}, which we denote by $\CatOf{Spaces}_{*/}$, has objects given by $\{x_0\} \subseteq X$ for some choice of singleton $\{x_0\}$, and a morphism $(\{x_0\} \subseteq X) \to (\{y_0\} \subseteq Y)$ between two such pointed spaces is given by a map $f \in \CatOf{Spaces}(X, Y)$ with the property $f(x_0) = y_0$.
Pointed spaces and their maps arise when defining reduced homology or homotopy groups (e.g., the fundamental group).
This presentation also makes plain that pointed spaces form a full subcategory of relative pairs: they are the special case where the privileged subset is a singleton.
\end{definition}

These categories admit all of the same categorical constructions as $\CatOf{Spaces}$: they have products, coproducts, and quotients, they glue, and one can build function objects.
In many case, if one forgets about the privileged subset and considers just the underlying object of $\CatOf{Spaces}$, the construction even agrees with the one in $\CatOf{Spaces}$.
For instance, the coproduct of pairs is given by \[(A \subseteq X) \sqcup (B \subseteq Y) = (A \sqcup B) \subseteq (X \sqcup Y).\]
The product also has this property: \[(A \subseteq X) \times (B \subseteq Y) = (((A \times Y) \cup (X \times B)) \subseteq X \times Y),\] but the privileged subspace has become more complicated.
One can also define a function object \[((B \subseteq Y)^{(A \subseteq A)} \subseteq (B \subseteq Y)^{(A \subseteq X)}),\] itself a relative pair which we abbreviate to $(B \subseteq Y)^{(A \subseteq X)}$ for sanity's sake.
It satisfies an analogue of currying:%
\marginnote{\citep[Proposition 0.13]{Switzer}}
\begin{align*}
(B \subseteq Y)^{((Z \times A) \cup (C \times X)) \subseteq Z \times X} & = (B \subseteq Y)^{(C \subseteq Z) \times (A \subseteq X)} \\
& = \left( (B \subseteq Y)^{(C \subseteq Z)} \right)^{(A \subseteq X)}.
\end{align*}

In trying to find analogues of these claims in pointed spaces, there is the following snag: the coproduct of two pointed spaces \emph{in the category of relative pairs} is given by \[(\{x_0\} \subseteq X) \sqcup (\{y_0\} \subseteq Y) = (\{x_0, y_0\} \subseteq (X \sqcup Y)),\] but this coproduct has escaped the subcategory $\CatOf{Spaces}_{*/}$.
One can forcefully correct this by quotienting the privileged subspace so that it becomes a point---and, in fact, this recovers the correct coproduct in pointed spaces:

\begin{definition}
For two pointed spaces $\{x_0\} \subseteq X$ and $\{y_0\} \subseteq Y$, their \define{wedge sum} $X \vee Y$ is the coproduct in the category of pointed spaces: \[\CatOf{Spaces}_{*/}(X \vee Y, T) = \CatOf{Spaces}_{*/}(X, T) \times \CatOf{Spaces}_{*/}(Y, T).\]
In terms of the pair, it is given by \[X \vee Y = \frac{\{x_0, y_0\} \subseteq (X \sqcup Y)}{\{x_0, y_0\}}.\]
\end{definition}

The relationship between the products in relative pairs and in pointed spaces is much looser: the product of pointed spaces is given by \[(\{x_0\} \subseteq X) \times (\{y_0\} \subseteq Y) = (\{(x_0, y_0)\} \subseteq (X \times Y)),\] where the complexity in the privileged subspace has seemingly evaporated.
\marginnote{The reason why we didn't have to quotient this time is related to the chirality of the limits constructions involved: coproducts and quotients are both colimits, but a product is a limit.}
However, we find ourselves in a further awkward position when we investigate the currying law for the function object for pointed spaces:
\begin{align*}
(\{y_0\} \subseteq Y)^{((Z \times \{x_0\}) \cup (\{z_0\} \times X)) \subseteq Z \times X} & = (\{y_0\} \subseteq Y)^{(\{z_0\} \subseteq Z) \times (\{x_0\} \subseteq X)} \\
& = \left( (\{y_0\} \subseteq Y)^{(\{z_0\} \subseteq Z)} \right)^{(\{x_0\} \subseteq X)}.
\end{align*}
In particular, the exponent on the far left is \emph{not} the product in pointed spaces---indeed, it isn't a pointed space at all.
We can, again, force it to become one:

\begin{definition}\marginnote{\citep[2.4]{Switzer}}
The \define{smash product} of two pointed spaces $\{x_0\} \subseteq X$ and $\{z_0\} \subseteq Z$ is given by \[(\{x_0\} \subseteq X) \sm (\{z_0\} \subseteq Z) := \frac{X \times Z}{(X \times \{z_0\}) \cup (\{x_0\} \times Z)} = \frac{X \times Z}{X \vee Z}.\]
This gives a \define{monoidal structure} on $\CatOf{Spaces}_{*/}$ which is \emph{not} the Cartesian one.
\marginnote{That is: the one given by the ordinary product.}
\end{definition}

\begin{corollary}
For pointed spaces $X$, $Y$, and $Z$, the currying law for function objects takes the form
\begin{align*}
Y^{Z \sm X} & \cong (Y^Z)^X, &
\CatOf{Spaces}_{*/}(Z \sm X, Y) & \cong \CatOf{Spaces}_{*/}(X, Y^Z).
\end{align*}
\end{corollary}

\begin{definition}\label{HoCatDefn}%
\marginnote{\citep[0.7]{Switzer}}
Finally, we introduce the \define{homotopy category of (pointed) spaces}, $h\CatOf{Spaces}_{*/}$, whose objects are the same as that of $\CatOf{Spaces}_{*/}$ but whose morphism sets are given by quotients \[h\CatOf{Spaces}_{*/}(X, Y) = \frac{\CatOf{Spaces}_{*/}(X, Y)}{\text{$f \sim g$ when there is a homotopy $H \co X \sm I_+ \to Y$ between them}}.\]
\todo{Introduce the notation $X_+$, and perhaps the adjunctions between the different categories.}
Because we will work in this category so often, we will abbreviate this mapping set by \[h\CatOf{Spaces}_{*/}(X, Y) = [X, Y].\]
\end{definition}




\section{Perspectives on the fundamental group}

\begin{definition}\marginnote{\citep[2.1]{Switzer}}
The \define{pathspace} $PX$ of $X$ is given by $PX = X^I$, $I = [0, 1]$ the closed interval.
We write $\pi_0(X)$ for the set of path-components of $X$, i.e., the quotient of $X$ by the relation $\sim$, where $x \sim x'$ when there exists $\gamma \in X^I$ such that $\gamma(0) = x$, $\gamma(1) = x'$.
\end{definition}

\begin{remark}
The mapping set given in \Cref{HoCatDefn} can be equivalently described by $[X, Y] = \pi_0 Y^X$.
By specializing to $X = *$, we also have $\pi_0 Y = [S^0, Y]$ for $S^0 = (\{\pm 1\}, 1) = *_+$.
\end{remark}

We can use the adjunctions from last time to give several equivalent definitions of the fundamental group:
\begin{align*}
\pi_1(Y) & := \{\text{homotopy classes of pointed loops in $X$}\} \\
& = [S^1, Y] = [S^0 \sm S^1, Y] = [S^0, Y^{(S^1)}] = \pi_0 Y^{S^1}.
\end{align*}
One might wonder what properties of $X$ and $Y$ make $[X, Y]$ into a group, since we know that $[S^1, -]$ and $[S^0, (-)^{S^1}]$ are group-valued.
As we intend to explore this essentially categorical question using yet more category theory, it will be helpful to have a categorical definition of a group.

\begin{definition}\marginnote{\citep[pg.\ 14]{Switzer}}
A \define{group} is a pointed set $G$ with pointed maps $\mu\co G \times G \to G$ and $\chi\co G \to G$ which make the following diagrams commute:
\begin{center}
\begin{tikzcd}
G \times G \times G \arrow["\mu \times \id"]{r} \arrow["\id \times \mu"]{d} & G \times G \arrow["\mu"]{d} \\
G \times G \arrow["\mu"]{r} & G,
\end{tikzcd}
\begin{tikzcd}
G \arrow[equal]{rd} \arrow["\eta \times \id"]{r} & G \times G \arrow["\mu"]{d} & G \arrow["\id \times \eta"']{l} \arrow[equal]{ld} \\
& G,
\end{tikzcd}
\begin{tikzcd}
G \arrow["0"]{d} \arrow["\chi \times \id"]{r} & G \times G \arrow["\mu"]{d} & G \arrow["\id \times \chi"']{l} \arrow["0"]{d} \\
* \arrow["\eta"]{r} & G & * \arrow["\eta"']{l} .
\end{tikzcd}
\end{center}
\end{definition}

\begin{lemma}\label{RepresentableGroupsLemma}
If $G$ is a group object in a category $\CatOf{C}$, then $\CatOf{C}(-, G)$ is a functor from $\CatOf{C}^{\op}$ to $\CatOf{Groups}$.
\marginnote{In fact, this is biconditional: if a representable functor factors through $\CatOf{Groups}$, then the representing object inherits the structure of a group object.}
\todo{I think this lecture is a better place to introduce Yoneda than the previous one.}
\end{lemma}
\begin{proof}
Recall that the defining property of the product $X \times Y$ is \[\CatOf{C}(T, X \times Y) \cong \CatOf{C}(T, X) \times \CatOf{C}(T, Y).\]
For a fixed test object $T$, the functor $\CatOf{C}(T, -)$ applied to the multiplication $\mu$ on $G$ gives \[\CatOf{C}(T, G) \times \CatOf{C}(T, G) \cong \CatOf{C}(T, G \times G) \xrightarrow{\mu_*} \CatOf{C}(T, G),\] and in this way $\mu_*$ becomes a multiplication on $\CatOf{C}(T, G)$.
The inversion $\chi$ on $G$ similarly induces an inversion $\chi_*$ on $\CatOf{C}(T, G)$, and together these make the group object diagrams commute.
\marginnote{For a pair of maps $f, g\co X \to K$, it is sometimes helpful to factor $(f, g)\co X \to K \times K$ through the diagonal on $X$, as in $X \xrightarrow\Delta X \times X \xrightarrow{f \times g} K \times K$, as this lets on leverage the bifunctoriality of $\times$.}
\end{proof}

\begin{definition}\marginnote{\citep[Definition 2.9]{Switzer}}
An \define{$H$--group} $K$ is a pointed space with maps $\mu$ and $\chi$ satisfying the group diagrams up to homotopy (i.e., as diagrams in $h\CatOf{Spaces}_{*/}$).
\end{definition}

\begin{corollary}\marginnote{\citep[Proposition 2.14]{Switzer}}
The functor $[-, K]$ is valued in groups. \qed
\end{corollary}

\begin{example}\marginnote{\citep[Examples 2.15.ii]{Switzer}}
The usual verification that the fundamental group of a space is indeed a group can be viewed as giving an $H$--space structure on $Y^{S^1} =: \Loops Y$.
The multiplication map $\mu$ is given by rescaling and concatenating two loops, and the inversion map $\chi$ is given by running a loop backward.
Hence, not only is the fundamental group $\pi_1 Y = \pi_0 \Omega Y = [S^0, \Omega Y]$ a group, but actually $[X, \Omega Y]$ is a group for any choice of $X$.
\end{example}

What about the other formulation?
We also have $\pi_1 Y = [S^1, Y]$, which is a group-valued functor even when $Y$ is freely varying, and so one might suspect the magic does not reside in the $Y$--dependent object ``$\Loops Y$'' but rather in $S^1$ alone.
In order to address this by the same tactic, we must confront the structure required on $X$ in order to make $[X, Y]$ into a group.
Using the identity $[X \vee X, Y] \cong [X, Y] \times [X, Y]$, we are led to the following definition:

\begin{definition}\marginnote{\citep[Definition 2.16]{Switzer}}
An \define{$H$--cogroup} $K$ has pointed maps $\mu'\co K \to K \vee K$ and $\chi'\co K \to K$ which make the following diagrams commute in the homotopy category:
\begin{center}
\begin{tikzcd}
K \vee K \vee K \arrow["\mu' \vee \id", leftarrow]{r} \arrow["\id \vee \mu'", leftarrow]{d} & K \vee K \arrow["\mu'", leftarrow]{d} \\
K \vee K \arrow["\mu'", leftarrow]{r} & K,
\end{tikzcd}
\begin{tikzcd}
K \arrow[equal]{rd} \arrow["0 \vee \id", leftarrow]{r} & K \vee K \arrow["\mu'", leftarrow]{d} & K \arrow["\id \vee 0"', leftarrow]{l} \arrow[equal]{ld} \\
& K,
\end{tikzcd}
\begin{tikzcd}
K \arrow[leftarrow, "\eta"]{d} \arrow["\chi \vee \id", leftarrow]{r} & K \vee K \arrow["\mu", leftarrow]{d} & K \arrow["\id \vee \chi"', leftarrow]{l} \arrow[leftarrow, "\eta"]{d} \\
* \arrow[leftarrow, "0"]{r} & K & * \arrow[leftarrow, "0"]{l} .
\end{tikzcd}
\end{center}
\end{definition}

\begin{corollary}\marginnote{\citep[Proposition 2.21]{Switzer}}
The functor $[H, -]$ is valued in groups. \qed
\end{corollary}

\begin{example}
$S^1$ is an $H$--cogroup.
\todo{Draw pictures of $\mu'$, $\chi'$.}
\end{example}

\begin{example}\marginnote{\citep[2.22]{Switzer}}
In fact, $S^1 \sm X =: \Susp X$ is an $H$--cogroup for \emph{any} $X$.
\todo{Draw pictures of $\mu'$, $\chi'$.}
\end{example}

\begin{lemma}\marginnote{\citep[Proposition 2.23]{Switzer}}
The adjunction $[\Susp X, Y] \cong [X, \Loops Y]$ is an isomorphism of groups.
\end{lemma}
\begin{proof}[Proof sketch]
This is a matter of writing out the formulas for \[\Susp X \xrightarrow{\mu'} \Susp X \vee \Susp X \xrightarrow{f' \vee g'} Y \vee Y \xrightarrow{\Delta'} Y\] and \[X \xrightarrow\Delta X \times X \xrightarrow{f \times g} \Loops Y \times \Loops Y \xrightarrow\mu Y. \qedhere\]
\end{proof}




\section{Higher homotopy groups}

Just as we used the exponential adjunction to give a few equivalent definitions of the fundamental group, higher homotopy groups have a similar bunch of definitions.

\begin{lemma}\marginnote{\citep[Lemma 2.27]{Switzer}}
For all $n \ge 0$, there is a homeomorphism $S^1 \sm S^n \cong S^{n+1}$.
\end{lemma}
\begin{proof}
\todo[inline]{Carve $S^1$ into two halves, the smash against each of which gives a kind of hemisphere: ``linearly interpolate between $S^n$ + the basepoint of $S^n$ and project to the lower hemisphere''.}
\end{proof}

\begin{definition}\marginnote{\citep[3.1]{Switzer}}
The \define{$n$\textsuperscript{th} homotopy group} of a pointed space $X$ is defined by $\pi_n X = [S^n, X]$.
Equivalently, one may use any of \[\pi_n X = [\Susp^n(S^0), X] = [\Susp^{n-1} S^0, \Loops X] = \cdots = [S^0, \Loops^n X].\]
\end{definition}

We've quietly asserted that $\pi_n X$ is a group, but \emph{which} group structure we mean is not immediately clear: there appear to be many different multiplications on the homotopy mapping sets, each coming from any one of the applications of $\Susp$ or of $\Loops$.
Today we tame this complexity by showing that each of these choices gives the same multiplication.

\begin{lemma}[Eckmann--Hilton]\label{EckmannHilton}%
\marginnote{\citep[Proposition 2.24]{Switzer}}
Let $S$ be a set with two products $\circ$ and $*$ which share a unit $e$ and which satisfy \[(x * x') \circ (y * y') = (x \circ y) * (x' \circ y').\]  Then $\circ = *$ and both are associative and commutative.
\end{lemma}
\begin{proof}
We cleverly redistribute:
\begin{align*}
x \circ y & = (x * e) \circ (e * y) = (x \circ e) * (e \circ y) = x * y, \\
x \circ y & = (e * x) \circ (y * e) = (e \circ y) * (x \circ e) = y * x. \qedhere
\end{align*}
\end{proof}

\begin{corollary}\marginnote{\citep[Proposition 2.25]{Switzer}}
Let $K$ be an $H$--cogroup and $L$ an $H$--group.
The two multiplications so-induced on the mapping set $[K, L]$ are equal and commutative.
\end{corollary}
\begin{proof}
We want the following diagram to commute:
\begin{center}
\begin{tikzcd}
K \times K \arrow{r} & (K \vee K) \times (K \vee K) \arrow["{(f \vee f') \times (g \vee g')}"]{r} & (L \vee L) \times (L \vee L) \arrow{r} & L \times L \arrow{d} \\
K \arrow{u} \arrow{d} & \phantom{\begin{array}{c}(K \times *) \times (K \times *) \\ \cup \\ (* \times K) \times (* \times K) \end{array}} & \phantom{\begin{array}{c}(L \times *) \times (L \times *) \\ \cup \\ (* \times L) \times (* \times L) \end{array}} & L \\
K \vee K \arrow{r} & (K \times K) \vee (K \times K) \arrow["{(f \times g) \vee (f' \times g')}"]{r} & (L \times L) \vee (L \times L) \arrow{r} & L \vee L \arrow{u}.
\end{tikzcd}
\end{center}
The top composite corresponds to $(f +_K f') +_L (g +_K g')$, and the bottom composite corresponds to $(f +_L g) +_K (f' +_L g')$.
Hence, if these were to agree, we could apply \Cref{EckmannHilton}.

The diagonal map $\Delta\co K \to K \times K$ replicates its input onto both coordinates, as in $x \mapsto (x, x)$.
Writing $\mu'(k) = (k_1, k_2)$, we can apply $\Delta$ either before or after the comultiplication map $\mu'$ on $K$, and get the same answer up to a twist:
\begin{align*}
(\Delta \vee \Delta) \circ \mu'(k) & = (\Delta \times \Delta)(k_1, k_2) = (k_1, k_1, k_2, k_2), \\
(\mu' \times \mu') \circ \Delta(k) & = (\mu' \times \mu')(k, k) = (k_1, k_2, k_1, k_2).
\end{align*}
Additionally, because $\mu'\co K \to K \vee K$ targets the wedge sum, it is always the case that at least one of $k_1$ or $k_2$ equals the basepoint.
From these two considerations, it follows that the left-hand portion of the following diagram commutes:
\begin{center}
\begin{tikzcd}
K \times K \arrow{r} & (K \vee K) \times (K \vee K) \arrow{r} & (L \vee L) \times (L \vee L) \arrow{r} & L \times L \arrow{d} \\
K \arrow{u} \arrow{d} \arrow{r} & \begin{array}{c}(K \times *) \times (K \times *) \\ \cup \\ (* \times K) \times (* \times K) \end{array} \arrow{u} \arrow["\id \times T \times \id"]{d} & \phantom{\begin{array}{c}(L \times *) \times (L \times *) \\ \cup \\ (* \times L) \times (* \times L) \end{array}} & L \\
K \vee K \arrow{r} & (K \times K) \vee (K \times K) \arrow{r} & (L \times L) \vee (L \times L) \arrow{r} & L \vee L \arrow{u}.
\end{tikzcd}
\end{center}
Now, we apply $(f, f', g, g')$ on top and $(f, g, f', g')$ on bottom.
Again, the diagram commutes up to transposition of the two middle coordinates.
\begin{center}
\begin{tikzcd}
K \times K \arrow{r} & (K \vee K) \times (K \vee K) \arrow{r} & (L \vee L) \times (L \vee L) \arrow{r} & L \times L \arrow{d} \\
K \arrow{u} \arrow{d} \arrow{r} & \begin{array}{c}(K \times *) \times (K \times *) \\ \cup \\ (* \times K) \times (* \times K) \end{array} \arrow{u} \arrow["\id \times T \times \id"]{d} \arrow{r}  & \begin{array}{c}(L \times *) \times (L \times *) \\ \cup \\ (* \times L) \times (* \times L) \end{array} \arrow{u} \arrow["\id \times T \times \id"]{d} & L \\
K \vee K \arrow{r} & (K \times K) \vee (K \times K) \arrow{r} & (L \times L) \vee (L \times L) \arrow{r} & L \vee L \arrow{u}.
\end{tikzcd}
\end{center}
Finally, we make use of the specific subspace we'd highlighted in the middle.
In (either half of) this subspace, two of the coordinates are constrained to lie at the basepoint.
Because of this, we can apply the unit axioms for multiplication and for fold to conclude that they produce the same value.
This amounts to the commutativity of the final two rectangles:
\begin{center}
\begin{tikzcd}
K \times K \arrow{r} & (K \vee K) \times (K \vee K) \arrow{r} & (L \vee L) \times (L \vee L) \arrow{r} & L \times L \arrow{d} \\
K \arrow{u} \arrow{d} \arrow{r} & \begin{array}{c}(K \times *) \times (K \times *) \\ \cup \\ (* \times K) \times (* \times K) \end{array} \arrow{u} \arrow["\id \times T \times \id"]{d} \arrow{r}  & \begin{array}{c}(L \times *) \times (L \times *) \\ \cup \\ (* \times L) \times (* \times L) \end{array} \arrow{u} \arrow["\id \times T \times \id"]{d} \arrow{u} \arrow{r} & L \\
K \vee K \arrow{r} & (K \times K) \vee (K \times K) \arrow{r} & (L \times L) \vee (L \times L) \arrow{r} & L \vee L \arrow{u}.
\end{tikzcd}
\end{center}
Since each of the squares commutes, the two outer composites are equal.
\end{proof}

\begin{corollary}\marginnote{\citep[Proposition 2.26]{Switzer}}
For $n \ge 2$, $\pi_n X = [\Susp^{n-1} S^0, \Loops X]$ has only one multiplication and it is commutative. \qed
\end{corollary}




\section{Exact sequences in $\CatOf{Spaces}$}

An \emph{extremely} common device in algebraic topology is the exact sequence:

\begin{definition}\marginnote{\citep[2.29]{Switzer}}
A pair of maps of groups \[N \xrightarrow f G \xrightarrow g H\] is called \define{exact} when $\im f = \ker g$.
More weakly, a sequence of pointed sets $X \xrightarrow f Y \xrightarrow g Z$ is \define{exact} if $\im f = g^{-1}(*)$.
\end{definition}

\noindent
Last time, we put structure onto a space so that $[K, -]$ or $[-, L]$ became valued in groups.
Today we are after something similar: we would like to study when continuous maps $A \to B \to C$ induce exact sequences of the functors they co/represent.

\begin{definition}\marginnote{\citep[{Definitions 2.30 and 2.49}]{Switzer}}
If the sequence of pointed sets \[[-, A] \to [-, B] \to [-, C]\] is exact, then we say that the underlying sequence of spaces is \define{exact}.
If the sequence of pointed sets \[[A, -] \from [B, -] \from [C, -]\] is exact, then we say that the underlying sequence of spaces is \define{coexact}.%
\marginnote{These more commonly go by the names \define{fiber} and \define{cofiber} sequences of spaces.  We will work later to justify these alternative names.}
\end{definition}

At first brush, one might imagine that such sequences are somewhat rare, or that they at least require some nice properties of $A$, $B$, or $f$.%
\marginnote{This scarcity is true for maps of groups: $f\co N \to G$ can only participate in an exact sequence when the image of $f$ is a normal subgroup.
It is a good exercise to check that these constructions for spaces do not violate this.}
In fact, these sequences are extremely plentiful in homotopy theory.

\begin{lemma}\label{CoexactSeqsExist}%
\marginnote{\citep[Proposition 2.35]{Switzer}}
Any map $f\co X \to Y$ extends to a coexact sequence \[X \xrightarrow f Y \xrightarrow g Z.\]
\end{lemma}
\begin{proof}
We first define the \define{cone} of $X$ by $CX = X \sm I$, and we use this to set \[Z = Y \cup_f CX = \frac{Y \sqcup CX}{f(x) \sim (x, 1)}.\]
\todo{Define ``mapping cone''.}
We claim that the inclusion $g\co Y \to Z$ induces the desired exact sequence on mapping sets.
To check this, consider a test space $T$, the induced sequence \[[Y \cup_f CX, T] \to [Y, T] \to [X, T],\] and a function $\phi\co Y \to T$.
If $\phi \circ f\co X \to T$ is null, then a choice of null-homotopy defines a map $CX \to T$ agreeing with $f$ on the edge of $I$.
The gluing operation then determines a map $Z \to T$ which restricts to $\phi$.
Conversely, a map $\widetilde \phi\co Z \to T$ which restricts to $\phi$ can itself be restricted to $CX$.
This gives the required null-homotopy of $\phi \circ f$.
\end{proof}

The remarkable lack of hypotheses in \Cref{CoexactSeqsExist} mean that coexact sequences can be extended indefinitely to the right: \[X \xrightarrow f Y \xrightarrow g Y \cup_f CX \to (Y \cup_f CX) \cup_g CY \to \cdots.\]
The presentations of these spaces given by \Cref{CoexactSeqsExist} quickly become unwieldy, but the following alternative presentation remains quite tractible:

\begin{lemma}\label{CollapseDoubleCone}\marginnote{\citep[Lemma 2.37]{Switzer}}
There is a homotopy equivalence \[(Y \cup_f CX) \cup_g CY \to ((Y \cup_f CX) \cup_g CY) / CY. \qed\]
\end{lemma}

\begin{lemma}\label{ConeAndQuotIsRedundant}\marginnote{\citep[Proposition 2.38]{Switzer}}
For $A \subseteq X$ a subspace, there is a homeomorphism \[(X \cup_i CA) / CA \cong X / A. \qed\]
\end{lemma}

\begin{corollary}\marginnote{\citep[Lemma 2.40]{Switzer}}
The infinite coexact sequence takes the form \[X \xrightarrow f Y \to Y \cup_f CX \to \Susp X \xrightarrow{-f} \Susp Y \to \Susp(Y \cup_f CX) \to \cdots.\]
\end{corollary}
\begin{proof}
We need only identify the next two terms after $Z$ and the map between them.
Once that is in hand, we need only note that suspension\todo{is the word ``suspension'' defined?} commutes with the cone and quotient operations used to define $Z$.

To see the claim for the first term after $Z$, we apply the two Lemmas in turn:
\begin{align*}
(Y \cup_f CX) \cup_g CY & \simeq ((Y \cup_f CX) \cup_g CY) / CY \tag{\Cref{CollapseDoubleCone}} \\
& \cong (Y \cup_f CX) / Y \tag{\Cref{ConeAndQuotIsRedundant}} \\
& \cong CX/X \cong \Susp X
\end{align*}
\todo{Is $CX / X \cong \Susp X$ obvious?}
Identically, the second term after $Z$ is described by \[((Y \cup_f CX) \cup_g CY) \cup_{g'} C(Y \cup_f CX) \simeq \Susp Y.\]
The coexact sequence thus takes the desired form.
\end{proof}

Coupling this to our results from last time, this gives altogether an exact sequence
\begin{center}
\begin{tikzcd}
{[X, T]} & {[Y, T]} \arrow{l} & {[\overset{Z}{\overbrace{Y \cup_f CX}}, T]} \arrow{l} & & \text{ptd.\ sets} \\
{[\Susp X, T]} \arrow[out=90, in=270]{rru} & {[\Susp Y, T]} \arrow{l} & {[\Susp Z, T]} \arrow{l} & & \text{groups} \\
{[\Susp^2 X, T]} \arrow[out=90, in=270]{rru} & {[\Susp^2 Y, T]} \arrow{l} & {[\Susp^2 Z, T]} \arrow{l} & \cdots \arrow{l} & \text{ab.\ groups}
\end{tikzcd}
\end{center}
\marginnote{
Exact sequences are best behaved on abelian groups, but not all of the above are abelian groups---or even groups!
What can be said about the edges, where at least one term is a(n abelian) group?
\begin{construction}
Pinching the middle of the cone $CX$ gives a map \[Y \cup_f CX \to (Y \cup_f CX) \vee \Susp X,\] which gives an action \[[Z, T] \times [\Susp X, T] \to [Z, T].\]
\end{construction}
\begin{lemma}[2.42--48]
The map $[Z, T] \to [Y, T]$ is invariant under this action, and on orbits it is an injection.
\end{lemma}
\begin{proof}[Proof sketch]
We indicate the construction underlying injectivity.
Supposing that $f_1, f_2\co Z \to T$ restrict to the same map on $Y$, so that they in particular agree on $X$ but might differ on $CX$.
Glue them together to get a map $d(f_1, f_2)\co \Susp X = CX \cup_X CX \to T$.
One can show that $f_1 = d(f_1, f_2) \cdot f_2$.
\end{proof}
}

There are also dual results for exact sequences: any map of spaces participates in an (infinite) exact sequence.

\begin{lemma}\marginnote{\citep[Proposition 2.54]{Switzer}}
Any map $f\co X \to Y$ extends to an exact sequence \[P \to X \xrightarrow f Y,\] where $P$ is given by \[P = \{(x, \gamma) \in X \times PY \mid f(x) = \gamma(1)\}. \qed\]
\end{lemma}
\begin{proof}[Proof sketch]
Again, the construction of $P$ rests on building into it the data of a null-homotopy.
Suppose that $\theta\co T \to Y$ is a null-homotopic map.
A null-homotopy of $\theta$ is the same as a map $T \sm I = CT \to Y$ restricting to $\theta$, which is in turn the same as a map $T \to PY = Y^I$ restricting to $\theta$.
To model a map $\phi\co T \to X$ for which $\theta = f \circ \phi$ becomes null-homotopic, we attach $PY$ along $X$ as in the statement.
\end{proof}

\begin{lemma}\label{LexseqInvolvingP}%
\marginnote{\citep[Proposition 2.58]{Switzer}}
Iterating this gives \[\cdots \to \Loops^2 X \to \Loops P_f \to \Loops X \to \Loops Y \to P_f \to X \xrightarrow f Y. \qed\]
\end{lemma}

In particular, by applying $\pi_0(-) = [S^0, -]$ and employing the definition $\pi_n X = \pi_0 \Loops^n X$, we get an exact sequence of homotopy groups: \[\cdots \to \pi_2 X \to \pi_2 Y \to \pi_1 P \to \pi_1 X \to \pi_1 Y \to \pi_0 P \to \pi_0 X \to \pi_0 Y.\]
One of our goals in this class will be to understand what understanding $\pi_*$ earns us, how it changes as $X$ changes, and how $X$ can be effectively dissected to build up knowledge of $\pi_*$ along the way.
This long exact sequence will be one of our main tools for doing so.




\section{Relative homotopy groups}

The construction $P$ from the previous lecture is a little mysterious.
To close out this Chapter, we will explore it in two lights: today and a bit later on.

Let us return to the setting of pairs $(X, A, x_0)$ of topological spaces, and let's use $i\co A \to X$ to denote the inclusion of the preferred subspace.

\begin{lemma}\marginnote{\citep[Definition 3.8]{Switzer}}
The exact continuation $P$ of $i$ is given by the function object \[P(i) = (X, A, x_0)^{(I, \partial I, 0)}. \qed\]
\end{lemma}

Inspired by this, we consider the adjunction juggle:

\begin{definition}\marginnote{\citep[pg.\ 38]{Switzer}}
We define the \define{$n$\textsuperscript{th} relative homotopy group} of the pair $(X, A)$ by \[\pi_n(X, A) := [(D^n, S^{n-1}), (X, A)] \cong [S^{n-1}, P] = \pi_{n-1} P.\]
That is, $\pi_n(X, A)$ consists of $n$--disk maps into $X$ with boundary $\partial D^n$ lying in $A$.
\end{definition}

\begin{corollary}\label{LexseqOfRelativeHtpy}%
\marginnote{\citep[Proposition 3.9]{Switzer}}
There is a long exact sequence \[\cdots \to \pi_{n+1}(X, A) \to \pi_n A \to \pi_n X \to \pi_n (X, A) \to \pi_{n-1} A \to \cdots. \qed\]
\end{corollary}

% \begin{remark}
% $\pi_n(X, x_0) = \pi_n(X, \{x_0\})$.  So, we can regard the map $\pi_n X \to \pi_n (X, A)$ as induced by $(X, \{x_0\}) \to (X, A)$.
% \end{remark}

These groups track the discrepancy between $\pi_* A$ and $\pi_* X$.
This observation moves us to remark on when there is \emph{no} discrepancy:

\begin{definition}\label{ConnectedDefn}%
\marginnote{\citep[3.12--13 and 3.17]{Switzer}}
A pair $A \subseteq X$ is \define{$n$--connected} if $\pi_{\le n}(X, A) = 0$.
Equivalently, the map $\pi_* A \to \pi_* X$ is an isomorphism for $* < n$ and an epimorphism for $* = n$.
The pair $A \subseteq X$ is a \define{weak equivalence} if it is $\infty$--connected.
\end{definition}

\begin{remark}\marginnote{\citep[3.15]{Switzer}}
We can extend these definitions to a generic $f\co Y \to X$ using the \define{mapping cylinder}: \[M_f := (Y \times I) \cup_f X.\]
This space receives a map $X \to M_f$ which is a weak equivalence.
\marginnote{It is a good exercise to check that it is a weak equivalence.}
It also receives two maps $Y \to M_f$, one along the ``free'' end of the cylinder $Y \times I$ and one along the ``attached'' end.
Along the free end, the map from $Y$ to $M_f$ is an inclusion, so that $(M_f, Y)$ can be thought of as a pair.
Along the attached end, the map from $Y$ to $M_f$ factors through $X$, where it is shown to agree with $f$.
The situation is summarized in the following diagram:
\begin{center}
\begin{tikzcd}
Y \times \{1\} \arrow{d} \arrow[hook]{rd} & M_f \arrow[equal]{d} \\
Y \times I \arrow{r} & (Y \times I) \cup_f X & X \arrow["\simeq"']{l} . \\
Y \times \{0\} \arrow{u} \arrow{ru} \arrow{rru}
\end{tikzcd}
\end{center}
\end{remark}

\newthought{We pause to emphasize} something that the reader might have dismissed as serendipity: by switching notation from $(A \subseteq X)$ to $\pi_n(X, A)$, \Cref{LexseqOfRelativeHtpy} looks very much like a corresponding theorem about relative \emph{homology} groups.
It is extremely productive to see how far this analogy can be pushed: what theorems about homology can be replicated for homotopy?---and, when a theorem for homology fails for homotopy, is it partially recoverable?

As a warm-up to this program, recall that the relative homology groups of a pair of inclusions $x_0 \in B \subseteq A \subseteq X$ can be interrelated.
We will consider whether the same can be said of relative homotopy groups.
To this end, consider the long exact sequences associated to the pairs $(X, A)$, $(X, B)$, and $(A, B)$, which arrange into the following commutative diagram:
\begin{figure*}
\begin{center}
\begin{tikzcd}[column sep=0.3em]
\cdots & \pi_{n+1}(X, A) \arrow{rr} \arrow[blue]{rd} & & \pi_n(A, B) \arrow{rd} \arrow[orange]{rr} & & \pi_{n-1} B \arrow[orange]{rd} \arrow[purple]{rr} & & \pi_{n-1} X & \cdots \\
\cdots & & \pi_n A \arrow[orange]{ru} \arrow[blue]{rd} & & \pi_n(X, B) \arrow[purple]{ru} \arrow{rd} & & \pi_{n-1} A \arrow[blue]{ru} \arrow[orange]{rd} & & \cdots \\
\cdots & \pi_n B \arrow[orange]{ru} \arrow[purple]{rr} & & \pi_n X \arrow[purple]{ru} \arrow[blue]{rr} & & \pi_n(X, A) \arrow[blue]{ru} \arrow{rr} & & \pi_{n-1}(A, B) & \cdots
\end{tikzcd}
\end{center}
\end{figure*}
To help with readiability, we have colored the long exact sequences associated to \textcolor{blue}{$(X, A)$}, \textcolor{orange}{$(A, B)$}, and \textcolor{purple}{$(X, B)$}.
To complete the symmetric pattern, we have additionally included maps $\pi_n(A, B) \to \pi_n(X, B)$ and $\pi_n(X, B) \to \pi_n(X, A)$ granted to us by naturality, as well as maps $\pi_n(X, A) \to \pi_{n-1}(A, B)$ granted by completing the relevant triangle.
Tracing through the middle, we see that these new maps form an interesting sequence, highlighted in red:
\begin{figure*}
\begin{center}
\begin{tikzcd}[column sep=0.3em]
\cdots & \textcolor{red}{\pi_{n+1}(X, A)} \arrow[red]{rr} \arrow{rd} & & \textcolor{red}{\pi_n(A, B)} \arrow[red]{rd} \arrow{rr} & & \pi_{n-1} B \arrow{rd} \arrow{rr} & & \pi_{n-1} X & \cdots \\
\cdots & & \pi_n A \arrow{ru} \arrow{rd} & & \textcolor{red}{\pi_n(X, B)} \arrow{ru} \arrow[red]{rd} & & \pi_{n-1} A \arrow{ru} \arrow{rd} & & \cdots \\
\cdots & \pi_n B \arrow{ru} \arrow{rr} & & \pi_n X \arrow{ru} \arrow{rr} & & \textcolor{red}{\pi_n(X, A)} \arrow{ru} \arrow[red]{rr} & & \textcolor{red}{\pi_{n-1}(A, B)} & \cdots
\end{tikzcd}
\end{center}
\end{figure*}

\begin{lemma}\marginnote{\citep[Theorem 3.20]{Switzer}}
This sequence is exact.
\end{lemma}
\begin{proof}[Proof sketch]
Checking that the composites are zero is easy enough.
We describe how the lifting condition is proved for the left-most map pictured.
Suppose that we have an element $x \in \pi_n(A, B)$ mapping to zero in $\pi_n(X, B)$.
Its image in $\pi_{n-1} B$ is then also zero, so that we may lift it to $y \in \pi_n A$.
Define $z \in \pi_n X$ to be the image of $y$; since $y$ is zero by the time it makes it to $\pi_n(X, B)$, so must $z$ be.
We may then also lift $z$ to $w \in \pi_n B$.
Pushing forward $w$ to $y' \in \pi_n A$ gives a second element in the same group, and the difference of the two elements $y - y'$ pushes forward to zero in $\pi_n X$.
Hence, we can lift the difference to $\pi_{n+1}(X, A)$.
This ultimately gives the desired lift of $x$.
\todo{Overlay this diagram chase as a series of diagrams.}
\end{proof}

\begin{remark}\marginnote{\citep[3.21]{Switzer}}
As a simple structural application, consider an inclusion $i\co A \subseteq X$ which admits a retraction $r\co X \to A$.\sidenote{That is: $r i \simeq \id_A$.}
The induced map $\pi_* A \to \pi_* X$ is then an inclusion, and the boundary map $\pi_*(X, A) \to \pi_{*-1}(A)$ is zero.
It follows that there are short exact sequences
\begin{center}
\begin{tikzcd}
0 \arrow{r} & \pi_n A \arrow{r} & \pi_n X \arrow{r} \arrow[bend right]{l} & \pi_n(X, A) \arrow{r} & 0.
\end{tikzcd}
\end{center}
For $n \ge 3$, all the groups are abelian, from which we deduce that $\pi_n X$ splits as a sum of $\pi_n A$ and $\pi_n(X, A)$.
At $n = 2$, we learn naively that $\pi_2 X$ is a semidirect product of $\pi_2 A$ and $\pi_2(X, A)$---but this forces $\pi_2(X, A)$ to be abelian has well.
Hence, in this situation we have
\[\pi_{\ge 2} X = \pi_{\ge 2} A \oplus \pi_{\ge 2}(X, A).\]
\end{remark}




\section{Fibrations}

\todo{We promised to explain the name ``co/fiber''. Now is the time to do it.}

Relative to the small and polite model for the continuation of the coexact sequence, the function space $P$ feels large and unwieldly.
In this section we explore a more familiar context in which spaces \emph{like} $P$ arise, as well as conditions that they are weak models for exact continuations.

To get off the ground, recall the whole point of the design of $P$ was its participation in the long exact sequence%
\marginnote{\Cref{LexseqInvolvingP}}
\[\cdots \to \pi_2 X \to \pi_1 P \to \pi_1 X \to \pi_1 Y \to \pi_0 P \to \pi_0 X \to \pi_0 Y.\]
Inasmuch as the exact sequence is truly what we're after, we might consider alternative constructions in which such sequences arise.

\begin{example}\marginnote{\citep[Theorem 4.1]{Switzer}}
For each $n \ge 0$, there is an isomorphism \[\pi_n(X \times Y) = [S^n, X \times Y] \cong = [S^n, X] \times [S^n, Y] \cong \pi_n(X) \times \pi_n(Y).\]
Rearranged as an exact sequence, the maps
\begin{center}
\begin{tikzcd}
Y \arrow["i_Y"]{r} & X \times Y \arrow["\pi_X"]{r} \arrow[bend right, "\pi_Y"']{l} & X \arrow[bend right, "i_X"']{l}
\end{tikzcd}
\end{center}
induce a split-exact sequence
\begin{center}
\begin{tikzcd}
\cdots \arrow["0"]{r} & \pi_n Y \arrow{r} & \pi_n(X \times Y) \arrow[bend right]{l} \arrow{r} & \pi_n X \arrow[bend right]{l} \arrow["0"]{r} & \cdots.
\end{tikzcd}
\end{center}
\end{example}

\noindent
We would like to axiomatize the part of the geometry of the Cartesian product that we need to induce exact sequences on $\pi_*$.
Its main features that we used were the projection and retraction maps, which combine to let us recover a map $S^n \to X \times Y$ from the projections $S^n \to X$ and $S^n \to Y$.
We will ultimately ask for ``one half'' of this data (i.e., one projection and one inclusion), together with a requirement that they satisfy a variant of this recovery property.

\begin{definition}\marginnote{\citep[Definition 4.2]{Switzer}}
A map $p\co E \to B$
\marginnote{``$E$'' for Espace (fr.), ``$B$'' for Base.}
has the \define{homotopy lifting property} with respect to a space $X$ when for all solid diagrams
\begin{center}
\begin{tikzcd}
X \arrow["f"]{r} \arrow["- \times 0"]{d} & E \arrow["p"]{d} \\
X \times I \arrow["H"]{r} \arrow[densely dotted, "{\widetilde H}"]{ru} & B
\end{tikzcd}
\end{center}
there exists a dashed diagonal lift $\widetilde H$.
That is, homotopies in $B$ lift to homotopies in $E$.
A \define{fibration} has the HLP for all spaces.
A \define{weak fibration} has the HLP for at least the disks $D^n$.
The \define{fiber} of a fibration is $F = p^{-1}(b_0) \subseteq E$.
\end{definition}

\begin{example}
The projection $\pi_X\co X \times Y \to X$ forms a fibration with fiber $Y$.
\end{example}

\begin{example}\marginnote{\citep[Proposition 4.3]{Switzer}}
The evaluation map $\mathit{ev}\co PX \to X$ form a fibration with fiber $\Loops X$.
\end{example}

For the remainder of the lecture, we will work to justify that this definition yields the intended long exact sequence of homotopy groups.
The first step is a version of the homotopy lifting property for pairs and its relation to the original definition above.

\begin{lemma}\label{HLPsForPairsVsPtds}%
\marginnote{\citep[Proposition 4.5]{Switzer}}
Consider a fibration $p\co E \to B$, as well as a preferred subset $B' \subseteq B$ of the base and the subset $E' := p^{-1}(B') \subseteq E$ of the total space lying over it.
If $p$ has the HLP for $X \times I$, then $p'\co (E, E')^{(I, \partial I)} \to (B, B')^{(I, \partial I)}$ has the HLP for $X$.
\end{lemma}
\begin{proof}
Let us begin with an HLP diagram for $p'$, as in
\begin{center}
\begin{tikzcd}
X \arrow["f"]{r} \arrow{d} & (E, E')^{(I, \partial I)} \arrow["p'"]{d} \\
X \times I \arrow["H"]{r} & (B, B')^{(I, \partial I)}
\end{tikzcd}.
\end{center}
We may remove the function spaces by applying the exponentiation adjunction:
\begin{center}
\begin{tikzcd}
X \times (I \vee I) \arrow["f''"]{r} \arrow["i"]{d} & E \arrow["p"]{d} \\
X \times I \times I \arrow["H'"]{r} & B.
\end{tikzcd}
\end{center}
The homeomorphism $h\co I \vee I \cong I$ extends to a homeomorphism $h'$ as in
\begin{center}
\begin{tikzcd}
I \vee I \arrow[hook]{d} \arrow["{h, \cong}"]{r} & I \arrow[hook]{d} \\
I \times I \arrow["{h', \cong}"]{r} & I \times I,
\end{tikzcd}
\end{center}
which can be used to smooth out the top-left corner:
\begin{center}
\begin{tikzcd}
X \times I \arrow["{f'' \circ h^{-1}}"]{r} \arrow["{- \times 0}"]{d} & E \arrow["p"]{d} \\
X \times I \times I \arrow["H' \circ h^{-1}"]{r} \arrow[densely dotted, "{\widetilde H'}"]{ru} & B.
\end{tikzcd}
\end{center}
Applying the HLP for $p$ gives $\widetilde{H'}$.  Reversing the application of $h'$ and the exponential adjunction ultimately yields the desired $\widetilde{H}$ in the Lemma statement.
\end{proof}

\begin{corollary}\marginnote{\citep[Theorem 4.6]{Switzer}}
If $p$ as above is a weak fibration, then the natural map \[\pi_n(E, E') \to \pi_n(B, B')\] is an isomorphism for all $n \ge 1$.
\end{corollary}
\begin{proof}
Begin with a class $\omega \co (I, \partial I) \to (B, B')$ in $\pi_1(B, B')$, and consider the diagram
\begin{center}
\begin{tikzcd}
* \arrow["e_0"]{r} \arrow{d} & E \arrow["p"]{d} \\
* \times I \arrow["\omega"]{r} \arrow[densely dotted, "\widetilde{\omega}"]{ru} & B.
\end{tikzcd}
\end{center}
Since $\omega(1) \in B'$ and $E' := p^{-1}(B')$, the class $\widetilde \omega$ can be considered as a map $\widetilde\omega\co (I, \partial I) \to (E, E')$.
Hence, $\pi_1(E, E') \to \pi_1(B, B')$ a surjection.

For injectivity, suppose that $\omega_1, \omega_2\co (I, \partial I) \to E$ are two relative homotopy classes and that $H\co I \times I \to B$ is a homotopy connecting them in $B$.
We form the homotopy lifting diagram
\begin{center}
\begin{tikzcd}
U \arrow{r} \arrow{d} & E \arrow{d} \\
I \times I \arrow["H"]{r} \arrow["\widetilde{H}", densely dotted]{ru} & B,
\end{tikzcd}
\end{center}
where $U = \partial I \setminus (I \times \{1\})$ is a $\sqcup$--shaped figure, the top arrow acts by $\omega_1$ and $\omega_2$ on the two legs, and it carries the bottom of the figure to the basepoint.
The same procedure as in \Cref{HLPsForPairsVsPtds} produces a filler $\widetilde H$, which witnesses equality in $\pi_1(E, E')$.
Injectivity follows.

For $\pi_{> 1}$, one uses relative pathspaces to induct up.
\todo{This is cryptic.}
\end{proof}

\begin{corollary}\marginnote{\citep[4.7]{Switzer}}
There is a long exact sequence \[\cdots \to \pi_n F \to \pi_n E \to \pi_n B \to \pi_{n-1} F \to \cdots.\]
\end{corollary}
\begin{proof}
Set $B' = \{b_0\}$ in the above, and identify these terms respectively with \[\cdots \to \pi_n F \to \pi_n E \to \pi_n(E, F) \to \pi_{n-1} F \to \cdots. \qedhere\]
\end{proof}

\begin{corollary}\label{LoopsShiftsPi}\marginnote{\citep[Corollary 4.8]{Switzer}}
If $E$ is contractible (e.g., as in the fibration $\mathit{ev}\co PX \to X$), then the going-around map $\pi_{n+1} B \to \pi_n F$ is an isomorphism. \qed
\end{corollary}




\section{Fiber bundles and examples}

\todo{Add non-example: the S-shaped graph ``fibering'' over the line.}
Fibrations are all over classical geometry.
Most commonly, they take the form of a \define{fiber bundle}, which is a less homotopically-minded take on the important properties of the Cartesian product.

\begin{definition}\label{FiberBundleDefn}%
\marginnote{\citep[Definition 4.9]{Switzer}}
A \define{fiber bundle} is a pair of maps $F \subseteq E \xrightarrow p B$ such that $B$ has an open cover $\{U_\alpha\}_\alpha$ admitting local homeomorphisms $\phi_\alpha\co U_\alpha \times F \to p^{-1}(U_\alpha)$ which satisfy $p \phi_\alpha = p|_{U_\alpha}$.
\end{definition}

\begin{lemma}\marginnote{\citep[Proposition 4.10]{Switzer}}
Every fiber bundle is a weak fibration.
\marginnote{If $B$ is paracompact, then $p$ is actually a fibration.}
\end{lemma}
\begin{proof}[Proof sketch]
Fix an open cover as in \Cref{FiberBundleDefn}.  Given a map $f\co D^n \times I \to B$, subdivide $D^n \times I$ so finely that its pieces are each contained in one member of the cover.  Once this is arranged, one can build the desired lift purely locally.
\end{proof}

The most common source of these ``local projections'' come from Lie theory: a quotient of a Lie group by a subgroup has everywhere-isomorphic fibers, and in good cases the coset space can be arranged to admit one of these covers.
We spend most of today making this precise.

\begin{lemma}\marginnote{\citep[Theorem 4.13]{Switzer}}
Take $H \le G$ to be a closed subgroup of a topological group.
\marginnote{We do not assume normality!}
Suppose that the identity coset $H \in G/H$ has an open neighborhood $U$ with a section $U \xrightarrow s G \xrightarrow p G/H$.
The map $p$ is then a fiber bundle with fiber $H$.
\end{lemma}
\begin{proof}
Left-multiplying by $g \in G$, the condition at $H$ begets sections near all $gH \in G/H$.
The sections together become the data of a fiber bundle by \[U_{gH} \times H \xrightarrow{s_g \times 1} G \times G \xrightarrow\mu G. \qedhere\]
\end{proof}

\begin{example}[Stiefel manifolds]\marginnote{\citep[Examples 4.14.1]{Switzer}}
Consider the subgroup $O(n) \subseteq O(n+k)$ of block matrices \[\left\{\left[\begin{array}{c|c} O(n) & 0 \\ \hline 0 & I \end{array} \right]\right\} \subseteq O(n+k).\]
The quotient is the space of orthonormal $k$--frames in $\R^{n+k}$.
\marginnote{For example, $O(n) / O(n-1)$ is homeomorphic to $S^{n-1}$.}
To construct a local section, note there is an open neighborhood $U$ of the standard frame $(e_{n+1}, \ldots, e_{n+k})$ consisting of those $(u_{n+1}, \ldots, u_{n+k})$ satisfying the condition \[\det\left[e_1 \mid \cdots \mid e_n \mid u_{n+1} \mid \cdots \mid u_{n+k}\right] \ne 0.\]
On $U$, the local section is defined by applying Gram--Schmidt to the block matrix of columns $[e_1 \mid \cdots \mid e_n \mid u_{n+1} \mid \cdots \mid u_{n+k}]$ to produce an element of $O(n+k)$.
\end{example}

\begin{example}[Grassmannians]\label{GrassmanniansAreFiberBundles}%
\marginnote{\citep[Examples 4.14.2]{Switzer}}
The further quotient by \[\left\{ \left[ \begin{array}{c|c} O(n) & 0 \\ \hline 0 & O(k) \end{array} \right] \right\} \le O(n+k)\] gives the space of $k$--dimensional subspaces in $\R^{n+k}$.
Consider the open neighborhood $U$ of $(e_{n+1}, \ldots, e_{n+k})$ consisting of those subspaces $W$ which trivially intersect the subspace $\<e_1, \ldots, e_n\>$.
By projecting $e_{n+1}, \ldots, e_{n+k}$ into $W$ and applying Gram--Schmidt, this produces an orthonormal $k$--frame, i.e., a section landing in $O(n+k) / O(n)$.
This can ultimately be used to construct a fiber bundle \[O(n) \to \frac{O(n+k)}{O(k)} \to \frac{O(n+k)}{O(n) \times O(k)},\] even though the middle term is not a group.
\end{example}

\begin{example}\marginnote{\citep[Examples 4.14.4, 5, 7]{Switzer}}
These linear algebraic examples do not rest on properties of the reals.
One can build analogues of these examples with $U(n)$ and $\C^n$, or with $\Sp(n)$ and $\H^n$.
\end{example}

\begin{example}\label{SOtoOisAFiberBundle}%
\marginnote{\citep[Examples 4.14.3]{Switzer}}
The sequence \[SO(n) \to O(n) \xrightarrow{\det} O(1)\] admits a section \[\pm 1 \mapsto \left( \begin{array}{c|c} \pm 1 & 0 \\ 0 & I \end{array} \right),\] which witnesses it as a fiber bundle.
\end{example}

\begin{example}\marginnote{\citep[Proposition 4.15]{Switzer}}
Applying the long exact sequence of homotopy for a fiber bundle to \Cref{SOtoOisAFiberBundle} shows \[\pi_{\ge 1} SO(m) \cong \pi_{\ge 1} O(n).\]
\end{example}

This isomorphism of homotopy groups is attractive enough that we give fiber bundles with discrete fibers a special name:

\begin{definition}\marginnote{\citep[Definition 4.16]{Switzer}}
A \define{covering} of $B$ is a fiber bundle with discrete fiber.
\end{definition}

\begin{example}\marginnote{\citep[Examples 4.18.1]{Switzer}}
The quotient sequence \[\Z \to \R \to S^1\] witnesses $\R$ as a cover of $S^1$ with discrete fiber $\Z$.
The local section in this case is given by \[z \mapsto \frac{1}{2 \pi i} \log z.\]
\end{example}

A discrete fiber has sufficiently simple structure that one can often automatically construct fiber bundles as quotients by group actions, without needing to manufacture a local section to guarantee sane behavior.
As our final task for today, we record this easier situation.

\begin{definition}\marginnote{\citep[4.19]{Switzer}}
A discrete group $G$ acts \define{properly discontinuously} on $X$ when\ldots
\begin{enumerate}
    \item \ldots for all points $x \in X$ there is a neighborhood $U_x$ so that $g U_x$ never intersects $U_x$ for non-identity $g$.\label{PropDiscon1}
    \item \ldots for all points $x, y \in X$ in different orbits, there are neighborhoods $U_x$, $U_y$ so that $gU_x$ never meets $U_y$.\label{PropDiscon2}
\end{enumerate}
\end{definition}

\begin{lemma}\marginnote{\citep[Proposition 4.20]{Switzer}}
If $G$ acts properly discontinuously on $X$, then $G \to X \xrightarrow p X/G$ is a covering (and $X/G$ is Hausdorff).
\end{lemma}
\begin{proof}
For $[x] \in X/G$, choose a neighborhood $U_x$ guaranteed by \eqref{PropDiscon1}, and let $p(U_x)$ be the neighborhood of $[x]$ in the base.  We using the unicity in \eqref{PropDiscon1} to define the local section \[p^{-1}(p(U_x)) \to G \times p(U_x). \qedhere\]
\end{proof}

\begin{remark}\marginnote{\citep[Remarks 4.21.i]{Switzer}}
If $\pi_0 X = 0$, then in the short exact sequence of pointed sets \[0 \to \pi_1 X \to \pi_1 X/G \to \pi_0 G \to 0\] the last map is actually a group homomorphism.
\end{remark}

\begin{example}\label{Pi1S1Calculation}\marginnote{\citep[Remarks 4.21.iv]{Switzer}}
$\Z$ acting on $\R$ by $1 \cdot x = x+1$ is properly discontinuous.
Since $\R$ is contractible, we may conclude \[\pi_n S^1 = \begin{cases} \Z & \text{when $n = 1$}, \\ 0 & \text{otherwise}. \end{cases}\]
\end{example}

\begin{remark}\marginnote{\citep[Remarks 4.21.ii]{Switzer}}
A covering $p\co E \to B$ is \define{regular} if $p_*(\pi_1 E) \subseteq \pi_1 B$ is normal.
All regular covers arise as quotients by a $G$--action.
\end{remark}

\begin{remark}\marginnote{\citep[Remarks 4.21.iii]{Switzer}}
For $G$ finite, $X$ Hausdorff, and no fixed points, the $G$--action is properly discontinuous.
\end{remark}





\begin{subappendices}

\section{The action of $\pi_1$}

\todo{I remember not being convinced of the value of this lecture.}
Before continuing our study of $P_f$, there is one other ``old'' fact about homotopy groups we should investigate: their dependence on $x_0 \in X$.  You might even remember that paths $\gamma$ in $X$ induce isomorphisms $\pi_1(X; \gamma(0)) \to \pi_1(X; \gamma(1))$.  This fits into a framework we have considered already.  We begin with the algebraic thing we are trying to model.

\begin{definition}
For $G$, $A$ groups, an action $\alpha\co G \times A \to A$ is \define{compatible} when $g(a_1 a_2) = (g a_1)(g a_2)$, i.e., the following commutes
\begin{center}
\begin{tikzcd}
G \times A \times A \arrow["\Delta \times \id \times \id"]{r} \arrow["\id \times \mu"]{d} & G \times G \times A \times A \arrow["\simeq"]{r} & G \times A \times G \times A \arrow["\alpha \times \alpha"]{r} & A \times A \arrow["\mu"]{d} \\
G \times A \arrow["\alpha"]{rrr} & & & A.
\end{tikzcd}
\end{center}
\end{definition}

\begin{example}
$G$ acts compatibly on itself by conjugation.
\end{example}

\begin{example}
For $A$ abelian (i.e., a $\Z$--module), this is equivalent to a $\Z[G]$--module structure.
\end{example}

The theorem we want to prove is that $\pi_1 X$ acts compatibly on $\pi_n X$, $n \ge 1$.

\begin{definition}
An $H$--cogroup $K$ acts \define{compatibly} on an $H$--cogroup $L$ when the diagram dual to the one above commutes.
\end{definition}

\begin{corollary}
For such $H$--cogroups, the action of $[K, T]$ on $[L, T]$ is compatible. \qed
\end{corollary}

\begin{lemma}
The following defines a compatible coaction of $S^1$ on $S^n$: \todo[inline]{Picture!}
\end{lemma}
\begin{proof}
\todo[inline]{Picture!}
\marginnote{The proof is a little more obnoxious when $n = 1$.}
\end{proof}

\begin{remark}
There is also a relative version of this story: $S^1$ has a compatible coaction on $(CS^n, S^n)$, hence $\pi_1 A$ acts compatibly on $\pi_n(X, A)$ for $n \ge 2$.
\end{remark}

\begin{lemma}
The action satisfies various naturalities:
\begin{enumerate}
    \item $(X, A) \to (Y, B)$ induces
    \begin{center}
    \begin{tikzcd}
    \pi_1 X \times \pi_n X \arrow{r} \arrow{d} & \pi_n X \arrow{d} \\
    \pi_1 Y \times \pi_1 Y \arrow{r} & \pi_n Y, 
    \end{tikzcd}
    \begin{tikzcd}
    \pi_1 A \times \pi_n(X, A) \arrow{r} \arrow{d} & \pi_n(X, A) \arrow{d} \\
    \pi_1 B \times \pi_n(Y, B) \arrow{r} & \pi_n(Y, B).
    \end{tikzcd}
    \end{center}
    \item The following diagrams commute:
    \begin{center}
    \begin{tikzcd}
    \pi_1 A \times \pi_n A \arrow{r} \arrow{d} & \pi_n A \arrow{d} \\
    \pi_1 X \times \pi_n X \arrow{r} & \pi_n X,
    \end{tikzcd}
    \begin{tikzcd}
    \pi_1 A \times \pi_n(X, A) \arrow{r} \arrow{d} & \pi_n(X, A) \arrow{d} \\
    \pi_1 A \times \pi_{n-1} A \arrow{r} & \pi_{n-1} A.
    \end{tikzcd}
    \end{center}
    \item The action of $\pi_1 X$ on itself is by conjugation: $\gamma \cdot \alpha = \gamma \alpha \gamma^{-1}$. \qed
\end{enumerate}
\end{lemma}

\begin{remark}
A weak equivalence $f\co Y \to X$ thus induces an isomorphism of $\Z[\pi_1 X]$--modules, in addition to just abelian groups, a strictly stronger condition.  There is much more structure; this is the start of the study of $\Pi$--algebras.
\end{remark}

\begin{remark}
There is a messy version of this that lets us encode the change of basepoint maps from the intro\todo{See Switzer pg.\ 47--49}.  The main conclusion is that $\pi_n(X, x_0) \cong \pi_n(X, x'_0)$ \emph{unnaturally} if $\pi_0 X = *$ and \emph{naturally} if $\pi_1 X = 1$.
\end{remark}

\begin{remark}
Take discrete $G$ to act properly discontinuously on $X$.  If $\pi_1 X = 0$, then $\pi_1 X/G \cong \pi_0 G$, hence we get a $G$--action on $\pi_{\ge 2}(X/G)$.  In fact, $\pi_n X \to \pi_n(X/G)$ respects this action.
\end{remark}

\end{subappendices}
