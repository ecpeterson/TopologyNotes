% -*- root: main.tex -*-

\section{Where to go from here (Oh, the places you'll go!)}

You now know enough algebraic topology to be dangerous.  (Anyone who hears ``spectrum'' and doesn't flinch is dangerous.)  However, we're sitll a ways from the forefront of the field, and I wanted to point out some current pedagogical landmarks, c.\ 2017 (and tailored to my preferences, of course).

A lot of what we learned in this class feeds directly into the \textit{vector fields on spheres} problem, a major accomplishment of 1970's topology, and among the first conceptual problems solved by broad-reaching computational invention.  This is totally a ``next step'' for this class.

Stable homotopy houses a lot of geometry through \textit{bordism homology}, where formal sums of singular simplices are replaced by maps in from manifolds with boundary.  Even the bordism homology of a point is of great interest---it houses a kind of homotopical intersection theory for manifolds.

\textit{Algebraic $K$--theory} is a generic tool that captures a lot of geometric information about any kind of context: algebraic geometry, manifolds, groups, \ldots. It's kind of bottomless, and being able to compute anything about it is often met with cheers and wild career success.

\textit{Homological stability} refers to the phenomenon that ``things'', like symmetry groups, often occur in families, like $\{\Sigma_m\}_{m=1}^\infty$, and that these families have highly compatible homologies.  This comes in many shapes and sizes and generally has geometric content.

\textit{Equivariant homotopy theory} is of geometric interest because symmetric groups appear everywhere in geometric contexts, and requiring homotopy theory to be mindful of it makes homotopy theory itself considerably more geometry.

Homotopy theory has surprising applications in algebraic geomtery.  Artin--Mazur gives access to a homotopy type associated to a site on a scheme, and e.g.\ the \define{\'etale homotopy type} carries a remarkable amount of information about the scheme.  Meanwhile, \define{motivic homotopy theory} is a modern blend of homotopical techniques with algebro-geometric ones, with large implications for both sides: Voevodsky used this to settle the Milnor/Bloch--Kato conjectures, and Isaksen has used this to push $\pi_* \S$ computations considerably further.

\textit{Goodwillie calculus} describes a natural sequence of stable invariants to almost any homotopical functor, and it turns out that these invariants generalize / unify a bunch of unrelated classical invariants from practically every corner of homotopy theory.

\textit{Spectral algebra(ic geometry)} is a re-encoding of classical algebra into spectra.  We talked about ``ring spectra'' in this class, but it turns out that asking for, e.g., associativity requires real heavy machinery.  The rewards are great: the theory of ``modules'' associated to such a spectrum is very rich and eases a lot of arguments, and the extra operations impressed on such a spectrum encode useful (and often classically relevant) data.  (One can then try ``doing algebraic geometry'' in this setting, which appears to be a hole without bottom.)

There is an older program in homotopy theory, called \textit{chromatic homotopy theory}, which continues to give the tightest results on the ``global'' behavior of homotopy theory be comparing it to particular (highly complex) algebraic models.  You can get a lot of intuition very quickly by learning some of this.  Its genesis was in understanding certain periodic phenomena in the Adams spectral sequence---hence is often taught from a highly computational point of view, if that's your thing.