% -*- root: main.tex -*-

\chapter{Manipulating Homotopy Types}




\section{CW complexes (5)}

A \define{CW-structure} on a space $X$ is a presentation of $X$ by inductively ``attaching $n$--cells''.

\begin{definition}
Given a space $Y$ and a continuous map $g\co \bigvee_\alpha S^{n-1}_\alpha \to Y$, we say that $Y \cup_g \bigvee_\alpha CS^{n-1}$ is \define{formed from $Y$ by attaching $n$--cells}.
\end{definition}

\begin{definition}
Let $X^{-\infty} = \cdots = X^{-1} = \{x_0\}$, and let $X^n$ be formed from $X^{n-1}$ by attaching $n$--cells.  The union $X$ of the $X^n$ (with the weak topology) is called a \define{CW-structure} on $X$.  (Or, if $A$ is a CW-complex, then setting $X^{-\infty} = \cdots = X^{-1} = A$ gives rise to a relative CW-structure $(X, A)$.)
\end{definition}

\begin{example}
\begin{itemize}
    \item $S^n$ with 
    \todo[inline]{Finish copying these pictures.}
\end{itemize}
\end{example}

The inductive presentation means we can attack problems cell-by-cell. For instance:
\begin{corollary}
A function $f\co X \to Y$ is continuous if and only if $\bigvee_n \bigvee_{\alpha \in A_n} CS^{n-1}_\alpha \to X \to Y$ is continuous. \qed
\end{corollary}

\begin{lemma}
If $X$ is CW and $Y$ is \emph{finite} CW, then $X \times Y$ is CW using the homeomorphism $D^n \times D^m \cong D^{n+m}$. \qed
\end{lemma}

\begin{lemma}
If $X$ is CW and $A \subseteq X$ is a subcomplex, then $X/A$ is CW. \qed
\end{lemma}

\begin{corollary}
Homotopies can also be constructed inductively. \qed
\end{corollary}

\begin{remark}
If $X$ and $Y$ are infinite, then the weak topology on $X \times Y$ may not agree with the product of the weak topologies.  We mostly concede this point and pick the one we want (usually the former).
\end{remark}

\begin{example}
$S^n \to Y$ agrees with maps $D^n \to Y$ sending $\partial D^n$ to $y_0$.
\end{example}

\begin{example}
Given $S^{n-1} \to Y$ and two choices of null-homotopies
\begin{center}
\begin{tikzcd}
S^{n-1} \arrow{d} \arrow{r} & Y \\
CS^{n-1} \arrow[shift left]{ru} \arrow[shift right]{ru},
\end{tikzcd}
\end{center}
we can form a difference class $S^n \to Y$.
\end{example}

\begin{example}
Given $\omega\co S^1 \to Y$ and a null-homotopy
\begin{center}
\begin{tikzcd}
S^1 \arrow["2\omega"]{r} \arrow{d} & Y \\
D^2 \arrow["H"]{ru},
\end{tikzcd}
\end{center}
we can form $\RP^2 \to Y$.
\end{example}

Some topological facts that might interest you:
\begin{itemize}
    \item Every CW complex is Hausdorff.
    \item Every CW complex is the disjoint union of the interiors of its cells.
    \item Each cell has only finitely many immediate faces.
    \item More generally, any compact subset has this property.
\end{itemize}

\begin{lemma}
Let $X = A \cup_g e^n$, $(K, L)$ a \emph{finite} simplicial pair, and $(|K|, |L|) \xrightarrow{f} (X, A)$ some map.  There exists a subdivision $(K', L')$ of $(K, L)$ and a map $f'\co (|K'|, |L'|) \to (X, A)$ such that
\begin{enumerate}
    \item $f|_{f^{-1}(A)} = f'|_{f^{-1}(A)}$ and $f \simeq_{rel f^{-1}(A)} f'$, and
    \item for $\sigma \in K'$, if $f'(|\sigma|)$ means $\overset \circ e^n$ then $f'(|\sigma|) \subseteq \overset \circ e^n$ \emph{and} $f'|_{|\sigma|}$ is a linear map. \qed
\end{enumerate}
\end{lemma}





\section{The homotopy theory of CW complexes I (6.3--6.28)}

An important feature of CW complexes is that their homotopy type depends only on their assembly data up to homotopy.

\begin{lemma}
Given two maps $\omega_1, \omega_2\co S^{n-1} \to X$, a homotopy $H\co \omega_1 \sim \omega_2$ begets a homotopy equivalence $X \cup_{\omega_1} CS^{n-1} \xrightarrow{\simeq} X \cup_{\omega_2} CS^{n-1}$.
\end{lemma}
\begin{proof}
Subdivide $CS^{n-1}$ into an outer annulus ($[1/2, 1]$ in the cone coordinate) and an inner disk.  Extend the identity map on $X$ by running the homotopy $H$ on the annulus, then using the homeomorphism $S^{n-1} \sm [0, 1/2] \cong CS^{n-1}$ to cover the $n$--cells in the target.
\end{proof}

In fact, CW complexes are very homotopically well-behaved.  Our goal is to get \emph{familiar} (not prove!) with some of these facts today and to compute $\pi_{* \le n} S^n$.

\begin{lemma}
For $(X, A)$ a relative CW-complex, $(X, (X, A)^n)$ is $n$-connected. \qed
\end{lemma}

\begin{corollary}
The inclusion $X^n \to X$ is $n$-connected. \qed
\end{corollary}

\begin{corollary}
$\pi_{< n} S^n = 0$.
\end{corollary}
\begin{proof}
There's a cell structure on $S^n$ with $(S^n)^{n-1} = \{s_0\}$, and we have a long exact sequence of relative homotopy groups that shows $\pi_{* < n} S^n = \pi_{* < n} (S^n)^{n-1} = \pi_{* < n} \{s_0\} = 0$.
\end{proof}

\begin{lemma}
If $(X, A)$ is $n$--connected, then there exists an equivalence $(X, A) \sim (X', A')$ with $(X', A')^n = A'$.  (Try setting $A = *$.)\footnote{This is a kind of converse to the previous Lemma.} \qed
\end{lemma}

\begin{corollary}
For $X$ $n$--connected and $Y$ $m$--connected, $X \sm Y$ is $(n+m+1)$--connected.
\end{corollary}
\begin{proof}
The cells in $X \times Y$ take the form $* \times *$, $* \times e_\beta^j$, $e_\alpha^i \times *$, and $e_\alpha^i \times e_\beta^j$.  All the former are in $X \vee Y$, so the first nontrivial cell in $X \sm Y$ lies in dimension $n + m + 2$.\todo{This cell structure argument also feeds into the claim that the suspension of a CW complex has predictable cell structure and predictable attaching maps.}
\end{proof}

\begin{theorem}[Homotopy excision]
Take $A, B \subseteq X$ with $(A, A \cap B)$ $n$--connected and $(B, A \cap B)$ $m$--connected.  Then $\pi_*(A, A \cap B) \to \pi_*(X, B)$ is an isomorphism for $* < n+m$ and an epimorphism for $* = n+m$. \qed
\end{theorem}

\begin{corollary}
If $(X, A)$ is $n$--connected and $A$ is $m$--connected, then $\pi_*(X, A) \to \pi_* X/A$ is an isomorphism for $1 < * \le n+m$ and onto at $n+m+1$. \qed

\textbf{There's a diagram written in the margin:}
\begin{tikzcd}
(X, A) \arrow{r} \arrow{d} & (X \cup_A CA, CA) \arrow["\simeq"]{d} \\
(X/A, *) \arrow["\simeq"]{r} & (X \cup_A CA/CA, *)
\end{tikzcd}
\end{corollary}

\begin{corollary}[Freudenthal suspension theorem]
For $n$--connected $X$,
\begin{center}
\begin{tikzcd}
\pi_{*+1}(CX, X) \arrow{r} \arrow[equal]{d} & \pi_{*+1}(CX/X) \arrow[equal]{d} \\
\pi_* X \arrow{r} & \pi_{*+1} \Susp X
\end{tikzcd}
\end{center}
is an isomorphism for $* \le 2n$ and an epimorphism for $* = 2n+1$.
\end{corollary}

\begin{example}
There's a fibration
\begin{center}
\begin{tikzcd}
\C^\times \arrow{r} \arrow["\simeq"]{d} & \C^n \setminus 0 \arrow{r} \arrow["\simeq"]{d} & \CP^{n-1} \arrow["\simeq"]{d} \\
S^1 \arrow{r} & S^{2n-1} \arrow{r} & \CP^{n-1}.
\end{tikzcd}
\end{center}
Since $S^{2n-1}$ is $(2n-2)$--connected, $\pi_{*+1} \CP^{n-1} \cong \pi_* S^1$ for $* \le 2(n-1)$.  However, $(\CP^{n-1}, \CP^1)$ is $2$--connected, so $\pi_2 \CP^1 \cong \pi_2 \CP^{n-1} \cong \pi_1 S^1 \cong \Z$.  This feeds into Freudenthal: $\pi_n S^n \cong \pi_{n+1} S^{n+1}$ for $n \ge 2$, so $\pi_n S^n \cong \Z$ for all $n \ge 1$.
\end{example}

\begin{remark}
That exact sequence is a fibration using Grassmannians:
\begin{center}
\begin{tikzcd}
U(1) \arrow{r} \arrow["\simeq"]{d} & \frac{U(n)}{U(n-1)} \arrow{r} \arrow["\simeq"]{d} & \frac{U(n)}{U(n-1) \times U(n)} \arrow["\simeq"]{d} \\
S^1 \arrow{r} & S^{2n-1} \arrow{r} & \CP^{n-1}.
\end{tikzcd}
\end{center}
\end{remark}




\section{The homotopy theory of CW complexes II (6.29--)}

The following technical lemma appeared in our study of the relative homotopy long exact sequence of a pair $(Y, B)$:
\begin{lemma}[3.14]
For all
\begin{center}
\begin{tikzcd}
B \arrow{r} & Y \\
S^{n-1} \arrow["\omega|_{S^{n-1}}"]{u} \arrow{r} & D^n \arrow["\omega"]{u}
\end{tikzcd}
\end{center}
there exist fillers
\begin{center}
\begin{tikzcd}
B \arrow{r} & Y \\
S^{n-1} \arrow{u} \arrow{r} & D^n \arrow{u} \arrow["\omega'"]{lu}
\end{tikzcd}
\end{center}
\todo{Add a $2$--cell from $Y$ to filler arrow.}
\end{lemma}

This can be augmented in two ways:

\begin{lemma}
If $f\co Z \to Y$ is an $n$--equivalence and $\dim(X, A) \le n$, then for all
\begin{center}
\begin{tikzcd}
Z \arrow["f"]{r} & Y \\
A \arrow["g"]{u} \arrow{r} & X \arrow["h"]{u}
\end{tikzcd}
\end{center}
there exist fillers
\begin{center}
\begin{tikzcd}
Z \arrow{r} & Y \\
A \arrow{u} \arrow{r} & X \arrow{u} \arrow{lu}
\end{tikzcd}
\end{center}
\todo{Add a $2$--cell from $Y$ to filler arrow.}
\end{lemma}

\begin{corollary}
For $f\co Z \to Y$ an $n$--equivalence, for all $X$ with $\dim X \le n$, the map $[X, Z] \to [X, Y]$ is onto.  If $\dim X < n$, it's an isomorphism. \qed
\end{corollary}

\begin{corollary}[Whitehead]
A weak equivalence $f\co Z \to Y$ of CW-complexes is a homotopy equivalence. \qed
\end{corollary}

Filtering a CW-complex by its skeleta and applying the Lemma yields another useful result:
\begin{corollary}
All maps $f\co (X, A) \to (Y, B)$ of CW-complexes are homotopic (relative to $A$) to cellular maps (i.e., $f(X, A)^n \subseteq (Y, B)^n$), and homotopies between cellular maps admit cellular replacements. \qed
\end{corollary}

Now recall an observation from last time: for $X$ $n$--connected and $Y$ $m$--connected, we have $(X \times Y, X \vee Y)^{n+m+1} = (X \vee Y)^{n+m+1}$.

\begin{corollary}
For $* \le n+m$, $\pi_*(X \vee Y) \xrightarrow\cong \pi_*(X \times Y)$. \qed
\end{corollary}

\begin{corollary}
For $n \ge 2$, $\pi_n(\bigvee_\alpha S^n_\alpha) \cong \bigoplus_\alpha \pi_n S^n_\alpha$.  (Also, $\pi_1 \bigvee_\alpha S^1_\alpha \cong *_\alpha \pi_1 S^1_\alpha$.)\todo{Borrow bigast from book project} \qed
\end{corollary}

\begin{lemma}
For any abelian group $A$ and index $n \ge 2$, there exists a CW-complex $K(A, n)$ with $\pi_* K(A, n) = \begin{cases} A & \text{if $* = n$}, \\ 0 & \text{otherwise}. \end{cases}$
\end{lemma}
\begin{proof}
Select a presentation $\Z^J \xrightarrow g \Z^I \to A$.  We model the middle node as $\bigvee_I S^n$, and the Corollary gives a map $\bigvee_J S^n \xrightarrow{\widetilde g} \bigvee_I S^n$ induces $g$ on $\pi_n$.  The cone on $g$ gives a complex $X_n$ with $\pi_{* < n} X_n = 0$ and $\pi_n X_n = A$.  We inductively form $X_{n+j+1}$ from $X_{n+j}$ by killing the homotopy in degree $n+j+1$ by taking more mapping cones.
\end{proof}

\begin{lemma}
For $X$ with $\pi_{* < n} = 0$ and $Y$ with $\pi_{* > n} = 0$, homotopy classes $[X, Y]$ biject with homomorphisms $\pi_n X \to \pi_n Y$.
\end{lemma}
\begin{proof}
There exists a CW model of $X$ with $X^{n-1} = *$, so we have
\begin{center}
\begin{tikzcd}
\bigvee_I S^n \arrow[equal]{r} & X^n \arrow{r} & X^{n+1} \arrow{r} & X^{n+2} \arrow{r} & \cdots \arrow{r} & X \\
& \bigvee_J S^n \arrow{u} & \bigvee S^{n+1} \arrow{u} & \cdots \arrow{u}.
\end{tikzcd}
\end{center}
Maps into $Y$ can be constructed inductively: we begin with $\bigvee_I S^n \to Y$, then we need the precomposite to vanish $\bigvee_J S^n \to \bigvee_I S^n \to Y$ to guarantee an extension, the unicity of which is measured by $[\Susp \bigvee_J S^n, Y] = 0$.  This only gets easier as we go up the skeletal tower.
\end{proof}

\begin{corollary}
$K(A, n)$ is independent of choice of presentation. \qed
\end{corollary}

\begin{remark}
$\Loops K(A, n) \cong K(A, n-1)$ and $\Susp K(A, n-1) \to K(A, n)$ is an interesting map.
\end{remark}

\begin{remark}
$\pi_n(X^n, X^{n-1})$ is free on generators $\gamma \cdot [f^n_\alpha]$, $\gamma \in \pi_1 X^{n-1}$, and $f^n_\alpha$ a characteristic map of an $n$--cell.
\end{remark}

\begin{remark}
For \emph{all} spaces $X$, there exists a CW-complex $\widetilde X \to X$ such that the map is a weak equivalence.
\end{remark}




\section{Spectral sequences}
\todo[inline]{The Serre classes section was super thin: basically a list of unproven properties. It would be nice to prove something about arithmetic localizations instead.}

These arise \emph{everywhere} in algebraic topology, and they are as fundamental as the notion of a long exact sequence.  Long exact sequences appear when calculating $H_*$ of $A \xrightarrow j X \to X \cup_j CA$.  What if there are many coexact sequences constructing $X$?

\begin{center}
\begin{tikzcd}
* \arrow{r} & X_1 \arrow{r} & X_2 \arrow{r} & \cdots \arrow{r} & X_n \arrow{r} & X_{n+1} \arrow{r} & \cdots \arrow{r} & X \\
& A_1 \arrow{u} & A_2 \arrow{u} & \cdots & A_n \arrow{u} & A_{n+1} \arrow{u} & \cdots
\end{tikzcd}
\end{center}

If we apply $H_*$ to this diagram, we get a family of long exact sequences:
\begin{center}
\begin{tikzcd}
0 \arrow{r} & H_* X_1 \arrow{r} & H_* X_2 \arrow{r} \arrow[red]{dl} & \cdots \arrow{r} \arrow[red]{dl} & H_* X_n \arrow{r} \arrow[red]{dl} & H_* X_{n+1} \arrow{r} \arrow[red]{dl} & \cdots \arrow{r} \arrow[red]{dl} & X \\
& H_* A_1 \arrow{u} & H_* A_2 \arrow{u} & \cdots & H_* A_n \arrow{u} & H_* A_{n+1} \arrow{u} & \cdots,
\end{tikzcd}
\end{center}
where each red arrow shifts homological degree by one, e.g., $H_* X_{n+1} \to H_{*-1} A_n$.  Suppose we wanted to recover $H_* X$ by inspecting the bottom row of $H_* A_*$, similarly to how in the case of a long exact sequence one expects to recover $H_* X$ from $H_* A$ and $H_*(X \cup_j CA)$.  Start by noticing $H_* X = \colim_n H_* X_n$, so that $x \in H_* X$ arises as $\widetilde x \in H_* X_n$ for some $n$.  Note also that at exactly the \emph{minimal} such $n$, it pushes down to give a nonzero class $a_{n-1} \in H_* A_{n-1}$.

What about a generic class $a_n \in H_* A_n$?  Can we tell when it pushes down from a class that belongs to $H_* X$?  If it doesn't, what then is its role?

\begin{center}
\begin{tikzcd}
& & \textcolor{red}{x_n} \arrow[red, |->]{lddd} & \textcolor{red}{(\exists \; x_{n+1}???)} \arrow[red, |->]{lddd} \\
\cdots \arrow{r} & H_* X_{n-1} \arrow{r} & H_* X_n \arrow{r} \arrow{dl} & H_* X_{n+1} \arrow{r} \arrow{dl} & \cdots \\
& H_* A_{n-1} \arrow{u} & H_* A_n \arrow{u} & H_* A_{n+1} \arrow{u} \\
& \textcolor{red}{d_1(a_n)} & \textcolor{red}{a_n} \arrow[red, |->, bend left]{uuu}
\end{tikzcd}
\end{center}

\noindent Note that if $d_1(a_n) \ne 0$, then $x_n \ne 0$, hence $a_n \not\in \ker(H_n A_n \to H_* X_n)$, and $\not\exists x_{n+1}$.  On the other hand, if $d_1(a_n) = 0$, then we can form a preimage $x_{n_1}$ of $x_n$, as in

\begin{center}
\begin{tikzcd}
& & \textcolor{red}{x_{n-1}} \arrow[red,|->]{r} \arrow[red,|->]{lddd} & \textcolor{red}{x_n} \arrow[red, |->]{lddd} & \textcolor{red}{(\exists \; x_{n+1}???)} \arrow[red, |->]{lddd} \\
\cdots \arrow{r} & H_* X_{n-2} \arrow{r} & H_* X_{n-1} \arrow{r} & H_* X_n \arrow{r} \arrow{dl} & H_* X_{n+1} \arrow{r} \arrow{dl} & \cdots \\
& H_* A_{n-2} \arrow{u} & H_* A_{n-1} \arrow{u} & H_* A_n \arrow{u} & H_* A_{n+1} \arrow{u} \\
& \textcolor{red}{d_2(a_n)} & \textcolor{red}{0} & \textcolor{red}{a_n} \arrow[red, |->, bend left]{uuu}
\end{tikzcd}
\end{center}

We make a series of claims:
\begin{enumerate}
    \item This assignment $d_2$ is well-defined up to $\im d_1$, and hence it determines a function $d_2\co H_*(H_* A_*; d_1) \to H_{*-1}(H_* A_{*-2}; d_1)$.
    \item This process continues ad infinitum.  Since $X_0 = *$, eventually $x_0 = 0$ is guaranteed but then $x_j = 0$ for all $j \le n$, and hence there exists $x_{n+1}$.
    \item The surviving elements in a spectral sequence are the associated graded of a filtration of $H_* X$ (by minimal lift degree).
\end{enumerate}

\begin{remark}
This story is complicated some by using a bi-infinite filtration or by using a cohomology functor / a descending filtration, primarily because such spectral sequences need not \emph{stabilize}.  You have to incorporate taking inverse limits of the subquotients of $H_* A_*$, which can destroy some of the exactness in the third Claim or the argument used in the second Claim.
\end{remark}

\begin{lemma}
If a map of spectral sequences is ever an isomorphism, it is an isomorphism forever after.  Hence, their targets are isomorphic by the same map. \qed
\end{lemma}

\begin{example}
Filter $X$ by skeleta, so $A_n = \bigvee_\alpha S^n_\alpha$ and $H_* A_n = (\Susp^n G)^{\bigoplus_\alpha}$.  The map $d_1$ is exactly the cellular differential:
\begin{center}
\begin{tikzcd}
H_* X_{n_1} \arrow{r} & H_* X_n \arrow{ld} \\
H_* \bigvee_\beta S^{n-1}_\beta \arrow{u} & H_* \bigvee_\alpha S^n_\alpha \arrow{u} \arrow["d_1"]{l},
\end{tikzcd}
\end{center}
so that $H_*(H_* A_*; d_1) = H_*^{\mathrm{cell}}(X)$.  All higher differentials are zero because $\bigoplus_\alpha \Susp^n G \xrightarrow{[-1]} \bigoplus_\gamma \Susp^{n-r} G$ has the wrong degree.  We say that the spectral sequence \define{collapses at $E_2$}, and this is a proof of that cellular homology computes homology.
\end{example}

\begin{corollary}
More genrally, if $E$ and $F$ satisfy Eilenberg--Steenrod and $E_*(S^n) \to F_*(S^n)$ is an isomorphism for all $n$, then $E_*(X) \to F_*(X)$ is an isomorphism for all CW complexes $X$. \qed
\end{corollary}




\section{Obstruction theory}

Obstruction theory is generally concerned with trying to extend a diagram like
\begin{center}
\begin{tikzcd}
A \arrow{r} \arrow{d} \arrow[densely dotted]{rd} & B \\
X \arrow[densely dotted]{r} \arrow[densely dotted]{ru} & Y \arrow{u}
\end{tikzcd}
\end{center}
along any of the indicated dashed maps.  We are going to summarize our results in terms fo compute $\pi_0 Y^X$, leaving the relative cases of $A \to X$ and $Y \to B$ to the reader.

Recall one of our older Lemmas:
\begin{lemma}
If $\pi_{<n} Y = 0$ and $\pi_{>n} Z = 0$, then $[Y, Z] \xrightarrow\cong [\pi_n Y, \pi_n Z]$. \qed
\end{lemma}

\begin{corollary}
If $Y$ is $(n-1)$--connected, there is a canonical map $Y \to K(\pi_n Y, n)$ inducing an isomorphism on $\pi_n$. \qed
\end{corollary}

\begin{corollary}
For $Y$ $(n-1)$--connected, the fiber of $Y \to K(\pi_n Y, n)$ witnesses the $(n+1)$--truncation of $Y$, $Y(n, \infty)$.  It has the properties \[\pi_* Y(n, \infty) = \begin{cases} \pi_* Y & \text{if $* > n$}, \\ 0 & \text{otherwise} \end{cases}\] and $\pi_* Y(n, \infty) \to \pi_* Y$ is an isomorphism for $* > n$. \qed
\end{corollary}

Successive applications of this lead to the \define{Postnikov tower}:
\begin{center}
\begin{tikzcd}
Y \arrow{d} & Y(1, \infty) \arrow{l} \arrow{d} & Y(2, \infty) \arrow{l} \arrow{d} & \cdots \arrow{l} & Y(n, \infty) \arrow{l} \arrow{d} & \cdots \arrow{l} & * \arrow{l} \\
K(\pi_1 Y, 1) & K(\pi_2 Y, 2) & K(\pi_3 Y, 3) & & K(\pi_{n+1} Y, n+1),
\end{tikzcd}
\end{center}
a diagram of interlocking fiber sequences.  This situation is ripe for a spectral sequence.

\begin{remark}
Applying $\pi_*$ to this diagram gives an extremely boring spectral sequence.
\end{remark}

Instead, we apply $\pi_* F(X, -)$, where $X$ is some fixed test space.  The functor $F(X, -)$ preserves fiber sequences and $\pi_*$ turns them into long exact sequences, giving
\begin{definition}
Federer's spectral sequence has signature \[E^1_{m, n} = \pi_m F(X, K(\pi_{n+1} Y, n+1)) = H^{n-m+1}(X; \pi_{n+1} Y) \Rightarrow \pi_m F(X, Y).\]
\end{definition}

\begin{example}
Another interesting case is when $Y = \OS{E}{0}$ is an infinite-loopspace.  This spectral sequence then recovers the Atiyah--Hirzebruch spectral sequence from the previous lecture.
\end{example}

\begin{remark}
For arbitrary $X$ and $Y$, $\pi_0$ and $\pi_1$ of $F(X, Y)$ are \emph{sets} and \emph{groups} respectively.  This situation is called a \define{fringed spectral sequence} and they are considerably more obnoxious.  Set $X = \Susp^2 X'$ or $Y = \Loops^2 Y'$ to avoid this situation.
\end{remark}

\begin{remark}
If $\pi_{< n} X = 0$ and $\pi_{> n} Y = 0$, we have $H^{< n}(X; \text{any}) = 0$ and $H^{\text{any}}(X; \pi_{> n} Y) = 0$.  This puts a single nonvanishing group in the $0$--line of the spectral sequence: $H^n(X; \pi_n Y) \cong \CatOf{AbGps}(H_n X, \pi_n Y) \cong \CatOf{AbGps}(\pi_n X, \pi_n Y)$.
\end{remark}

\begin{remark}
The relative version of this spectral sequence also recovers the lifting lemma about \todo{BACKREFERENCE HERE.}
\end{remark}

What information would we need to understand nontrivial examples of this spectral sequence?  The $d_1$--differential is induced by pushforward along the map $k_n$ in
\begin{center}
\begin{tikzcd}
Y(n-1, \infty) \arrow{dd} & & Y(n, \infty) \arrow{ll} \arrow{dd} \\
& \Loops K(\pi_n Y, n) \arrow["k_n"]{rd} \\
K(\pi_n Y, n) & & K(\pi_{n+1} Y, n+1).
\end{tikzcd}
\end{center}
This map is part of the homotopy data of $Y$, called the \define{$n$\textsuperscript{th} $k$--invariant} of $Y$.  Note $\Loops K(\pi_n Y, n) \simeq K(\pi_n Y, n-1) \to K(\pi_{n+1} Y, n+1)$ is a map of Eilenberg--Mac Lane spaces, so can be considered as the data of a natural transformation $\sigma\co H^{n-1}(-; \pi_n Y) \to H^{n+1}(-; \pi_{n+1} Y)$.

\begin{enumerate}
    \item What are these? How many are there? How can they be discerned for some random space $Y$?
    \item This is kind of the ``dual problem'' to compute $[S^m, S^n]$.
\end{enumerate}




\section{A complicated example}

We haven't yet had an example of a spectral sequence in which we can compute.  This is because it is impossible to find a spectral sequence that is easy, tangible, nontrivial, and well-motivated all at once.  Today we will do the first three.

\todo[inline]{I propose swapping this lecture out for some sort of Bockstein spectral sequence---especially if you can find one that has the ``already been hit'' phenomenon.}