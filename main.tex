\documentclass{amsart}

\usepackage{todonotes}
% \usepackage{spectralsequences}
\usepackage{tikz-cd}

\newcommand{\CatOf}[1]{\mathsf{#1}}
\newcommand{\define}[1]{\textit{#1}}

\newcommand{\Z}{\mathbb Z}
\renewcommand{\S}{\mathbb S}
\newcommand{\CC}{\mathcal{C}}

\newcommand{\co}{\colon\thinspace}
\newcommand{\Loops}{\Omega}
\newcommand{\sm}{\wedge}
\newcommand{\Susp}{\Sigma}

\DeclareMathOperator{\id}{id}
\DeclareMathOperator{\Tor}{Tor}

\theoremstyle{plain}
\newtheorem{dummy}{Dummy}[section]
\newtheorem{theorem}[dummy]{Theorem}
\newtheorem*{theorem*}{Theorem}
\newtheorem*{lemma*}{Lemma}
\newtheorem{proposition}[dummy]{Proposition}
\newtheorem{lemma}[dummy]{Lemma}
\newtheorem{corollary}[dummy]{Corollary}
\newtheorem{conjecture}[dummy]{Conjecture}
\theoremstyle{definition}
\newtheorem{definition}[dummy]{Definition}
\newtheorem{construction}[dummy]{Construction}
\newtheorem{warning}[dummy]{Important Warning}
\theoremstyle{remark}
\newtheorem{remark}[dummy]{Remark}
\newtheorem{example}[dummy]{Example}

\title{Untitled Homotopy Theory Notes}
\author{Eric Peterson}

\begin{document}

\maketitle

This document is an elaboration on a set of lecture notes delivered at Harvard in Spring 2017 (Math 231b), covering various homotopy theory under the assumption of some prior exposure to ordinary co/homology.  The course passes through four stages:

\begin{enumerate}
	\item Decompositions of spaces (e.g., as co/fiber sequences).\label{DecompGoal}
	\item Invariants constructed from such decompositions (e.g., $H_*$, $\pi_*$) and their properties (e.g., Whitehead, Hurewicz, \ldots).\label{InvariantsGoal}
	\item Representability theorems (e.g., Brown, Adams) and the stable category ($H\Z$, $KU$ and $KO$, $\S$, $MG$, mod--$\CC$ theory, \ldots).\label{RepresentabilityGoal}
	\item Computations (characteristic classes, Bott periodicity, the Steenrod operations, the Adams and Serre spectral sequences, \ldots).\label{ComputationsGoal}
\end{enumerate}

Here's an example of the kind of analysis you can perform with these tools at your disposal.  Start with your favorite (simply-connected) space, like $S^{n \ge 2}$.  Its homotopy groups are notoriously difficult and important to compute, because of their role in \eqref{DecompGoal}.  The Hurewicz theorem says $\pi_n(X) \cong H_n(X; \Z)$ when $\pi_{* < n}(X) = 0$, an example of facts from \eqref{InvariantsGoal}.  So, we get one homotopy group for free.  The decompositions studied in \eqref{DecompGoal} give a fiber sequence \[X[n+1, \infty) \to X \to K(\pi_n X, n),\] where $K(\pi_n X, n)$ is a space arising in \eqref{RepresentabilityGoal} with the property \[\pi_* K(\pi_n X, n) = \begin{cases} \pi_n X & \text{if $* = n$}, \\ 0 & \text{otherwise}. \end{cases}\]  Again employing theorems from \eqref{InvariantsGoal}, it follows that \[\pi_* X[n+1, \infty) = \begin{cases} \pi_* X & \text{if $* \ge n + 1$}, \\ 0 & \text{otherwise}. \end{cases}\]  The Serre spectral sequence from \eqref{ComputationsGoal} takes as input $H_* X$ and $H_* K(\pi_n X, n)$ and produces $H_* X[n+1, \infty)$.  One can then apply the Hurewicz theorem to compute $\pi_{n+1} X[n+1, \infty)$, hence $\pi_{n+1} X$, and repeat.




\todo{Improve title.}
\todo[inline]{We have a limit of 35-37 lectures.}
\todo[inline]{One of the stated goals of the course was to introduce exotic cohomology theories, which we did none of. Should we? $K$--theory?}
\todo[inline]{I also gave a bunch of homework exercises that I'd prefer to be solved inline in the notes: for instance, facts about localizations, or the minimal models portion of unstable rational homotopy theory.}
\todo[inline]{My goal when teaching this class was to introduce the ``algebraic'' perspective on the homotopy theory of spaces, with the primary intention of showing that you can both build tools and do computations without recourse to underlying geometry.}
\todo[inline]{Emphasize the prevalence of moduli problems in homotopy theory.}
\todo[inline]{Spectral sequences, early and often.}
\todo[inline]{Emphasize when categorical constructions are used in a ``wrong way'' fashion.}




\newpage
\tableofcontents
\newpage




\section{The categories $\CatOf{Spaces}$ and $\CatOf{Spaces}_*$ (0)}
\todo[inline]{I remember regretting that the course started so ``slowly'' with point-set details. If these can be compressed, please do.}

Today I wanted to remind you of some basic topological facts, so that I can guiltlessly assume them later on.

Three basic constructions:\todo{Mention the word ``sheaf''}
\begin{description}
	\item[Locality] For $X = \bigcup_j A_j$ a decomposition into closed subsets, $\{f\co X \to Y \text{continuous}\}$ bijects with $\{(f_i\co A_i \to Y \text{cts}) \mid f_i|_{A_i \cap A_j} = f_j|_{A_i \cap A_j}\}$.
	\item[Products] For $X$, $Y$ spaces there's a space $X \times Y$ such that $\{f\co T \to X \times Y \text{cts}\}$ bijects with $\{(f_X\co T \to X \text{cts}, f_Y\co T \to Y \text{cts}\}$.\todo{Discuss coproducts too}
	\item[Quotients] For $R$ an equivalence relation on $X$, there is a space $X / R$ such that $\{\overline f\co X / R \to Y \text{cts}\}$ bijects with $\{f\co X \to Y \text{cts} \mid xRx' \Rightarrow f(x) = f(x')\}$.
\end{description}

\begin{example}
For $A \subseteq X$ nonempty, define $X/A$ by extending the total relation on $A$ by the identity relation on $X$.  As an edge case, set $X / \emptyset = X \cup \{*\}$.\todo{Make this less mysterious.}
\end{example}

\begin{lemma}
If $Y$ is locally compact, then \[\frac{X \times Y}{\alpha \times \id} \xrightarrow{\cong} \left( \frac{X}{\alpha}\right) \times Y. \qed\]
\end{lemma}

This Lemma is what allows us to ``fatten'' constructions, by setting $Y = I := [0, 1]$.

It also powers the discussion of \define{relative homotopy}:
\begin{lemma}
If $H\co X \times I \to Y$ factors as $X \times \{t\} \to X/\alpha \times \{t\} \to Y$ for all $t$, then it factors as a whole as $X \times I \to X/\alpha \times I \xrightarrow{\widetilde H} Y$. \qed
\end{lemma}

\begin{lemma}
If $A \subseteq X$ is closed and $H(a, t) = H(a', t)$ for all $a, a' \in A$ and $t \in I$, then it factors as $X \times I \to X/A \times I \to Y$. \qed
\end{lemma}

\begin{description}
	\item[Function spaces/exponential objects] For $X$, $Y$ spaces, $Y^X = \{f\co X \to Y \text{cts}\}$.  If $X$ is locally compact, then $ev\co Y^X \times X \to Y$ is continuous.  If $X$ and $Z$ are additionally Hausdorff, then $Y^{Z \times X} \to (Y^Z)^X$ is a homeomorphism.
\end{description}

We will perpetually arrange to be in this situation, so that we have access to these \define{function spaces}.

We will also often want to track a preferred point in a space: $(X, x_0)$, and restrict attention to maps $f\co (X, x_0) \to (Y, y_0)$ that are continuous functions $f\co X \to Y$ such that $f(x_0) = y_0$.  (This is $\CatOf{Spaces}_{*/}$.)  More generally, we might want preferred subspaces
\begin{center}
\begin{tikzcd}
X \arrow["f"]{r} & Y \\
A \arrow{u} \arrow[densely dotted, "f|_A"]{r} & B \arrow{u}.
\end{tikzcd}
\end{center}

One can re-envision the operations above for such ``relative'' objects.  For instance, $(X, A) \times (Y, B) = (X \times Y, (X \times B) \cup (A \times Y))$, or \[(Y, B)^{(X, A)} = \{f \co X \to Y \text{cts} \mid f(A) \subseteq B\}.\]  These interact as expected: \[(Y, B)^{(Z, C) \times (X, A)} = \left((Y, B)^{(Z, C)}\right)^{(X, A)}.\]  In particular, \[\left((Y, y_0)^{(Z, z_0)}\right)^{(X, x_0)} = (Y, y_0)^{(X \times Z, (X \times z_0) \cup (x_0 \times Z))}.\]  This exponent is the product in relative Spaces; its right-hand part is the coproduct in $\CatOf{Spaces}_{*/}$, and the quotient \[\frac{X \times Z}{X \vee Z} =: X \sm Z\] is a ``monoidal product'' in $\CatOf{Spaces}_{*/}$.




\section{Perspectives on the fundamental group (2--2.23)}

Recall that the \define{pathspace} of $X$ is $X^I$, $I = [0, 1]$.

\begin{definition}
$\pi_0(X)$ is the set of path-components of $X$, i.e., $[x] = [x']$ when there exists $\gamma \in X^I$ such that $\gamma(0) = x$, $\gamma(1) = x'$.
\end{definition}

Writing $[Y, X]$ for the set of homotopy classes of functions $Y \to X$, we also have $\pi_0 X = [*, X]$, or $\pi_0(X, X_0) = [S^0, X]$ for $S^0 = (\{\pm 1\}, 1)$.  In terms of the exponential objects from last time, homotopy classes themselves can be defined as $\pi_0 X^Y$, from which it follows that $[Z \sm X, Y] = [X, Y^Z]$ for pointed spaces.

Let's use this to perturb the definition of a fundamental group:
\begin{align*}
\pi_1(X) & := \{\text{homotopy classes of pointed loop in $X$}\} \\
& = [S^1, X] = [S^0 \sm S^1, X] = [S^0, X^{(S^1)}] = \pi_0 X^{S^1}.
\end{align*}

One might wonder what properties of $Y$ and $X$ make $[Y, X]$ into a group, since we know that $[S^1, -]$ and $[S^0, (-)^{S^1}]$ are group-valued.

\begin{definition}
A \define{group} is a pointed set $G$ with pointed maps $\mu\co G \times G \to G$, $\eta\co * \to G$\todo{Make this pointing symmetric with the definition below}, $\chi\co G \to G$ satisfying
\begin{center}
\begin{tikzcd}
G \times G \times G \arrow["\mu \times \id"]{r} \arrow["\id \times \mu"]{d} & G \times G \arrow["\mu"]{d} \\
G \times G \arrow["\mu"]{r} & G,
\end{tikzcd}
\begin{tikzcd}
G \arrow[equal]{rd} \arrow["\eta \times \id"]{r} & G \times G \arrow["\mu"]{d} & G \arrow["\id \times \eta"']{l} \arrow[equal]{ld} \\
& G,
\end{tikzcd}
\begin{tikzcd}
G \arrow{d} \arrow["\chi \times \id"]{r} & G \times G \arrow["\mu"]{d} & G \arrow["\id \times \chi"']{l} \arrow{d} \\
* \arrow["\eta"]{r} & G & * \arrow["\eta"]{l} .
\end{tikzcd}
\end{center}
\end{definition}

This definition is \emph{categorical}, using diagrams and products.  Recall the product of spaces from the last lecture: its defining property might be better written as \[\CatOf{Spaces}(T, X \times Y) \cong \CatOf{Spaces}(T, X) \times \CatOf{Spaces}(T, Y).\]  That is, $\CatOf{Spaces}(T, -)$ converts products to products.

\begin{definition}
An \define{$H$--space} $K$ is a space with maps $\mu$, $\eta$, $\chi$ satisfying the group diagrams up to homotopy.
\end{definition}

\begin{corollary}
The functor $[-, K]$ is valued in groups.\footnote{$X \xrightarrow\Delta X \times X \xrightarrow{f \times g} K \times K \xrightarrow\mu K$.} \qed
\end{corollary}

Previously, you defined an $H$--space structure on $X^{S^1} =: \Loops X$: two loops can be scaled and concatenated, and loops can be run backward.  Hence, not only is $\pi_0 \Omega X = [S^0, \Omega X]$ a group, but $[Y, \Omega X]$ always is.

What about the other formulation?  We also have $\pi_1 X = [S^1, X]$, and the magic may not rest in the output ``$\Loops X$'' but in $S^1$ alone.

\begin{definition}
An \define{$H$--cogroup} $K$ has pointed $\mu'\co K \to K \vee K$, $\chi'\co K \to K$ so that the diagrams
\begin{center}
\begin{tikzcd}
K \vee K \vee K \arrow["\mu' \vee \id", leftarrow]{r} \arrow["\id \vee \mu'", leftarrow]{d} & K \vee K \arrow["\mu'", leftarrow]{d} \\
K \vee K \arrow["\mu'", leftarrow]{r} & K,
\end{tikzcd}
\begin{tikzcd}
K \arrow[equal]{rd} \arrow["0 \vee \id", leftarrow]{r} & K \vee K \arrow["\mu'", leftarrow]{d} & K \arrow["\id \vee 0"', leftarrow]{l} \arrow[equal]{ld} \\
& K,
\end{tikzcd}
\begin{tikzcd}
K \arrow[leftarrow]{d} \arrow["\chi \vee \id", leftarrow]{r} & K \vee K \arrow["\mu", leftarrow]{d} & K \arrow["\id \vee \chi"', leftarrow]{l} \arrow[leftarrow]{d} \\
* \arrow[leftarrow]{r} & K & * \arrow[leftarrow]{l} .
\end{tikzcd}
\end{center}
commute up to homotopy.
\end{definition}

Again, \[\CatOf{Spaces}_*(X \vee Y, T) = \CatOf{Spaces}(X, T) \times \CatOf{Spaces}(Y, T),\] so $\CatOf{Spaces}(K, -)$ is naturally group-valued.

\begin{example}
$S^1$ is an $H$--cogroup.\todo{Draw pictures of $\mu'$, $\chi'$.}
\end{example}

\begin{example}
In fact, $S^1 \sm X =: \Susp X$ is an $H$--cogroup for \emph{any} $X$.\todo{Draw pictures of $\mu'$, $\chi'$.}\todo{Discuss Yoneda?}
\end{example}

\begin{lemma}
The adjunction $[\Susp X, Y] \cong [X, \Loops Y]$ is an isomorphism of groups.
\end{lemma}
\begin{proof}
This is a matter of writing out the formulas for \[\Susp X \xrightarrow{\mu'} \Susp X \vee \Susp X \xrightarrow{f' \vee g'} Y \vee Y \xrightarrow{\Delta'} Y\] and \[X \xrightarrow\Delta X \times X \xrightarrow{f \times g} \Loops Y \times \Loops Y \xrightarrow\mu Y. \qed\]
\end{proof}




\section{Higher homotopy groups (??)}

Higher homotopy groups have a similar bunch of definitions.

\begin{definition}
$\pi_n X = [\Susp^n(S^0), X] = [\Susp^{n-1} S^0, \Loops X] = \cdots = [S^0, \Loops^n X]$.
\end{definition}

In the middle stages, there are two multiplications on the homotopy mapping set, coming from $\Susp$ and $\Loops$ both.

\begin{lemma}[Eckmann--Hilton]
Let $S$ be a set with two products $\circ$ and $*$ which share a unit and which satisfy \[(x * x') \circ (y * y') = (x \circ y) * (x' \circ y').\]  Then $\circ = *$ and both are associative and commutative.
\end{lemma}
\begin{proof}
\begin{align*}
x \circ y & = 
\end{align*}
\end{proof}




\section{Exact sequences in $\CatOf{Spaces}$}




\section{Relative homotopy groups}




\section{The action of $\pi_1$}




\section{Fibrations}




\section{Fiber bundles and examples}




\section{CW complexes}




\section{The homotopy theory of CW complexes I}




\section{The homotopy theory of CW complexes II}




\section{Brown representability}




\section{Spectra}




\section{Co/homology theories from spectra}




\section{Spectral sequences}
\todo[inline]{The Serre classes section was super thin: basically a list of unproven properties. It would be nice to prove something about arithmetic localizations instead.}




\section{Obstruction theory}




\section{A complicated example}




\section{Smash products}




\section{The Serre spectral sequence}




\section{Some cohomological computations}




\section{$G$--bundles and fiber bundles}
\todo[inline]{I skipped the bar spectral sequence stuff in class because no one knew what $\Tor$ was, which in turn makes the bar construction and $G$--bundle stuff less relevant.}




\section{Properties of Chern classes}




\section{The bar construction}




\section{The Steenrod algebra: calculation}




\section{The Steenrod algebra: interaction with the Serre spectral sequence}




\section{Serre classes}




\section{Serre's method}




\section{The Adams spectral sequence}
\todo[inline]{I have a daydream of explaining how this arises from a limiting process in Serre's method.  I'd \emph{really} like to work this out.}




\section{Hopf invariants and EHP fiber sequences}




\section{Calculations in the EHP spectral sequence}




\section{The May spectral sequence}




\section{Where to go from here}





\end{document}
